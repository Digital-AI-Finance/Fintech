% L01_core.tex -- Lecture 1: Fintech Foundations and Overview (Core Variant)
% Frames: 10 | Extraction: frame-index algorithm from L01_full.tex
% Frames extracted: 1, 4, 6, 10, 14, 18, 21, 24, 26, 30
% Generated for: Financial Technology (FinTech) -- MSc Course, Spring 2026
\documentclass[aspectratio=169, 11pt]{beamer}
\usetheme{Madrid}
\usecolortheme{whale}
\usepackage{graphicx,booktabs,tikz,pgfplots,amsmath,hyperref,multicol,xcolor}
\usepackage[T1]{fontenc}
\usepackage[utf8]{inputenc}
\usetikzlibrary{arrows.meta,positioning,shapes.geometric,calc,decorations.pathmorphing}
\pgfplotsset{compat=1.18}
\definecolor{MLPURPLE}{HTML}{9467BD}
\definecolor{MLBLUE}{HTML}{1F77B4}
\definecolor{MLRED}{HTML}{D62728}
\definecolor{MLORANGE}{HTML}{FF7F0E}
\definecolor{MLGREEN}{HTML}{2CA02C}
\definecolor{MLGRAY}{HTML}{7F7F7F}
\definecolor{MLTEAL}{HTML}{0D7377}
\definecolor{MLCYAN}{HTML}{14BDEB}
\colorlet{mlpurple}{MLPURPLE}
\colorlet{mlblue}{MLBLUE}
\colorlet{mlred}{MLRED}
\colorlet{mlorange}{MLORANGE}
\colorlet{mlgreen}{MLGREEN}
\colorlet{mlgray}{MLGRAY}
\colorlet{mlteal}{MLTEAL}
\colorlet{mlcyan}{MLCYAN}
\setbeamercolor{structure}{fg=MLTEAL}
\setbeamercolor{palette primary}{bg=MLTEAL,fg=white}
\setbeamercolor{frametitle}{bg=MLTEAL!10,fg=MLTEAL}
\setbeamercolor{block title}{bg=MLTEAL,fg=white}
\newcommand{\bottomnote}[1]{\vfill\begin{beamercolorbox}[wd=\textwidth,ht=2ex,dp=1ex]{palette primary}\tiny\hspace{1em}#1\end{beamercolorbox}}
\setbeamertemplate{navigation symbols}{}
\graphicspath{{figures/}}
\title{Fintech Foundations and Overview}
\subtitle{Core Concepts (10 Slides)}
\author{Joerg Osterrieder}
\institute{University of Zurich}
\date{Spring 2026}

\begin{document}

% --- Frame 1: Title Page ---
\begin{frame}{Title Page}
  \titlepage
\end{frame}

% --- Frame 4: Bridge / Welcome ---
\begin{frame}{Welcome to Financial Technology}
  \begin{columns}[T]
    \begin{column}{0.55\textwidth}
      Welcome to Financial Technology.

      \vspace{0.5em}
      This course examines how technology is transforming every corner of
      financial services --- from payments and lending to insurance and wealth
      management.

      \vspace{0.5em}
      This first lecture establishes the foundation:
      \begin{itemize}
        \item What fintech \textbf{is}
        \item Where it \textbf{came from}
        \item Where it is \textbf{going}
      \end{itemize}
    \end{column}
    \begin{column}{0.42\textwidth}
      \includegraphics[width=\textwidth]{figures/04_fintech_ecosystem_overview/chart.pdf}
    \end{column}
  \end{columns}
  \bottomnote{By the end of this lecture you will have a framework for understanding every topic in the rest of the course.}
\end{frame}

% --- Frame 6: What Is Fintech? ---
\begin{frame}{What Is Fintech? Definitions Across Perspectives}
  \begin{columns}[T]
    \begin{column}{0.55\textwidth}
      \begin{tabular}{@{}l p{4.8cm}@{}}
        \toprule
        \textbf{Perspective} & \textbf{Definition Focus} \\
        \midrule
        Academic   & Technology-enabled financial innovation \\
        Industry   & Companies using tech to improve financial services \\
        Regulatory & New entrants requiring new oversight frameworks \\
        Consumer   & Faster, cheaper, more accessible financial products \\
        \bottomrule
      \end{tabular}
    \end{column}
    \begin{column}{0.42\textwidth}
      Notice: every definition emphasizes a different stakeholder.

      \vspace{0.4em}
      Academics see \textit{innovation}. Industry sees \textit{competition}.
      Regulators see \textit{risk}. Consumers see \textit{convenience}.

      \vspace{0.4em}
      \textcolor{mlpurple}{Fintech is all four simultaneously.}
    \end{column}
  \end{columns}
  \vspace{0.5em}
  \begin{block}{Working Definition}
    Fintech is not a product --- it is a force that reshapes how financial
    services are created, delivered, and consumed.
  \end{block}
  \bottomnote{The term `fintech' was first used in the early 1990s but gained mainstream adoption after 2010.}
\end{frame}

% --- Frame 10: Before 2008 ---
\begin{frame}{Before 2008 --- The Trust Assumption}
  \begin{columns}[T]
    \begin{column}{0.50\textwidth}
      Before the crisis, the banking landscape was stable and predictable:

      \vspace{0.5em}
      \begin{itemize}
        \item High trust in institutions
        \item Limited alternatives for consumers
        \item Regulatory frameworks designed for incumbents
        \item Innovation happened \textit{inside} banks, not outside
      \end{itemize}
    \end{column}
    \begin{column}{0.47\textwidth}
      \vspace{0.5em}
      Banks were the only game in town. Consumers trusted them by default.
      Regulation protected them from competition. Innovation meant a new
      savings product, not a new business model.

      \vspace{0.5em}
      \begin{alertblock}{Key Insight}
        Trust in banks was not earned --- it was \textbf{assumed}. The crisis
        exposed the assumption.
      \end{alertblock}
    \end{column}
  \end{columns}
  \bottomnote{In 2007, over 80\% of consumers in developed markets expressed high trust in their primary bank.}
\end{frame}

% --- Frame 14: The Collaboration Spectrum ---
\begin{frame}{The Collaboration Spectrum --- From Competition to Partnership}
  \begin{center}
    \includegraphics[width=0.82\textwidth]{figures/03_collaboration_models_matrix/chart.pdf}
  \end{center}
  \vspace{-0.3em}
  \begin{itemize}
    \item \textbf{What you see:} Four models for how banks and fintechs work
          together, scored across five dimensions.
    \item \textbf{Key pattern:} No single model dominates. Partnerships offer
          speed but less control. Acquisitions offer control but are expensive
          and slow.
    \item \textbf{Takeaway:} The ``right'' model depends on the bank's strategic
          priorities and the fintech's maturity.
  \end{itemize}
  \bottomnote{Most large banks use multiple models simultaneously --- partnering in payments, acquiring in lending, building white-label for compliance.}
\end{frame}

% --- Frame 18: When Fintech Fails ---
\begin{frame}{When Fintech Fails --- Common Failure Modes}
  \begin{enumerate}
    \item \textcolor{mlred}{\textbf{Regulatory risk}} --- Operating without
          adequate licenses; crossing jurisdictional boundaries without
          authorization.
    \item \textcolor{mlred}{\textbf{Trust risk}} --- Data breaches; lack of
          deposit insurance; unclear complaint resolution pathways.
    \item \textcolor{mlred}{\textbf{Scalability risk}} --- Customer acquisition
          costs exceed lifetime value; unit economics never work at scale.
    \item \textcolor{mlred}{\textbf{Systemic risk}} --- Fintech becomes ``too
          connected to fail''; concentration in a small number of cloud
          providers.
  \end{enumerate}
  \vspace{0.3em}
  \begin{alertblock}{The Bigger Question}
    Fintech disruption is not risk-free. The question is whether fintech creates
    \textit{new} risks or merely redistributes old ones.
  \end{alertblock}
  \bottomnote{See L04 (RegTech and Fintech Regulation) for detailed analysis of regulatory failure modes.}
\end{frame}

% --- Frame 21: Regional Patterns ---
\begin{frame}{Fintech Around the World --- Regional Patterns}
  \begin{center}
    \includegraphics[width=0.80\textwidth]{figures/09_fintech_impact_comparison/chart.pdf}
  \end{center}
  \vspace{-0.3em}
  \begin{itemize}
    \item \textbf{What you see:} Fintech adoption varies dramatically by region,
          with Asia-Pacific and Africa leading in mobile payments.
    \item \textbf{Key pattern:} Regions with weaker traditional banking
          infrastructure often \textcolor{mlpurple}{leapfrog} to fintech --- the
          ``leapfrog effect.''
    \item \textbf{Takeaway:} The most transformative fintech innovations
          (M-Pesa, Alipay, PIX) emerged \textit{outside} the US and Europe.
  \end{itemize}
  \bottomnote{Data is illustrative of broad adoption patterns. Actual metrics depend on definition and measurement methodology.}
\end{frame}

% --- Frame 24: Stakeholder Impact Analysis ---
\begin{frame}{Stakeholder Impact Analysis}
  \begin{columns}[T]
    \begin{column}{0.45\textwidth}
      \includegraphics[width=\textwidth]{figures/08_embedded_finance_architecture/chart.pdf}
    \end{column}
    \begin{column}{0.52\textwidth}
      \begin{itemize}
        \item \textbf{Consumers:} More choice, lower fees --- but less
              protection
        \item \textbf{Banks:} Competitive pressure, forced innovation,
              potential disintermediation
        \item \textbf{Regulators:} New oversight challenges, innovation vs.\
              stability trade-off
        \item \textbf{Fintechs:} Growth opportunity --- but funding cycles and
              regulatory uncertainty
        \item \textbf{Society:} Financial inclusion gains --- but digital
              divide risks
      \end{itemize}
    \end{column}
  \end{columns}
  \vspace{0.3em}
  \begin{block}{Not Zero-Sum}
    Fintech is not zero-sum. Both consumers and institutions can benefit --- but
    the benefits are \textcolor{mlpurple}{not evenly distributed}.
  \end{block}
  \bottomnote{Financial inclusion --- serving the unbanked and underbanked --- is examined in detail in L02 (Fintech Ecosystem).}
\end{frame}

% --- Frame 26: Five Questions Framework ---
\begin{frame}{Five Questions That Reveal Any Fintech's True Strategy}
  \begin{columns}[T]
    \begin{column}{0.55\textwidth}
      \begin{enumerate}
        \item \textbf{Who is the customer?}\\
              Consumer, SME, enterprise, or another fintech?
        \item \textbf{What part of the value chain does it attack?}\\
              Origination, distribution, servicing, or infrastructure?
        \item \textbf{How does it make money?}\\
              Transaction fees, subscription, data monetization, or float?
        \item \textbf{What is its regulatory position?}\\
              Licensed, partnered, or operating in a gap?
        \item \textbf{Does it create or capture value?}\\
              Building new markets or taking share from incumbents?
      \end{enumerate}
    \end{column}
    \begin{column}{0.42\textwidth}
      \includegraphics[width=\textwidth]{figures/10_key_concepts_summary/chart.pdf}
    \end{column}
  \end{columns}
  \vspace{0.2em}
  \begin{block}{A Universal Tool}
    These five questions work for any fintech company you encounter --- in this
    course, in the news, or in your career.
  \end{block}
  \bottomnote{Apply these questions to a fintech you use. You will use this framework in the Workshop~C evaluation exercise on Day~5.}
\end{frame}

% --- Frame 30: Key Takeaways ---
\begin{frame}{Key Takeaways}
  \begin{enumerate}
    \item \textbf{Fintech defined:} Technology-enabled innovation that creates
          new financial products, processes, or business models.
    \item \textbf{Historical arc:} From credit cards (1950s) through online
          banking (1990s) to embedded finance (2020s) --- each wave built on the
          last.
    \item \textbf{Crisis catalyst:} The 2008 financial crisis eroded trust,
          opened regulatory space, and released talent --- creating the
          conditions for fintech's explosive growth.
    \item \textbf{Unbundling:} Fintech companies attack specific layers of the
          banking value chain, not the entire bank.
    \item \textbf{Collaboration spectrum:} Banks and fintechs interact through
          partnership, acquisition, white-label, and open banking --- each with
          distinct trade-offs.
    \item \textbf{Global variation:} Fintech adoption is highest where
          traditional banking infrastructure is weakest (the leapfrog effect).
    \item \textbf{Evaluation tool:} Five questions (customer, value chain,
          revenue model, regulatory position, value creation) reveal any
          fintech's true strategy.
  \end{enumerate}
  \bottomnote{Review question: Which collaboration model would you recommend for a mid-sized European bank entering mobile payments? Why?}
\end{frame}

\end{document}
