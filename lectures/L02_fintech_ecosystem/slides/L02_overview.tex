% L02_overview.tex -- Lecture 2: Fintech Ecosystem (Overview Variant)
% Frames: ~27 | Charts: subset of 12 | Architecture: INTRO/CORE/CLOSING
% Framework: PMSP (Problem -- Method -- Solution -- Practice)
% Audience: MSc Finance/Business, no coding assumed
\documentclass[aspectratio=169, 11pt]{beamer}

% ============================================================
% THEME BASE
% ============================================================
\usetheme{Madrid}
\usecolortheme{whale}

% ============================================================
% PACKAGES
% ============================================================
\usepackage[T1]{fontenc}
\usepackage[utf8]{inputenc}
\usepackage{graphicx}
\usepackage{booktabs}
\usepackage{tikz}
\usepackage{pgfplots}
\usepackage{amsmath}
\usepackage{hyperref}
\usepackage{multicol}
\usepackage{xcolor}

% ============================================================
% TIKZ LIBRARIES
% ============================================================
\usetikzlibrary{arrows.meta, positioning, shapes.geometric, calc, decorations.pathmorphing}

% ============================================================
% PGFPLOTS COMPATIBILITY
% ============================================================
\pgfplotsset{compat=1.18}

% ============================================================
% COLOR DEFINITIONS (Fintech V4 Palette)
% ============================================================
\definecolor{MLPURPLE}{HTML}{9467BD}
\definecolor{MLBLUE}{HTML}{1F77B4}
\definecolor{MLRED}{HTML}{D62728}
\definecolor{MLORANGE}{HTML}{FF7F0E}
\definecolor{MLGREEN}{HTML}{2CA02C}
\definecolor{MLGRAY}{HTML}{7F7F7F}
\definecolor{MLTEAL}{HTML}{0D7377}
\definecolor{MLCYAN}{HTML}{14BDEB}

% Lowercase aliases for use in \textcolor{}
\colorlet{mlpurple}{MLPURPLE}
\colorlet{mlblue}{MLBLUE}
\colorlet{mlred}{MLRED}
\colorlet{mlorange}{MLORANGE}
\colorlet{mlgreen}{MLGREEN}
\colorlet{mlgray}{MLGRAY}
\colorlet{mlteal}{MLTEAL}
\colorlet{mlcyan}{MLCYAN}

% ============================================================
% BEAMER COLOR CUSTOMIZATION
% ============================================================
\setbeamercolor{structure}{fg=MLTEAL}
\setbeamercolor{palette primary}{bg=MLTEAL, fg=white}
\setbeamercolor{palette secondary}{bg=MLTEAL!80, fg=white}
\setbeamercolor{palette tertiary}{bg=MLTEAL!60, fg=white}
\setbeamercolor{palette quaternary}{bg=MLTEAL!40, fg=white}
\setbeamercolor{frametitle}{bg=MLTEAL!10, fg=MLTEAL}
\setbeamercolor{frametitle right}{bg=MLTEAL!5}
\setbeamercolor{block title}{bg=MLTEAL, fg=white}
\setbeamercolor{block body}{bg=MLTEAL!8, fg=black}
\setbeamercolor{block title alerted}{bg=MLRED, fg=white}
\setbeamercolor{block body alerted}{bg=MLRED!8, fg=black}
\setbeamercolor{block title example}{bg=MLGREEN, fg=white}
\setbeamercolor{block body example}{bg=MLGREEN!8, fg=black}
\setbeamercolor{title}{fg=white}
\setbeamercolor{subtitle}{fg=MLCYAN}
\setbeamercolor{author}{fg=white}
\setbeamercolor{institute}{fg=white}
\setbeamercolor{date}{fg=white}
\setbeamercolor{title page header}{bg=MLTEAL}

% ============================================================
% NAVIGATION AND FOOTLINE
% ============================================================
\setbeamertemplate{navigation symbols}{}

\setbeamertemplate{footline}{%
  \leavevmode%
  \hbox{%
    \begin{beamercolorbox}[wd=.333\paperwidth, ht=2.25ex, dp=1ex, center]{palette primary}%
      \usebeamerfont{author in head/foot}\insertshortauthor
    \end{beamercolorbox}%
    \begin{beamercolorbox}[wd=.334\paperwidth, ht=2.25ex, dp=1ex, center]{palette secondary}%
      \usebeamerfont{title in head/foot}\insertshorttitle
    \end{beamercolorbox}%
    \begin{beamercolorbox}[wd=.333\paperwidth, ht=2.25ex, dp=1ex, right]{palette tertiary}%
      \usebeamerfont{date in head/foot}%
      \insertframenumber{} / \inserttotalframenumber\hspace*{2ex}
    \end{beamercolorbox}%
  }%
  \vskip0pt%
}

% ============================================================
% BOTTOM NOTE COMMAND
% ============================================================
\newcommand{\bottomnote}[1]{%
  \vfill
  \begin{beamercolorbox}[wd=\textwidth, ht=2ex, dp=1ex]{palette primary}%
    \tiny\hspace{1em}#1
  \end{beamercolorbox}%
}

% ============================================================
% GRAPHICS PATH
% ============================================================
\graphicspath{{figures/}}

% ============================================================
% COURSE METADATA
% ============================================================
\title{Financial Technology (FinTech)}
\author{Joerg Osterrieder}
\institute{University of Zurich \\ Department of Finance}
\date{Spring 2026}


\subtitle{Growth, Social Impact, and Behavioral Dimensions}

% Short versions for footline
\title[Fintech: Ecosystem]{Financial Technology (FinTech) -- Lecture 2}
\author[J.\ Osterrieder]{Joerg Osterrieder}

\begin{document}

% =============================================
%   INTRO ZONE
%   Frames 1-4 -- Open, motivate, orient
% =============================================

\section{Introduction}

% ---------------------------------------------------------
% Frame 1: Title Page
% ---------------------------------------------------------
\begin{frame}
  \titlepage
  \bottomnote{Lecture 2 of 7 $\cdot$ Financial Technology (FinTech) $\cdot$ MSc Programme $\cdot$ Spring 2026}
\end{frame}

% ---------------------------------------------------------
% Frame 2: Opening Cartoon
% ---------------------------------------------------------
\begin{frame}{A Billion Reasons to Change}
  \begin{center}
    \includegraphics[width=0.90\textwidth]{figures/11_opening_cartoon/cartoon.pdf}
  \end{center}
  \bottomnote{The ecosystem question: who grows, who gets left behind, and who decides the rules?}
\end{frame}

% ---------------------------------------------------------
% Frame 3: Learning Objectives
% ---------------------------------------------------------
\begin{frame}{Learning Objectives}
  \begin{enumerate}
    \item \textbf{Identify} the key drivers of fintech ecosystem growth and explain
          why adoption rates differ across demographics.
          \textcolor{mlgray}{\small[Understand]}
    \item \textbf{Explain} the role of trust, perceived risk, and social influence
          in shaping consumer adoption of digital financial services.
          \textcolor{mlgray}{\small[Understand]}
    \item \textbf{Apply} the Technology Adoption Lifecycle to predict where a given
          fintech product sits on the adoption curve.
          \textcolor{mlgray}{\small[Apply]}
    \item \textbf{Analyse} the mechanisms of choice architecture and nudging as
          tools for influencing financial behaviour.
          \textcolor{mlgray}{\small[Analyse]}
    \item \textbf{Evaluate} the ethical trade-offs between inclusion-promoting nudges
          and manipulative dark patterns.
          \textcolor{mlgray}{\small[Evaluate]}
  \end{enumerate}
  \vspace{0.4em}
  \textcolor{mlpurple}{\textbf{Bloom's levels covered:}} Understand $\to$ Apply $\to$ Analyse $\to$ Evaluate
  \bottomnote{These objectives map directly to the quiz and workshop assessments for this lecture.}
\end{frame}

% ---------------------------------------------------------
% Frame 4: Bridge from L01
% ---------------------------------------------------------
\begin{frame}{Building on L01 -- From Foundations to Ecosystem}
  \begin{columns}[T]
    \begin{column}{0.54\textwidth}
      \textcolor{mlpurple}{\textbf{Where we left off (L01):}}
      \begin{itemize}
        \item Fintech unbundles the banking value chain
        \item The 2008 crisis opened the door
        \item Four collaboration models: partnership, acquisition, white-label, open banking
      \end{itemize}
      \vspace{0.5em}
      \textcolor{mlblue}{\textbf{Where we go today (L02):}}
      \begin{itemize}
        \item \emph{Who} is adopting fintech, and \emph{why}?
        \item What \textbf{trust barriers} slow adoption?
        \item How can \textbf{design choices} accelerate or retard inclusion?
      \end{itemize}
    \end{column}
    \begin{column}{0.42\textwidth}
      \includegraphics[width=\textwidth]{figures/01_fintech_ecosystem_map/chart.pdf}
    \end{column}
  \end{columns}
  \bottomnote{The ecosystem map gives us the full picture. Today we focus on the adoption and behaviour layers.}
\end{frame}

% =============================================
%   CORE ZONE -- PROBLEM
%   Frames 5-8 -- Growth drivers, economic benefits, inclusion gap
% =============================================

\section{Problem: Why Does Fintech Adoption Matter?}

% ---------------------------------------------------------
% Frame 5: Growth Drivers Dashboard
% ---------------------------------------------------------
\begin{frame}{What Is Driving the Fintech Explosion?}
  \begin{center}
    \includegraphics[width=0.88\textwidth]{figures/02_growth_drivers_dashboard/chart.pdf}
  \end{center}
  \vspace{-0.3em}
  \begin{itemize}
    \item Four forces reinforce each other: smartphone penetration, cheap cloud
          infrastructure, eroded bank trust, and post-2008 regulatory openings
    \item \textcolor{mlpurple}{\textbf{No single driver is sufficient.}} Remove one
          and the ecosystem stalls.
  \end{itemize}
  \bottomnote{Growth figures are illustrative of the trend direction. Absolute values depend on geography and measurement methodology.}
\end{frame}

% ---------------------------------------------------------
% Frame 6: Economic Benefits of Fintech Adoption
% ---------------------------------------------------------
\begin{frame}{The Economic Case for Fintech Adoption}
  \begin{columns}[T]
    \begin{column}{0.47\textwidth}
      \textcolor{mlgreen}{\textbf{For consumers:}}
      \begin{itemize}
        \item Lower transaction costs -- especially for cross-border payments
        \item Real-time access to credit and savings products
        \item Transparent fee structures vs.\ hidden banking fees
        \item Micro-investment entry points for low-income savers
      \end{itemize}
    \end{column}
    \begin{column}{0.47\textwidth}
      \textcolor{mlblue}{\textbf{For economies:}}
      \begin{itemize}
        \item Financial deepening in underserved regions
        \item SME lending unlocked by alternative credit scoring
        \item Remittance cost reduction frees household income
        \item Tax-base expansion via formalisation of informal transactions
      \end{itemize}
    \end{column}
  \end{columns}
  \vspace{0.5em}
  \begin{block}{The Stakes}
    Fintech adoption is not merely a consumer convenience story. At national scale,
    it determines which populations participate in formal economic life.
  \end{block}
  \bottomnote{The World Bank estimates that financial inclusion could raise GDP growth by 1--2 percentage points in emerging markets.}
\end{frame}

% ---------------------------------------------------------
% Frame 7: The Financial Inclusion Gap
% ---------------------------------------------------------
\begin{frame}{The Inclusion Gap -- Who Is Still Left Out?}
  \begin{center}
    \includegraphics[width=0.88\textwidth]{figures/03_financial_inclusion_gap/chart.pdf}
  \end{center}
  \vspace{-0.3em}
  \begin{itemize}
    \item Approximately \textcolor{mlpurple}{\textbf{1.7 billion adults}} remain unbanked --
          disproportionately women, rural populations, and low-income households
    \item Smartphone ownership is now \emph{ahead} of bank account ownership in many
          developing markets -- fintech's structural opportunity
  \end{itemize}
  \bottomnote{Inclusion data is illustrative. World Bank Global Findex provides the authoritative baseline (last published 2021).}
\end{frame}

% ---------------------------------------------------------
% Frame 8: M-Pesa -- The Canonical Inclusion Case
% ---------------------------------------------------------
\begin{frame}{M-Pesa -- Leapfrogging the Branch}
  \begin{columns}[T]
    \begin{column}{0.52\textwidth}
      \includegraphics[width=\textwidth]{figures/04_mpesa_adoption_flow/chart.pdf}
    \end{column}
    \begin{column}{0.44\textwidth}
      \textcolor{mlpurple}{\textbf{Why M-Pesa worked:}}
      \begin{itemize}
        \item Existing distribution via mobile airtime agents
        \item No smartphone required -- feature-phone USSD
        \item Regulatory permission from CBK without banking licence
        \item Network effects: adoption rose with merchant acceptance
      \end{itemize}
      \vspace{0.4em}
      \begin{block}{The Lesson}
        Inclusion succeeds when fintech \emph{meets users where they are}, not
        where the designer assumed they would be.
      \end{block}
    \end{column}
  \end{columns}
  \bottomnote{M-Pesa (Safaricom, Kenya, 2007) is the most extensively studied mobile money deployment. It now processes more transactions than Kenya's formal banking system.}
\end{frame}

% =============================================
%   CORE ZONE -- METHOD
%   Frames 9-12 -- Trust, barriers, lifecycle, demographics
% =============================================

\section{Method: Understanding Trust and Behavior}

% ---------------------------------------------------------
% Frame 9: Trust Framework
% ---------------------------------------------------------
\begin{frame}{Trust in Fintech -- A Three-Dimensional Framework}
  \begin{center}
    \includegraphics[width=0.88\textwidth]{figures/05_trust_framework_comparison/chart.pdf}
  \end{center}
  \vspace{-0.3em}
  \begin{itemize}
    \item \textcolor{mlpurple}{\textbf{Competence trust:}} Can this system execute the transaction reliably?
    \item \textcolor{mlblue}{\textbf{Integrity trust:}} Will it handle my data and money honestly?
    \item \textcolor{mlgreen}{\textbf{Benevolence trust:}} Does the provider act in my interest, not just its own?
  \end{itemize}
  \bottomnote{Research consistently shows that benevolence trust -- the hardest to build -- is the strongest predictor of sustained fintech adoption.}
\end{frame}

% ---------------------------------------------------------
% Frame 10: Behavioral Barriers to Adoption
% ---------------------------------------------------------
\begin{frame}{Why Rational Consumers Still Resist Fintech}
  \begin{columns}[T]
    \begin{column}{0.47\textwidth}
      \textcolor{mlred}{\textbf{Cognitive barriers:}}
      \begin{itemize}
        \item \textbf{Status quo bias} -- default to familiar bank
        \item \textbf{Loss aversion} -- fear of losing money to glitch outweighs gain of convenience
        \item \textbf{Ambiguity aversion} -- unfamiliar products trigger avoidance
        \item \textbf{Present bias} -- security concerns now outweigh future savings
      \end{itemize}
    \end{column}
    \begin{column}{0.47\textwidth}
      \textcolor{mlorange}{\textbf{Structural barriers:}}
      \begin{itemize}
        \item Digital literacy gaps -- especially among elderly users
        \item Language and interface barriers in multilingual markets
        \item Connectivity gaps -- rural 4G/5G exclusion
        \item Identity documentation gaps for KYC compliance
      \end{itemize}
    \end{column}
  \end{columns}
  \vspace{0.5em}
  \begin{alertblock}{}
    Designing only for the \textbf{tech-comfortable early adopter} systematically
    excludes the populations fintech claims to serve.
  \end{alertblock}
  \bottomnote{Behavioral economics distinguishes \emph{stated} preferences (``I would use it'') from \emph{revealed} preferences (actual adoption). The gap is consistently large.}
\end{frame}

% ---------------------------------------------------------
% Frame 11: Technology Adoption Lifecycle
% ---------------------------------------------------------
\begin{frame}{Where Is Your Product? The Adoption Lifecycle}
  \begin{center}
    \includegraphics[width=0.88\textwidth]{figures/06_technology_adoption_lifecycle/chart.pdf}
  \end{center}
  \vspace{-0.3em}
  \begin{itemize}
    \item The \textcolor{mlpurple}{\textbf{Chasm}} between Early Adopters and Early Majority is where most fintech products fail
    \item Crossing it requires \textbf{trust infrastructure}, not just better technology
    \item Different nudge strategies work for different segments of the curve
  \end{itemize}
  \bottomnote{Moore (1991) observed that mass-market products need a complete solution, not just features. Trust is the missing component for most fintech.}
\end{frame}

% ---------------------------------------------------------
% Frame 12: Demographic Adoption Patterns
% ---------------------------------------------------------
\begin{frame}{Who Adopts Fintech -- and When?}
  \begin{columns}[T]
    \begin{column}{0.47\textwidth}
      \textcolor{mlpurple}{\textbf{High-adoption segments:}}
      \begin{itemize}
        \item Millennials and Gen~Z (digital natives)
        \item Urban, high-income, educated users
        \item SME owners seeking faster credit
        \item Migrant workers (remittance need)
      \end{itemize}
    \end{column}
    \begin{column}{0.47\textwidth}
      \textcolor{mlgray}{\textbf{Low-adoption segments:}}
      \begin{itemize}
        \item Elderly and low-digital-literacy users
        \item Rural populations with connectivity gaps
        \item Low-income households with trust barriers
        \item Users in strict data-privacy cultures
      \end{itemize}
    \end{column}
  \end{columns}
  \vspace{0.5em}
  \begin{block}{The Paradox}
    Fintech adoption is highest among those who \emph{already have} good access to
    financial services. Reaching the underserved requires \textcolor{mlpurple}{\textbf{deliberate design}},
    not market forces alone.
  \end{block}
  \bottomnote{EY Global FinTech Adoption Index and World Bank Global Findex provide the most consistent cross-country adoption data.}
\end{frame}

% =============================================
%   CORE ZONE -- SOLUTION
%   Frames 13-16 -- Choice architecture, nudges, dark patterns, ethics
% =============================================

\section{Solution: Choice Architecture and Nudging}

% ---------------------------------------------------------
% Frame 13: What Is Choice Architecture?
% ---------------------------------------------------------
\begin{frame}{Choice Architecture -- Designing the Decision Environment}
  \begin{columns}[T]
    \begin{column}{0.52\textwidth}
      \textcolor{mlpurple}{\textbf{The core insight (Thaler \& Sunstein):}}
      \vspace{0.4em}

      \emph{Every financial interface is already a choice architecture.}
      The only question is whether it was designed deliberately or accidentally.

      \vspace{0.5em}
      \begin{itemize}
        \item Default settings shape the majority of outcomes
        \item Option ordering changes selection rates
        \item Framing (``save \$10'' vs.\ ``lose \$10'') changes decisions
        \item Feedback timing influences next-period behaviour
      \end{itemize}
    \end{column}
    \begin{column}{0.44\textwidth}
      \includegraphics[width=\textwidth]{figures/09_choice_architecture_examples/chart.pdf}
    \end{column}
  \end{columns}
  \bottomnote{Choice architecture does not restrict options -- it arranges them. The libertarian paternalism principle: preserve freedom while nudging toward better outcomes.}
\end{frame}

% ---------------------------------------------------------
% Frame 14: Five Fintech Nudges
% ---------------------------------------------------------
\begin{frame}{Five Nudges That Work in Fintech}
  \begin{center}
    \includegraphics[width=0.88\textwidth]{figures/08_nudging_architecture/chart.pdf}
  \end{center}
  \vspace{-0.3em}
  \begin{columns}[T]
    \begin{column}{0.32\textwidth}
      \textcolor{mlpurple}{\textbf{1. Smart defaults}}\\
      Opt-in to save; opt-out to stop
    \end{column}
    \begin{column}{0.32\textwidth}
      \textcolor{mlblue}{\textbf{2. Commitment devices}}\\
      Lock savings for 90~days
    \end{column}
    \begin{column}{0.32\textwidth}
      \textcolor{mlgreen}{\textbf{3. Social norms}}\\
      ``80\% of users your age save monthly''
    \end{column}
  \end{columns}
  \vspace{0.3em}
  \begin{columns}[T]
    \begin{column}{0.32\textwidth}
      \textcolor{mlorange}{\textbf{4. Friction reduction}}\\
      One-tap recurring transfer
    \end{column}
    \begin{column}{0.32\textwidth}
      \textcolor{mlteal}{\textbf{5. Progress feedback}}\\
      Visual savings-goal tracker
    \end{column}
    \begin{column}{0.32\textwidth}
      \small Each addresses a specific cognitive barrier from Frame 10.
    \end{column}
  \end{columns}
  \bottomnote{Evidence from field experiments consistently shows that smart defaults produce the largest effect sizes of any low-cost intervention.}
\end{frame}

% ---------------------------------------------------------
% Frame 15: Dark Patterns -- When Nudging Becomes Manipulation
% ---------------------------------------------------------
\begin{frame}{Dark Patterns -- The Shadow Side of Choice Architecture}
  \begin{columns}[T]
    \begin{column}{0.47\textwidth}
      \textcolor{mlgreen}{\textbf{Legitimate nudge:}}
      \begin{itemize}
        \item Default to pension contribution at hire
        \item Prompt to review spending before large purchase
        \item Social comparison using \emph{opt-in} peer data
        \item Clear, one-click unsubscribe path
      \end{itemize}
    \end{column}
    \begin{column}{0.47\textwidth}
      \textcolor{mlred}{\textbf{Dark pattern:}}
      \begin{itemize}
        \item Pre-ticked boxes for unwanted insurance add-ons
        \item Artificial urgency: ``Offer expires in 4 minutes!''
        \item Confirmation shaming: ``No thanks, I prefer high fees''
        \item Hidden unsubscribe buried in 7-click settings flow
      \end{itemize}
    \end{column}
  \end{columns}
  \vspace{0.5em}
  \begin{alertblock}{The Distinction}
    A nudge \textcolor{mlgreen}{\textbf{helps}} the user reach their own stated goals.
    A dark pattern \textcolor{mlred}{\textbf{exploits}} cognitive weaknesses to serve the
    platform's goals at the user's expense.
  \end{alertblock}
  \bottomnote{GDPR (EU), Consumer Duty (UK, 2023), and CFPB guidance (US) increasingly codify dark-pattern prohibitions into law.}
\end{frame}

% ---------------------------------------------------------
% Frame 16: Ethical Choice Architecture Checklist
% ---------------------------------------------------------
\begin{frame}{Evaluating a Nudge -- Four Ethical Questions}
  \begin{enumerate}
    \item \textcolor{mlpurple}{\textbf{Transparency:}} Is the nudge disclosed to the user?\\
          \small Can the user see what default has been set on their behalf?
    \vspace{0.4em}
    \item \textcolor{mlblue}{\textbf{Alignment:}} Does the nudge serve the user's \emph{own} stated goals?\\
          \small Or does it serve the platform's revenue targets?
    \vspace{0.4em}
    \item \textcolor{mlgreen}{\textbf{Opt-out ease:}} Can the user change or reverse the nudged outcome easily?\\
          \small One click to undo must be as easy as one click to accept.
    \vspace{0.4em}
    \item \textcolor{mlorange}{\textbf{Equity:}} Does the nudge function equally across literacy levels?\\
          \small A nudge that works only for numerically sophisticated users excludes the vulnerable.
  \end{enumerate}
  \vspace{0.3em}
  \begin{block}{}
    Apply these four questions to \emph{any} fintech interface feature.
    If any answer is ``no'', the design requires revision.
  \end{block}
  \bottomnote{This checklist is a condensed version of the FCA's Consumer Duty outcomes framework (2023) applied to behavioural design.}
\end{frame}

% =============================================
%   CORE ZONE -- PRACTICE
%   Frames 17-20 -- Stakeholders, success/failure, national nudging, trade-off
% =============================================

\section{Practice: Evidence and Evaluation}

% ---------------------------------------------------------
% Frame 17: Ecosystem Stakeholder Map
% ---------------------------------------------------------
\begin{frame}{The Fintech Ecosystem -- Who Holds the Power?}
  \begin{center}
    \includegraphics[width=0.88\textwidth]{figures/10_ecosystem_stakeholder_impact/chart.pdf}
  \end{center}
  \vspace{-0.3em}
  \begin{itemize}
    \item Four stakeholder rings: \textcolor{mlpurple}{Regulators}, \textcolor{mlblue}{Incumbents},
          \textcolor{mlgreen}{Fintechs}, \textcolor{mlorange}{Users}
    \item Each ring can accelerate or block adoption -- simultaneously
    \item The ecosystem is healthy only when all four rings are aligned
  \end{itemize}
  \bottomnote{Ecosystem analysis is a tool for identifying where friction originates. Solving the wrong ring is the most common strategy error.}
\end{frame}

% ---------------------------------------------------------
% Frame 18: Success and Failure Stories
% ---------------------------------------------------------
\begin{frame}{Adoption Success and Failure -- Two Contrasting Patterns}
  \begin{columns}[T]
    \begin{column}{0.47\textwidth}
      \textcolor{mlgreen}{\textbf{What success looks like:}}
      \vspace{0.3em}
      \begin{itemize}
        \item \textbf{Strong default design} reduces friction at onboarding
        \item \textbf{Social proof} from peer adoption accelerates diffusion
        \item \textbf{Agent networks} bridge digital literacy gaps (M-Pesa model)
        \item \textbf{Regulatory sandbox} allows iteration without full licence burden
      \end{itemize}
    \end{column}
    \begin{column}{0.47\textwidth}
      \textcolor{mlred}{\textbf{What failure looks like:}}
      \vspace{0.3em}
      \begin{itemize}
        \item Onboarding requires 12+ steps and a printer
        \item No fallback for low-connectivity users
        \item Customer support is chatbot-only with no escalation
        \item Data breach destroys competence trust overnight
        \item Regulatory crackdown kills network effects mid-scale
      \end{itemize}
    \end{column}
  \end{columns}
  \vspace{0.4em}
  \begin{block}{}
    Most adoption failures are \textcolor{mlpurple}{\textbf{design failures}}, not technology failures.
    The product worked; the interface excluded.
  \end{block}
  \bottomnote{Post-mortems of failed digital wallet deployments consistently identify onboarding complexity and trust failure as the top two root causes.}
\end{frame}

% ---------------------------------------------------------
% Frame 19: National-Scale Nudging
% ---------------------------------------------------------
\begin{frame}{Nudging at National Scale -- Policy as Choice Architecture}
  \begin{columns}[T]
    \begin{column}{0.55\textwidth}
      \textcolor{mlpurple}{\textbf{Governments as choice architects:}}
      \vspace{0.4em}
      \begin{itemize}
        \item \textbf{Auto-enrolment pensions} (UK, 2012): participation rose from
              55\% to over 85\% within five years of default-on enrolment
        \item \textbf{PIX (Brazil, 2020):} instant payment default built into all
              bank accounts; 100~million users in first year
        \item \textbf{India Account Aggregator:} consent-based data portability as
              regulated default -- borrowers share data, lenders offer tailored credit
        \item \textbf{Regulatory sandbox defaults:} opt-in inclusion of fintechs
              reduces testing cost and accelerates safe innovation
      \end{itemize}
    \end{column}
    \begin{column}{0.41\textwidth}
      \begin{block}{Scale Effect}
        A national default reaches \emph{everyone simultaneously}, including those
        who would never actively choose to adopt.\\[0.5em]
        This makes government choice architecture both the most powerful and the
        most ethically consequential form of nudging.
      \end{block}
    \end{column}
  \end{columns}
  \bottomnote{National defaults differ from commercial nudges in that users cannot easily opt out of an entire payment system. The ethical bar is correspondingly higher.}
\end{frame}

% ---------------------------------------------------------
% Frame 20: Inclusion vs. Protection Trade-Off
% ---------------------------------------------------------
\begin{frame}{The Central Trade-Off -- Inclusion vs.\ Protection}
  \begin{center}
    \Large
    \textcolor{mlteal}{\textbf{``Every design choice that eases access also eases misuse.''}}
  \end{center}
  \vspace{0.5em}
  \begin{columns}[T]
    \begin{column}{0.47\textwidth}
      \textcolor{mlgreen}{\textbf{Simplification helps inclusion:}}
      \begin{itemize}
        \item Reduced KYC friction onboards the unbanked
        \item Simplified terms reach low-literacy users
        \item Auto-approve credit reaches thin-file borrowers
      \end{itemize}
    \end{column}
    \begin{column}{0.47\textwidth}
      \textcolor{mlred}{\textbf{Simplification creates risk:}}
      \begin{itemize}
        \item Reduced KYC enables money-laundering
        \item Simplified terms obscure fees and penalties
        \item Auto-approve credit enables predatory lending
      \end{itemize}
    \end{column}
  \end{columns}
  \vspace{0.5em}
  \begin{alertblock}{No Free Lunch}
    There is no design choice that simultaneously maximises inclusion and maximises
    protection. Regulators and designers must explicitly \textcolor{mlred}{\textbf{choose a position}}
    on this trade-off and defend it.
  \end{alertblock}
  \bottomnote{The inclusion--protection trade-off is the central regulatory design problem of fintech. See L04 for the regulatory toolkit.}
\end{frame}

% =============================================
%   CORE ZONE -- SYNTHESIS
%   Frames 21-23 -- Evaluation, central tension, what comes next
% =============================================

\section{Synthesis}

% ---------------------------------------------------------
% Frame 21: Evaluation Framework
% ---------------------------------------------------------
\begin{frame}{Evaluating Fintech Ecosystem Health -- Five Indicators}
  \begin{enumerate}
    \item \textcolor{mlpurple}{\textbf{Adoption breadth:}} Are underserved populations included,
          or only the already-served?
    \vspace{0.3em}
    \item \textcolor{mlblue}{\textbf{Trust depth:}} Do users exhibit repeat engagement and
          high-value transaction transfer -- not just first-use?
    \vspace{0.3em}
    \item \textcolor{mlgreen}{\textbf{Barrier profile:}} Are barriers primarily technical,
          behavioural, or structural? Each requires a different intervention.
    \vspace{0.3em}
    \item \textcolor{mlorange}{\textbf{Nudge ethics:}} Are default settings and interface choices
          aligned with user welfare, or with platform revenue?
    \vspace{0.3em}
    \item \textcolor{mlteal}{\textbf{Regulatory alignment:}} Does the ecosystem operate within,
          at the edge of, or outside the regulatory boundary?
  \end{enumerate}
  \vspace{0.2em}
  \begin{block}{}
    These five indicators apply to any fintech market, product, or national ecosystem.
    Use them in the workshop case study.
  \end{block}
  \bottomnote{Apply these indicators to a fintech product you use. You will use this framework in Workshop B on Day 3.}
\end{frame}

% ---------------------------------------------------------
% Frame 22: Central Tension
% ---------------------------------------------------------
\begin{frame}{The Central Tension of the Fintech Ecosystem}
  \begin{center}
    \Large
    \textcolor{mlteal}{\textbf{``Fintech can include everyone -- or optimise for the already-included.\\
    The difference is entirely a matter of design intent.''}}
  \end{center}

  \vspace{0.6em}
  \begin{itemize}
    \item Will adoption curves flatten at 40\% or reach 80\%?
    \item Will nudges serve users or extract from them?
    \item Will trust be built on transparency or manufactured by inertia?
    \item Will regulators set the floor or raise the ceiling?
  \end{itemize}

  \vspace{0.5em}
  \begin{block}{}
    These are not \textbf{technology} questions. They are
    \textcolor{mlpurple}{\textbf{governance and ethics}} questions answered
    through design.
  \end{block}
  \bottomnote{Return to this tension after L04 (Regulation) and L07 (Technology). Each lecture adds a layer to the answer.}
\end{frame}

% ---------------------------------------------------------
% Frame 23: What Comes Next
% ---------------------------------------------------------
\begin{frame}{What Comes Next}
  \begin{itemize}
    \item \textcolor{mlpurple}{\textbf{Next: L03 (Payments and Transactions)}} \\
          Real-time payments, card networks, cross-border rails, CBDC -- where the
          ecosystem's largest transaction volumes actually flow
    \vspace{0.4em}
    \item \textcolor{mlblue}{\textbf{Before L03, reflect:}}
      \begin{itemize}
        \item Which fintech services do you trust with large amounts? Why?
        \item Have you ever changed a default in a financial app? What prompted you?
      \end{itemize}
    \vspace{0.4em}
    \item \textcolor{mlgreen}{\textbf{Workshop B preparation:}}
      Apply the five ecosystem health indicators (Frame 21) to one fintech product
      you use regularly. Bring a two-paragraph evaluation.
  \end{itemize}
  \vspace{0.4em}
  \begin{block}{Course Arc}
    L01: Foundations $\to$ \textbf{L02: Ecosystem} $\to$ L03: Payments $\to$ L04: Regulation $\to$
    L05: Wealth $\to$ L06: Insurance $\to$ L07: Technology
  \end{block}
  \bottomnote{All lecture slides and workshop case materials are available on the course website.}
\end{frame}

% =============================================
%   CLOSING ZONE
%   Frames 24-26 -- Cartoon, Takeaways, Summary
% =============================================

\section{Closing}

% ---------------------------------------------------------
% Frame 24: Closing Cartoon
% ---------------------------------------------------------
\begin{frame}{Design Intent Matters}
  \begin{center}
    \includegraphics[width=0.90\textwidth]{figures/12_closing_cartoon/cartoon.pdf}
  \end{center}
  \bottomnote{The ecosystem's ultimate shape is determined by design intent -- not by technology alone.}
\end{frame}

% ---------------------------------------------------------
% Frame 25: Key Takeaways
% ---------------------------------------------------------
\begin{frame}{Key Takeaways}
  \begin{enumerate}
    \item \textcolor{mlpurple}{\textbf{Growth drivers:}} Smartphone penetration, cheap cloud,
          eroded bank trust, and regulatory openings combine to produce the fintech growth wave
    \vspace{0.15em}
    \item \textcolor{mlblue}{\textbf{Inclusion gap:}} 1.7~billion adults remain unbanked; fintech's
          structural advantage is reaching them through mobile -- but only by deliberate design
    \vspace{0.15em}
    \item \textcolor{mlgreen}{\textbf{Trust framework:}} Competence, integrity, and benevolence trust
          must all be built; benevolence trust is the strongest predictor of sustained adoption
    \vspace{0.15em}
    \item \textcolor{mlorange}{\textbf{Behavioral barriers:}} Status quo bias, loss aversion, and
          digital literacy gaps explain why good technology alone does not drive adoption
    \vspace{0.15em}
    \item \textcolor{mlteal}{\textbf{Five nudges:}} Smart defaults, commitment devices, social norms,
          friction reduction, and progress feedback are the evidence-based toolkit
    \vspace{0.15em}
    \item \textcolor{mlred}{\textbf{Dark patterns:}} Nudges that serve the platform at users' expense
          are increasingly illegal under Consumer Duty and equivalent frameworks
    \vspace{0.15em}
    \item \textcolor{mlpurple}{\textbf{Inclusion--protection trade-off:}} There is no design choice
          that maximises both simultaneously; the position must be explicit and defended
  \end{enumerate}
  \bottomnote{Review question: A neobank wants to increase savings rates. Propose two nudges and evaluate each against the four ethical criteria from Frame 16.}
\end{frame}

% ---------------------------------------------------------
% Frame 26: Summary and Key Vocabulary
% ---------------------------------------------------------
\begin{frame}{Summary and Key Vocabulary}
  \begin{block}{Lecture Summary}
    The fintech ecosystem grows because four structural forces align: connectivity,
    cheap infrastructure, eroded trust, and regulatory openings. But growth does not
    automatically produce inclusion. Trust barriers, behavioural biases, and digital
    literacy gaps create a predictable adoption chasm. Choice architecture and nudging
    offer powerful tools for crossing that chasm -- but those same tools can be turned
    against users. The ethical designer must apply transparency, alignment, opt-out
    ease, and equity as non-negotiable constraints.
    \textcolor{mlpurple}{\textbf{Design intent determines whether fintech includes or exploits.}}
  \end{block}
  \vspace{0.4em}
  \begin{multicols}{2}
    \small
    \begin{itemize}
      \item \textbf{Fintech Ecosystem}
      \item \textbf{Financial Inclusion}
      \item \textbf{Leapfrog Effect}
      \item \textbf{Technology Adoption Lifecycle}
      \item \textbf{Trust (Competence / Integrity / Benevolence)}
      \item \textbf{Choice Architecture}
      \item \textbf{Nudge}
      \item \textbf{Dark Pattern}
      \item \textbf{Status Quo Bias}
      \item \textbf{Default Setting}
    \end{itemize}
  \end{multicols}
  \vspace{0.2em}
  \textcolor{mlgray}{\small \textbf{Next lecture:} Payments and Transactions -- real-time rails, card networks, cross-border infrastructure, and CBDC. L03 begins tomorrow morning.}
  \bottomnote{Bring your Workshop B ecosystem health evaluation to the L03 opening discussion.}
\end{frame}

\end{document}
