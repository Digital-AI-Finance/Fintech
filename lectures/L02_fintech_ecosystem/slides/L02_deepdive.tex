% ============================================================
% L02_deepdive.tex -- Lecture 2: Fintech Ecosystem (Deep Dive Variant)
% Growth, Social Impact, and Behavioral Dimensions
% Audience: MSc Finance/Business -- Advanced/Analytical Depth
% Frame count: ~12 main body + ~5 appendix = ~17 total
% Architecture: MAIN BODY + \appendix
% ============================================================

\documentclass[aspectratio=169, 11pt]{beamer}

% ============================================================
% THEME BASE
% ============================================================
\usetheme{Madrid}
\usecolortheme{whale}

% ============================================================
% PACKAGES
% ============================================================
\usepackage[T1]{fontenc}
\usepackage[utf8]{inputenc}
\usepackage{graphicx}
\usepackage{booktabs}
\usepackage{tikz}
\usepackage{pgfplots}
\usepackage{amsmath}
\usepackage{hyperref}
\usepackage{multicol}
\usepackage{xcolor}

% ============================================================
% TIKZ LIBRARIES
% ============================================================
\usetikzlibrary{arrows.meta, positioning, shapes.geometric, calc, decorations.pathmorphing}

% ============================================================
% PGFPLOTS COMPATIBILITY
% ============================================================
\pgfplotsset{compat=1.18}

% ============================================================
% COLOR DEFINITIONS (Fintech V4 Palette)
% ============================================================
\definecolor{MLPURPLE}{HTML}{9467BD}
\definecolor{MLBLUE}{HTML}{1F77B4}
\definecolor{MLRED}{HTML}{D62728}
\definecolor{MLORANGE}{HTML}{FF7F0E}
\definecolor{MLGREEN}{HTML}{2CA02C}
\definecolor{MLGRAY}{HTML}{7F7F7F}
\definecolor{MLTEAL}{HTML}{0D7377}
\definecolor{MLCYAN}{HTML}{14BDEB}

% Lowercase aliases for use in \textcolor{}
\colorlet{mlpurple}{MLPURPLE}
\colorlet{mlblue}{MLBLUE}
\colorlet{mlred}{MLRED}
\colorlet{mlorange}{MLORANGE}
\colorlet{mlgreen}{MLGREEN}
\colorlet{mlgray}{MLGRAY}
\colorlet{mlteal}{MLTEAL}
\colorlet{mlcyan}{MLCYAN}

% ============================================================
% BEAMER COLOR CUSTOMIZATION
% ============================================================
\setbeamercolor{structure}{fg=MLTEAL}
\setbeamercolor{palette primary}{bg=MLTEAL, fg=white}
\setbeamercolor{palette secondary}{bg=MLTEAL!80, fg=white}
\setbeamercolor{palette tertiary}{bg=MLTEAL!60, fg=white}
\setbeamercolor{palette quaternary}{bg=MLTEAL!40, fg=white}
\setbeamercolor{frametitle}{bg=MLTEAL!10, fg=MLTEAL}
\setbeamercolor{frametitle right}{bg=MLTEAL!5}
\setbeamercolor{block title}{bg=MLTEAL, fg=white}
\setbeamercolor{block body}{bg=MLTEAL!8, fg=black}
\setbeamercolor{block title alerted}{bg=MLRED, fg=white}
\setbeamercolor{block body alerted}{bg=MLRED!8, fg=black}
\setbeamercolor{block title example}{bg=MLGREEN, fg=white}
\setbeamercolor{block body example}{bg=MLGREEN!8, fg=black}
\setbeamercolor{title}{fg=white}
\setbeamercolor{subtitle}{fg=MLCYAN}
\setbeamercolor{author}{fg=white}
\setbeamercolor{institute}{fg=white}
\setbeamercolor{date}{fg=white}
\setbeamercolor{title page header}{bg=MLTEAL}

% ============================================================
% NAVIGATION AND FOOTLINE
% ============================================================
\setbeamertemplate{navigation symbols}{}

\setbeamertemplate{footline}{%
  \leavevmode%
  \hbox{%
    \begin{beamercolorbox}[wd=.333\paperwidth, ht=2.25ex, dp=1ex, center]{palette primary}%
      \usebeamerfont{author in head/foot}\insertshortauthor
    \end{beamercolorbox}%
    \begin{beamercolorbox}[wd=.334\paperwidth, ht=2.25ex, dp=1ex, center]{palette secondary}%
      \usebeamerfont{title in head/foot}\insertshorttitle
    \end{beamercolorbox}%
    \begin{beamercolorbox}[wd=.333\paperwidth, ht=2.25ex, dp=1ex, right]{palette tertiary}%
      \usebeamerfont{date in head/foot}%
      \insertframenumber{} / \inserttotalframenumber\hspace*{2ex}
    \end{beamercolorbox}%
  }%
  \vskip0pt%
}

% ============================================================
% BOTTOM NOTE COMMAND
% ============================================================
\newcommand{\bottomnote}[1]{%
  \vfill
  \begin{beamercolorbox}[wd=\textwidth, ht=2ex, dp=1ex]{palette primary}%
    \tiny\hspace{1em}#1
  \end{beamercolorbox}%
}

% ============================================================
% GRAPHICS PATH
% ============================================================
\graphicspath{{figures/}}

% ============================================================
% COURSE METADATA
% ============================================================
\title{Financial Technology (FinTech)}
\author{Joerg Osterrieder}
\institute{University of Zurich \\ Department of Finance}
\date{Spring 2026}


\subtitle{Growth, Social Impact, and Behavioral Dimensions --- Deep Dive}

% ============================================================
% BEGIN DOCUMENT
% ============================================================
\begin{document}

% ============================================================
% TITLE PAGE
% ============================================================
\begin{frame}[plain]
  \titlepage
\end{frame}

% ============================================================
% MAIN BODY (~12 frames)
% ============================================================

% --- Frame 2: Advanced Learning Objectives ---

\begin{frame}{Advanced Learning Objectives}
  \begin{columns}[T]
    \begin{column}{0.55\textwidth}
      \textbf{This deep dive targets the upper tiers of Bloom's Taxonomy:}
      \begin{itemize}
        \item \textbf{Analyze} the psychological mechanisms that explain fintech adoption patterns -- moving beyond demographic correlates to causal theory \hfill\textit{[Analyze]}
        \item \textbf{Evaluate} competing trust measurement frameworks and adjudicate their empirical validity \hfill\textit{[Evaluate]}
        \item \textbf{Critique} M-Pesa's economic impact evidence: distinguish correlation from causal identification \hfill\textit{[Evaluate]}
        \item \textbf{Construct} a regulatory response framework for dark patterns that balances innovation and protection \hfill\textit{[Create]}
        \item \textbf{Appraise} the ethical foundations of libertarian paternalism in financial product design \hfill\textit{[Evaluate]}
      \end{itemize}
    \end{column}
    \begin{column}{0.42\textwidth}
      \begin{block}{Assumed Background}
        You are familiar with: basic behavioral economics (loss aversion, present bias), financial inclusion concepts, and the standard technology adoption narrative. This session interrogates those foundations analytically.
      \end{block}
      \vspace{0.4em}
      \begin{alertblock}{Central Analytical Question}
        When fintech firms use behavioral science to shape user decisions, under what conditions does this constitute \emph{paternalistic beneficence} vs.\ \emph{exploitative manipulation}?
      \end{alertblock}
    \end{column}
  \end{columns}
  \bottomnote{Appendix contains a behavioral finance glossary, nudging taxonomy (Sunstein), academic references, and advanced seminar discussion questions.}
\end{frame}

% --- Frame 3: Behavioral Economics Foundations ---

\begin{frame}{Behavioral Economics Foundations: Three Frameworks}
  \begin{columns}[T]
    \begin{column}{0.55\textwidth}
      \textbf{1.\ Prospect Theory (Kahneman \& Tversky, 1979):}
      \begin{itemize}
        \item Individuals evaluate outcomes as \emph{gains and losses relative to a reference point}, not as absolute states of wealth
        \item Loss aversion coefficient $\lambda \approx 2.25$: losing \$100 feels roughly as bad as gaining \$225 feels good
        \item \textbf{Fintech implication}: displaying an account deficit as a ``loss from target'' increases saving behavior more than displaying an absolute balance
        \item Diminishing sensitivity: the psychological difference between losing \$100 and \$200 is smaller than between \$0 and \$100
      \end{itemize}
      \vspace{0.4em}
      \textbf{2.\ Dual-Process Theory (Kahneman, 2011):}
      \begin{itemize}
        \item System 1 (fast, automatic, heuristic) governs most financial micro-decisions
        \item System 2 (slow, deliberate, analytic) is cognitively expensive and rarely engaged
        \item UX design that triggers System 1 bypasses deliberate consent -- both the opportunity and the ethical hazard of fintech design
      \end{itemize}
    \end{column}
    \begin{column}{0.42\textwidth}
      \textbf{3.\ Temporal Discounting (Laibson, 1997):}
      \begin{itemize}
        \item Hyperbolic discounting: people discount the immediate future steeply but discount the distant future more gradually
        \item $\beta$--$\delta$ model: $U = u(c_t) + \beta \sum_{\tau=1}^{T} \delta^\tau u(c_{t+\tau})$, where $\beta < 1$ captures present bias
        \item Present bias ($\beta \approx 0.7$ empirically) explains persistent under-saving and BNPL over-borrowing -- consumers intend to repay at $t+1$ but face the same present bias again at $t+1$
      \end{itemize}
      \vspace{0.3em}
      \begin{exampleblock}{Synthesis}
        These three frameworks are not merely descriptive: they are design constraints. A fintech that ignores prospect theory, dual-process architecture, and present bias in product design is making implicit behavioral assumptions that are empirically refuted.
      \end{exampleblock}
    \end{column}
  \end{columns}
  \bottomnote{Kahneman \& Tversky (1979) \emph{Econometrica} 47(2); Kahneman (2011) \emph{Thinking, Fast and Slow}; Laibson (1997) ``Golden Eggs and Hyperbolic Discounting'' \emph{QJE} 112(2).}
\end{frame}

% --- Frame 4: Trust Measurement Models ---

\begin{frame}{Trust Measurement Models: From Construct to Operationalisation}
  \begin{columns}[T]
    \begin{column}{0.52\textwidth}
      \begin{center}
        \includegraphics[width=0.95\textwidth]{figures/05_trust_framework_comparison/chart.pdf}
      \end{center}
    \end{column}
    \begin{column}{0.45\textwidth}
      \textbf{The Measurement Challenge:}
      \begin{itemize}
        \item Trust is a \emph{latent construct} -- it cannot be observed directly and must be inferred from indicators
        \item Three dominant operationalisation approaches in the fintech literature:
      \end{itemize}
      \vspace{0.3em}
      \textbf{(i) Trust Propensity Scales (McKnight et al., 2002):}
      \begin{itemize}\small
        \item Disposition to trust: ``I generally trust others until proven otherwise''
        \item Structural assurance: trust in institutional safeguards (regulation, deposit insurance)
        \item Situational normality: perception that the environment is ordered and normal
      \end{itemize}
      \vspace{0.3em}
      \textbf{(ii) Perceived Risk Framework (Featherman \& Pavlou, 2003):}
      \begin{itemize}\small
        \item Decomposes risk into: financial, performance, privacy, time, psychological, and social risk dimensions
        \item Trust = $f(\text{competence, benevolence, integrity})$ -- three-factor model
      \end{itemize}
      \vspace{0.3em}
      \textbf{(iii) Technology Acceptance with Trust (TAM-T):}
      \begin{itemize}\small
        \item Extends Davis (1989) TAM with trust as a mediator between perceived usefulness and adoption intention
      \end{itemize}
    \end{column}
  \end{columns}
  \bottomnote{McKnight, Choudhury \& Kacmar (2002) \emph{Information Systems Research} 13(3); Featherman \& Pavlou (2003) \emph{European Journal of IS} 12(3); Davis (1989) \emph{MIS Quarterly} 13(3).}
\end{frame}

% --- Frame 5: M-Pesa Economic Impact Deep Dive ---

\begin{frame}{M-Pesa Economic Impact: Separating Evidence from Narrative}
  \begin{columns}[T]
    \begin{column}{0.55\textwidth}
      \textbf{The Causal Identification Problem:}
      \begin{itemize}
        \item Correlation between M-Pesa adoption and household consumption is well-established (Jack \& Suri, 2011; Suri \& Jack, 2016)
        \item But: M-Pesa was \emph{not} randomly assigned. Adoption correlates with: proximity to agent networks, literacy, social network density, baseline income
        \item A naive regression of welfare on adoption captures both the M-Pesa effect \emph{and} the correlates of being the type of person who adopts
      \end{itemize}
      \vspace{0.4em}
      \begin{block}{Suri \& Jack (2016) \emph{Science} Identification Strategy}
        Instrument: geographic variation in M-Pesa agent rollout as an exogenous shifter of adoption. IV estimate: households with M-Pesa access lifted 2\% of households out of poverty (194,000 households). Female-headed households benefited disproportionately through occupational switching from subsistence farming to business/retail.
      \end{block}
    \end{column}
    \begin{column}{0.42\textwidth}
      \textbf{GDP Contribution Evidence:}
      \begin{itemize}
        \item By 2023: M-Pesa processes value equivalent to $\sim$50\% of Kenya's GDP annually -- but this is \emph{transaction flow}, not value-added
        \item Financial inclusion: unbanked rate in Kenya fell from $\sim$80\% (2006) to $\sim$27\% (2021), but attribution to M-Pesa alone is contested (parallel bank expansion, regulatory change)
        \item Poverty trap reduction: access to mobile savings reduces consumption volatility during shocks -- this is the primary welfare-improving channel, not income growth per se
      \end{itemize}
      \vspace{0.3em}
      \begin{alertblock}{Replication Concerns}
        M-Pesa's success has not replicated at scale outside Kenya/Tanzania. Similar platforms in India, Pakistan, and West Africa achieved far lower adoption. Context-specificity (agent density, trust in Safaricom, regulatory environment) is underweighted in the standard narrative.
      \end{alertblock}
    \end{column}
  \end{columns}
  \bottomnote{Suri \& Jack (2016) ``The long-run poverty and gender impacts of mobile money'' \emph{Science} 354(6317); Jack \& Suri (2011) NBER WP 16721; Mbiti \& Weil (2016) \emph{Journal of African Economies}.}
\end{frame}

% --- Frame 6: Growth Drivers: Beyond the Surface ---

\begin{frame}{Growth Drivers: A Critical Examination of the Standard Narrative}
  \begin{columns}[T]
    \begin{column}{0.55\textwidth}
      \begin{center}
        \includegraphics[width=0.96\textwidth]{figures/06_technology_adoption_lifecycle/chart.pdf}
      \end{center}
    \end{column}
    \begin{column}{0.42\textwidth}
      \textbf{The Standard Narrative:}
      \begin{itemize}
        \item Fintech grew because smartphones proliferated, APIs democratized access, and consumers demanded better UX
      \end{itemize}
      \vspace{0.3em}
      \textbf{The Critique (Buchak et al., 2018):}
      \begin{itemize}
        \item Shadow bank / fintech mortgage market share growth in the US is approximately 60\% explained by \emph{regulatory arbitrage} (avoidance of Basel III capital requirements), not technology advantage
        \item Technology explains only $\sim$40\% of the growth premium
      \end{itemize}
      \vspace{0.3em}
      \textbf{Implications:}
      \begin{itemize}
        \item Growth projections that assume technology as the primary driver will overestimate long-run growth as regulatory arbitrage closes
        \item The policy question is not ``how do we accelerate fintech?'' but ``which fintech growth is attributable to genuine value creation vs.\ regulatory asymmetry?''
      \end{itemize}
      \begin{exampleblock}{Analytical Takeaway}
        Disaggregate the growth driver before forecasting. Regulatory arbitrage is not a durable competitive advantage.
      \end{exampleblock}
    \end{column}
  \end{columns}
  \bottomnote{Buchak, Matvos, Piskorski \& Seru (2018) ``Fintech, Regulatory Arbitrage, and the Rise of Shadow Banks'' \emph{Journal of Financial Economics} 130(3), 453--483.}
\end{frame}

% --- Frame 7: Advanced Choice Architecture ---

\begin{frame}{Advanced Choice Architecture: Sunstein's Taxonomy and the Paternalism Debate}
  \begin{columns}[T]
    \begin{column}{0.52\textwidth}
      \begin{center}
        \includegraphics[width=0.95\textwidth]{figures/08_nudging_architecture/chart.pdf}
      \end{center}
    \end{column}
    \begin{column}{0.45\textwidth}
      \textbf{Libertarian Paternalism (Thaler \& Sunstein, 2003):}
      \begin{itemize}
        \item \textbf{Libertarian}: preserves freedom of choice; no option is removed; people can always opt out
        \item \textbf{Paternalist}: the choice architect deliberately steers people toward outcomes judged to be in their interest
        \item The justification: because defaults and framing affect choices regardless of intent, choice architects cannot be ``neutral'' -- the question is only whether the architecture is conscious or unconscious, not whether it exists
      \end{itemize}
      \vspace{0.3em}
      \textbf{Sunstein's (2014) Nudge Taxonomy:}
      \begin{enumerate}\small
        \item \textit{Default rules}: opt-in vs.\ opt-out (most powerful)
        \item \textit{Simplification}: reduce cognitive load on form design
        \item \textit{Social norms}: ``most customers in your situation save X\%''
        \item \textit{Increases in ease}: friction removal for desired behaviors
        \item \textit{Disclosure}: salience of relevant information
        \item \textit{Warnings}: loss-framed risk communication
        \item \textit{Pre-commitment}: commitment devices for future selves
        \item \textit{Reminders}: temporal prompts for intended actions
      \end{enumerate}
    \end{column}
  \end{columns}
  \bottomnote{Thaler \& Sunstein (2003) ``Libertarian Paternalism'' \emph{AER P\&P} 93(2); Sunstein (2014) \emph{Why Nudge? The Politics of Libertarian Paternalism} Yale UP; Thaler \& Sunstein (2008) \emph{Nudge} Penguin.}
\end{frame}

% --- Frame 8: Dark Patterns: Regulatory Response Analysis ---

\begin{frame}{Dark Patterns: Anatomy and Regulatory Response}
  \begin{columns}[T]
    \begin{column}{0.55\textwidth}
      \begin{center}
        \includegraphics[width=0.96\textwidth]{figures/09_choice_architecture_examples/chart.pdf}
      \end{center}
    \end{column}
    \begin{column}{0.42\textwidth}
      \textbf{What Distinguishes a Dark Pattern from a Nudge?}
      \begin{itemize}
        \item \textbf{Nudge}: steers toward user's stated or revealed long-run interest; easy to reverse; transparent
        \item \textbf{Dark pattern}: steers toward firm's interest at user's expense; exploits System 1; deliberately obscures the opt-out path
      \end{itemize}
      \vspace{0.3em}
      \textbf{Regulatory Responses (Comparative):}
      \begin{itemize}\small
        \item \textit{EU Digital Services Act (2022)}: prohibits dark patterns broadly; requires ``equal ease'' of opt-out vs.\ opt-in
        \item \textit{UK FCA Consumer Duty (2023)}: outcomes-based standard -- firms must demonstrate good outcomes, not just compliant disclosures
        \item \textit{US CFPB}: case-by-case enforcement under UDAAP (Unfair, Deceptive, or Abusive Acts or Practices); no bright-line dark pattern rule
        \item \textit{India RBI}: requires ``cooling-off'' periods for auto-renewals; mandates one-click cancellation
      \end{itemize}
      \begin{alertblock}{The Enforcement Gap}
        Behavioral evidence of harm is necessary but not sufficient for enforcement under most frameworks. Proving intent to exploit (vs.\ poorly designed UX) remains a significant evidentiary challenge.
      \end{alertblock}
    \end{column}
  \end{columns}
  \bottomnote{FCA (2023) Consumer Duty PS22/9; EU DSA Art.\ 25 (deceptive design practices); Brignull (2010) ``Dark Patterns'' (original taxonomy); Mathur et al.\ (2019) \emph{ACM CSCW}.}
\end{frame}

% --- Frame 9: Trust Fragility: Quantitative Analysis ---

\begin{frame}{Trust Fragility: A Quantitative Perspective}
  \begin{columns}[T]
    \begin{column}{0.52\textwidth}
      \textbf{The Asymmetric Trust Dynamics:}
      \begin{itemize}
        \item Trust accumulates slowly: longitudinal studies suggest repeated positive interactions compound trust at approximately $+3$--$5$\% per satisfactory interaction
        \item Trust erodes rapidly: a single salient negative event reduces trust by $15$--$40$\% on average (depending on severity and perceived intentionality)
        \item \textbf{Trust is path-dependent}: the same data breach causes greater trust loss in a firm users perceived as highly trustworthy than in one already viewed with suspicion (\emph{violation of expectations effect})
      \end{itemize}
      \vspace{0.4em}
      \begin{block}{The Trust-Adoption Feedback Loop}
        Adoption $\rightarrow$ familiarity $\rightarrow$ trust $\rightarrow$ deeper engagement $\rightarrow$ data sharing $\rightarrow$ personalization $\rightarrow$ higher switching costs $\rightarrow$ higher trust (institutional). But: a breach at any stage reverses the loop rapidly.
      \end{block}
    \end{column}
    \begin{column}{0.45\textwidth}
      \textbf{Crisis Propagation in Trust Networks:}
      \begin{itemize}
        \item Trust failures in fintech exhibit \emph{contagion}: a breach at fintech A reduces adoption intent at unrelated fintech B (category-level stigma)
        \item The magnitude of contagion correlates with: media salience of the breach, product category similarity, brand proximity, and whether the failure was attributed to negligence vs.\ malice
      \end{itemize}
      \vspace{0.3em}
      \begin{exampleblock}{Policy Implication}
        Systemic trust stability is a public good that individual firms cannot fully internalise. This is one justification for mandatory security standards and breach disclosure rules -- the private optimal level of security investment is below the social optimum.
      \end{exampleblock}
      \vspace{0.3em}
      \textbf{Measurement instrument:} The Net Trust Score (NTS) proposed by Edelman (2022) decomposes trust into competence and ethical dimensions -- fintech firms score asymmetrically high on competence, low on ethics.
    \end{column}
  \end{columns}
  \bottomnote{Dunn \& Schweitzer (2005) \emph{JPSP} 93(3) on trust dynamics; Edelman Trust Barometer (2022/23) Financial Services sector; Luhmann (1979) \emph{Trust and Power} for the sociological foundations.}
\end{frame}

% --- Frame 10: Inclusion-Protection Policy Design ---

\begin{frame}{Inclusion-Protection Tension: Policy Design Principles}
  \begin{columns}[T]
    \begin{column}{0.55\textwidth}
      \begin{center}
        \includegraphics[width=0.96\textwidth]{figures/10_ecosystem_stakeholder_impact/chart.pdf}
      \end{center}
    \end{column}
    \begin{column}{0.42\textwidth}
      \textbf{The Fundamental Tension:}
      \begin{itemize}
        \item \textbf{Inclusion imperative}: lower barriers, reduce friction, simplify onboarding, allow thin-file lending -- each action increases access for previously excluded populations
        \item \textbf{Protection imperative}: impose suitability requirements, cooling-off periods, disclosure obligations, affordability checks -- each action creates friction that disproportionately excludes low-literacy users
      \end{itemize}
      \vspace{0.3em}
      \textbf{Three Policy Design Principles (Demirguc-Kunt et al., 2022):}
      \begin{enumerate}
        \item \textit{Progressive suitability}: apply graduated protection requirements proportional to product complexity and potential harm, not a flat standard
        \item \textit{Embedded disclosure}: disclose information at the decision point via the channel in use (in-app, not separate document) -- reduces the compliance-inclusion tradeoff
        \item \textit{Behavioral testing mandate}: require A/B testing evidence that disclosures are actually understood before market authorization, not mere compliance with a format standard
      \end{enumerate}
    \end{column}
  \end{columns}
  \bottomnote{Demirguc-Kunt, Klapper, Singer \& Ansar (2022) \emph{Global Findex Database 2021}, World Bank; Alliance for Financial Inclusion (2023) ``Digital Financial Services Policy Framework.'' }
\end{frame}

% --- Frame 11: Evaluation Framework: Academic Foundations ---

\begin{frame}{Evaluation Framework: Academic Foundations for Fintech Impact Assessment}
  \begin{columns}[T]
    \begin{column}{0.52\textwidth}
      \textbf{Four Criteria for Rigorous Impact Assessment:}
      \vspace{0.3em}

      \textbf{1.\ Counterfactual Clarity:}
      \begin{itemize}
        \item What would have happened to this population \emph{without} the fintech intervention? Impact = observed outcome $-$ counterfactual outcome
        \item Most fintech case studies omit the counterfactual entirely; this inflates estimated impact
      \end{itemize}
      \vspace{0.3em}
      \textbf{2.\ Attribution Precision:}
      \begin{itemize}
        \item Distinguish the fintech's specific contribution from: parallel infrastructure investment, macroeconomic trends, regulatory change, and other concurrent programs
        \item Instrumental variable strategies, difference-in-differences, and RCTs each have specific validity conditions in this context
      \end{itemize}
    \end{column}
    \begin{column}{0.45\textwidth}
      \textbf{3.\ Distributional Analysis:}
      \begin{itemize}
        \item Average treatment effects can conceal regressive distribution: if the top income quintile captures most of the welfare gain, an intervention may increase inequality even while raising the mean
        \item Require heterogeneous treatment effect estimation across income, gender, literacy, and geography
      \end{itemize}
      \vspace{0.3em}
      \textbf{4.\ Temporal Horizons:}
      \begin{itemize}
        \item Short-run effects (account opening, transaction volume) often diverge from long-run effects (sustained financial behavior change, credit quality)
        \item BNPL studies showing short-run consumption smoothing must be evaluated against long-run debt accumulation evidence
      \end{itemize}
      \begin{alertblock}{The Advocacy Problem}
        Fintech firms, investors, and regulators each have incentives to overstate impact. Independent academic replication with pre-registered hypotheses is the only robust counter.
      \end{alertblock}
    \end{column}
  \end{columns}
  \bottomnote{Imbens \& Wooldridge (2009) ``Recent Developments in the Econometrics of Program Evaluation'' \emph{JEL} 47(1); Duflo, Glennerster \& Kremer (2007) ``Using Randomization in Development Economics Research'' \emph{Handbook of Development Economics}.}
\end{frame}

% --- Frame 12: Central Tension: Research Frontiers ---

\begin{frame}{Central Tension: Research Frontiers}
  \begin{columns}[T]
    \begin{column}{0.55\textwidth}
      \textbf{Five Open Research Questions in the Field:}
      \begin{enumerate}
        \item \textbf{Behavioral mechanism vs.\ access channel}: Does fintech improve welfare because it changes \emph{how} decisions are made (behavioral design) or because it reduces \emph{barriers to entry} into financial markets? These require different policy responses.
        \item \textbf{Trust and systemic risk}: As fintech firms become systemically important, does the trust model of regulation (rely on consumer vigilance) need to shift to the systemic risk model (mandatory buffers, resolution regimes)?
        \item \textbf{Algorithmic discrimination}: Can ML-driven credit scoring simultaneously be more accurate \emph{and} more discriminatory by protected characteristics -- and if so, which metric takes precedence?
        \item \textbf{Present bias exploitation}: Where is the boundary between a pre-commitment device (protective) and a lock-in mechanism (exploitative) when both use the same psychological architecture?
        \item \textbf{Platform-level behavior}: When a platform hosts both a fintech product and a behavioral nudge engine, who owns the ethical responsibility for the interaction between them?
      \end{enumerate}
    \end{column}
    \begin{column}{0.42\textwidth}
      \begin{alertblock}{The Unresolved Core Tension}
        Fintech's most powerful tools -- behavioral design, personalization, AI-driven nudging -- are effective precisely \emph{because} they operate below the threshold of deliberate user cognition.

        The same property that makes them powerful welfare-enhancing tools also makes them the most potent instruments of exploitation in financial history.

        No existing regulatory framework has adequately resolved this duality.
      \end{alertblock}
      \vspace{0.4em}
      \begin{block}{For Dissertation Research}
        Each of the five open questions above is an under-studied area with significant empirical and theoretical work remaining. Causal identification strategies for behavioral fintech interventions are particularly scarce.
      \end{block}
    \end{column}
  \end{columns}
  \bottomnote{Fuster et al.\ (2022) ``Predictably Unequal? The Effects of ML on Credit Markets'' \emph{JF} 77(1); Acquisti, Taylor \& Wagman (2016) ``The Economics of Privacy'' \emph{JEL} 54(2).}
\end{frame}

% ============================================================
% APPENDIX
% ============================================================
\appendix

\section*{Appendix: Advanced Topics}

% --- Frame 13: Behavioral Finance Glossary ---

\begin{frame}{Behavioral Finance Glossary}
  \begin{columns}[T]
    \begin{column}{0.49\textwidth}
      \textbf{Core Concepts:}
      \vspace{0.2em}

      \textbf{Anchoring}: Over-reliance on the first piece of information encountered when making decisions. In fintech: initial balance display anchors subsequent saving targets.

      \vspace{0.3em}
      \textbf{Choice overload (Iyengar \& Lepper, 2000)}: Excessive options reduce decision quality and increase default selection. Fintech application: product menus with $>$7 options reduce cross-sell conversion.

      \vspace{0.3em}
      \textbf{Default effect}: The option presented as default receives disproportionate selection rates regardless of content. Most powerful nudging tool identified in the literature.

      \vspace{0.3em}
      \textbf{Endowment effect}: People value what they already own above its market price ($\lambda \approx 2$). Subscription inertia in fintech products is partially endowment-effect-driven.

      \vspace{0.3em}
      \textbf{Framing effect}: Logically equivalent choices produce systematically different decisions depending on how options are presented (gain frame vs.\ loss frame).
    \end{column}
    \begin{column}{0.49\textwidth}
      \textbf{Advanced Concepts:}
      \vspace{0.2em}

      \textbf{Hyperbolic discounting}: Discount function $D(\tau) = (1 + k\tau)^{-1}$ rather than exponential $D(\tau) = \delta^\tau$. Implies preference reversals: preferring ``\$100 today'' over ``\$110 tomorrow'' but also preferring ``\$110 in 31 days'' over ``\$100 in 30 days.''

      \vspace{0.3em}
      \textbf{Mental accounting (Thaler, 1985)}: People segregate wealth into non-fungible mental accounts (``holiday fund,'' ``emergency fund'') and treat money differently depending on its account label. Fintech applications: savings pots, goal-based investing.

      \vspace{0.3em}
      \textbf{Overconfidence bias}: Systematic overestimation of one's own knowledge, accuracy, and skill. In fintech: over-trading in robo-advisory platforms; under-diversification.

      \vspace{0.3em}
      \textbf{Status quo bias (Samuelson \& Zeckhauser, 1988)}: Preference for the current state; changes are evaluated as losses rather than neutral transitions. Amplifies default effect.

      \vspace{0.3em}
      \textbf{Sunk cost fallacy}: Continued commitment to a failing course of action due to prior investment. Relevant to subscription churn analysis.
    \end{column}
  \end{columns}
  \bottomnote{Thaler (1985) ``Mental Accounting and Consumer Choice'' \emph{Marketing Science}; Samuelson \& Zeckhauser (1988) \emph{JRU} 1(1); Iyengar \& Lepper (2000) \emph{JPSP} 79(6).}
\end{frame}

% --- Frame 14: Academic References and Key Papers ---

\begin{frame}{Academic References and Key Papers}
  \begin{columns}[T]
    \begin{column}{0.49\textwidth}
      \textbf{Foundational Behavioral Economics:}
      \begin{itemize}\small
        \item Kahneman, D.\ \& Tversky, A.\ (1979). ``Prospect Theory.'' \emph{Econometrica} 47(2), 263--291
        \item Laibson, D.\ (1997). ``Golden Eggs and Hyperbolic Discounting.'' \emph{QJE} 112(2), 443--478
        \item Thaler, R.H.\ \& Sunstein, C.R.\ (2008). \emph{Nudge.} Penguin
        \item Camerer, C., Loewenstein, G., \& Rabin, M.\ (eds., 2004). \emph{Advances in Behavioral Economics.} Princeton UP
      \end{itemize}
      \vspace{0.4em}
      \textbf{Fintech and Behavioral Design:}
      \begin{itemize}\small
        \item Hastings, J., Madrian, B., \& Skimmyhorn, W.\ (2013). ``Financial Literacy, Financial Education, and Economic Outcomes.'' \emph{Annual Review of Economics} 5
        \item Lusardi, A.\ \& Mitchell, O.S.\ (2014). ``The Economic Importance of Financial Literacy.'' \emph{JEL} 52(1)
        \item Bhattacharya, U., et al.\ (2012). ``Is Unbiased Financial Advice to Retail Investors Sufficient?'' \emph{RFS} 25(4)
      \end{itemize}
    \end{column}
    \begin{column}{0.49\textwidth}
      \textbf{Trust, Inclusion, and Impact:}
      \begin{itemize}\small
        \item Suri, T.\ \& Jack, W.\ (2016). ``The long-run poverty and gender impacts of mobile money.'' \emph{Science} 354(6317)
        \item Demirguc-Kunt, A., et al.\ (2022). \emph{Global Findex Database 2021.} World Bank
        \item McKnight, D.H., Choudhury, V., \& Kacmar, C.\ (2002). ``The Impact of Initial Consumer Trust on Intentions to Transact.'' \emph{Journal of Strategic Information Systems} 11(3--4)
        \item Featherman, M.S.\ \& Pavlou, P.A.\ (2003). ``Predicting e-services Adoption.'' \emph{European Journal of IS} 12(3)
      \end{itemize}
      \vspace{0.4em}
      \textbf{Regulatory and Policy:}
      \begin{itemize}\small
        \item Sunstein, C.R.\ (2014). \emph{Why Nudge?} Yale UP
        \item Thaler, R.H.\ \& Sunstein, C.R.\ (2003). ``Libertarian Paternalism.'' \emph{AER P\&P} 93(2)
        \item FCA (2023). Consumer Duty Policy Statement PS22/9
        \item EU (2022). Digital Services Act, Art.\ 25
      \end{itemize}
    \end{column}
  \end{columns}
  \bottomnote{Full reading list available at \texttt{website/downloads/L02\_reading\_list.pdf}. Items marked with ($*$) are core required reading; others are recommended for dissertation-level study.}
\end{frame}

% --- Frame 15: Nudging Taxonomy ---

\begin{frame}{Nudging Taxonomy: Sunstein's Classification Applied to Fintech}
  \begin{columns}[T]
    \begin{column}{0.52\textwidth}
      \begin{center}
        \includegraphics[width=0.95\textwidth]{figures/08_nudging_architecture/chart.pdf}
      \end{center}
    \end{column}
    \begin{column}{0.45\textwidth}
      \textbf{Classification by Mechanism and Fintech Application:}
      \vspace{0.3em}

      \begin{tabular}{@{}p{2.0cm}p{3.5cm}@{}}
        \toprule
        \textbf{Nudge Type} & \textbf{Fintech Application} \\
        \midrule
        Default rule & Auto-enrolled savings round-ups (e.g., Monzo Coin Jar) \\
        \addlinespace[0.2em]
        Simplification & One-tap payments; progress bars on application forms \\
        \addlinespace[0.2em]
        Social norms & ``Customers like you saved \pounds X last month'' \\
        \addlinespace[0.2em]
        Ease increase & Biometric authentication reduces transaction friction \\
        \addlinespace[0.2em]
        Disclosure & Real-time spend categorisation; fee salience \\
        \addlinespace[0.2em]
        Warning & Overdraft pre-alerts; BNPL affordability warnings \\
        \addlinespace[0.2em]
        Pre-commitment & Savings lock-ins; investment direct debits \\
        \addlinespace[0.2em]
        Reminder & Push notifications at decision-relevant moments \\
        \bottomrule
      \end{tabular}

      \vspace{0.3em}
      \begin{exampleblock}{Design Principle}
        Effective fintech nudging stacks multiple mechanism types. A savings feature that combines a default (auto-enrol), social norm (peer comparison), and pre-commitment (lock-in) outperforms any single mechanism in RCT evidence.
      \end{exampleblock}
    \end{column}
  \end{columns}
  \bottomnote{Sunstein (2014) \emph{Why Nudge?} ch.\ 2--4; Madrian \& Shea (2001) ``The Power of Suggestion: Inertia in 401(k) Participation'' \emph{QJE} 116(4) for the canonical default effect RCT.}
\end{frame}

% --- Frame 16: Adoption Lifecycle: Mathematical Foundations ---

\begin{frame}{Adoption Lifecycle: Mathematical Foundations}
  \begin{columns}[T]
    \begin{column}{0.52\textwidth}
      \begin{center}
        \includegraphics[width=0.95\textwidth]{figures/06_technology_adoption_lifecycle/chart.pdf}
      \end{center}
    \end{column}
    \begin{column}{0.45\textwidth}
      \textbf{Bass Diffusion Model (Bass, 1969):}
      \[
        \frac{dN(t)}{dt} = \left[ p + q \frac{N(t)}{M} \right] \bigl[ M - N(t) \bigr]
      \]
      where $N(t)$ = cumulative adopters at time $t$, $M$ = market potential, $p$ = innovation coefficient (external influence), $q$ = imitation coefficient (internal influence, word-of-mouth).

      \vspace{0.4em}
      \textbf{Key Parameter Insights:}
      \begin{itemize}
        \item High $p$, low $q$: adoption driven by marketing / external awareness (typical of B2B fintech)
        \item Low $p$, high $q$: adoption driven by social contagion (typical of consumer payments, M-Pesa)
        \item Peak adoption rate occurs at $t^* = \frac{\ln(q/p)}{p+q}$
      \end{itemize}

      \vspace{0.3em}
      \begin{block}{Limitation for Fintech}
        Bass assumes a fixed market potential $M$ and homogeneous population. Fintech adoption is heterogeneous (early adopters $\neq$ late majority in needs and behaviors) and $M$ is itself endogenous to product evolution and regulatory change.
      \end{block}
    \end{column}
  \end{columns}
  \bottomnote{Bass (1969) ``A New Product Growth for Model Consumer Durables'' \emph{Management Science} 15(5); for fintech applications see Gomber et al.\ (2017) ``On the Fintech Revolution'' \emph{JMIS} 35(1).}
\end{frame}

% --- Frame 17: Discussion Questions for Advanced Seminar ---

\begin{frame}{Discussion Questions for Advanced Seminar}
  \begin{columns}[T]
    \begin{column}{0.49\textwidth}
      \textbf{Analytical Questions:}
      \begin{enumerate}
        \item Kahneman's dual-process theory implies that most financial decisions engage System 1. If this is true, what are the implications for the ``informed consent'' model of financial regulation -- is meaningful consent to complex financial products actually possible?
        \item Suri \& Jack (2016) use geographic variation in M-Pesa agent rollout as an instrument. What are the exclusion restriction assumptions required for this IV to be valid? Under what conditions might they be violated?
        \item Buchak et al.\ (2018) attribute 60\% of fintech mortgage growth to regulatory arbitrage. If this is correct, what happens to fintech market share when the regulatory gap closes? Design a test for this prediction.
        \item The Bass model assumes $M$ is fixed. Propose a modified specification that allows market potential to grow endogenously with the adoption level. What parameter would govern this feedback, and what data would you use to estimate it?
      \end{enumerate}
    \end{column}
    \begin{column}{0.49\textwidth}
      \textbf{Policy and Ethics Questions:}
      \begin{enumerate}\setcounter{enumi}{4}
        \item Thaler \& Sunstein argue that because choice architecture is unavoidable, the only question is whether it is conscious or unconscious. Does this justify libertarian paternalism, or does it commit the naturalistic fallacy?
        \item Suppose a fintech firm uses ML to identify users with present bias and displays a pre-commitment savings tool prominently to those users. Is this a nudge (protective) or targeted exploitation? What regulatory test would you propose to distinguish them?
        \item The FCA Consumer Duty requires firms to demonstrate good outcomes. Propose a behavioral testing methodology that a fintech firm could use to satisfy this requirement for a BNPL product. What outcomes would you measure, over what time horizon, and in which populations?
        \item If trust is a public good (as argued in Frame 9), what are the implications for market structure in fintech? Should trust-building be subsidised, mandated, or left to markets?
      \end{enumerate}
    \end{column}
  \end{columns}
  \bottomnote{These questions are designed for 90-minute seminars. Questions 1--4 are primarily analytical; Questions 5--8 integrate analytical and normative reasoning, suitable for essay or dissertation topics.}
\end{frame}

\end{document}
