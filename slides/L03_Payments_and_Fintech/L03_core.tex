% ============================================================
%  L03_core.tex  --  Payments and Fintech
%  Core Slides (10 frames)
%  Self-contained (no \input{} commands)
%  Compile: pdflatex L03_core.tex  (twice for overlays)
% ============================================================

\documentclass[aspectratio=169, 11pt]{beamer}
\usetheme{Madrid}
\usecolortheme{whale}
\usepackage{tikz,pgfplots,booktabs,multicol,amsmath,graphicx}
\usetikzlibrary{arrows.meta,positioning,shapes.geometric,calc,decorations.pathmorphing}
\pgfplotsset{compat=1.18}

% ---- Colour palette ----------------------------------------
\definecolor{mlpurple}{HTML}{9467BD}
\definecolor{mlblue}{HTML}{1F77B4}
\definecolor{mlred}{HTML}{D62728}
\definecolor{mlorange}{HTML}{FF7F0E}
\definecolor{mlgreen}{HTML}{2CA02C}
\definecolor{mlgray}{HTML}{7F7F7F}
\definecolor{mlteal}{HTML}{0D7377}
\definecolor{mlcyan}{HTML}{14BDEB}

% ---- Beamer colour settings --------------------------------
\setbeamercolor{structure}{fg=mlteal}
\setbeamercolor{palette primary}{bg=mlteal,fg=white}
\setbeamercolor{palette secondary}{bg=mlteal!80,fg=white}
\setbeamercolor{palette tertiary}{bg=mlteal!60,fg=white}
\setbeamercolor{palette quaternary}{bg=mlteal!40,fg=white}
\setbeamercolor{frametitle}{bg=mlteal!10,fg=mlteal}
\setbeamercolor{frametitle right}{bg=mlteal!5}
\setbeamercolor{block title}{bg=mlteal,fg=white}
\setbeamercolor{block body}{bg=mlteal!8,fg=black}
\setbeamercolor{block title alerted}{bg=mlred,fg=white}
\setbeamercolor{block body alerted}{bg=mlred!8,fg=black}
\setbeamercolor{block title example}{bg=mlgreen,fg=white}
\setbeamercolor{block body example}{bg=mlgreen!8,fg=black}
\setbeamercolor{title}{fg=white}
\setbeamercolor{subtitle}{fg=mlcyan}
\setbeamercolor{author}{fg=white}
\setbeamercolor{institute}{fg=white}
\setbeamercolor{date}{fg=white}

% ---- Bottom-note command -----------------------------------
\newcommand{\bottomnote}[1]{%
  \vfill
  \begin{beamercolorbox}[wd=\textwidth,ht=2ex,dp=1ex]{palette primary}%
    \tiny\hspace{1em}#1%
  \end{beamercolorbox}}

% ---- Graphics path -----------------------------------------
\graphicspath{{}}

% ---- Metadata ----------------------------------------------
\title{Payments and Fintech}
\subtitle{From Cash to Digital: The Transformation of Money Movement}
\author{Joerg Osterrieder}
\institute{University of Zurich \\ Department of Finance}
\date{Spring 2026}
\setbeamertemplate{navigation symbols}{}

% ============================================================
\begin{document}
% ============================================================

% --- Frame 1: Title Page ---
\begin{frame}{Title Page}
  \titlepage
\end{frame}

% --- Frame 2: Bridge from Lecture 2 ---
\begin{frame}{Bridge from Lecture~2}
  \begin{columns}[T]
    \begin{column}{0.55\textwidth}
      In Lecture~2 we explored the \textbf{behavioral} side of fintech:
      trust, nudging, choice architecture, and the inclusion-protection
      trade-off.

      \vspace{0.5em}
      Now we apply that lens to the largest fintech vertical:
      \textcolor{mlpurple}{\textbf{payments}}.
      \begin{itemize}
        \item \textbf{Choice architecture:} Every payment interface shapes
              spending behavior --- tap-to-pay removes the ``pain of paying.''
        \item \textbf{Trust:} Consumers entrust payment providers with every
              transaction. Trust failure here is existential.
        \item \textbf{Inclusion:} Real-time payment rails (UPI, PIX) are the
              most powerful inclusion infrastructure ever built.
      \end{itemize}

      \vspace{0.3em}
      L03 shifts from \textcolor{mlpurple}{behavioral theory} to
      \textcolor{mlpurple}{payment system design as applied choice architecture}.
    \end{column}
    \begin{column}{0.42\textwidth}
      \includegraphics[width=\textwidth]{01_payment_history_timeline/chart.pdf}
    \end{column}
  \end{columns}
  \bottomnote{L02 gave you the behavioral lens. L03 shows you the infrastructure through which billions of financial decisions flow every day.}
\end{frame}

% --- Frame 3: The Four-Party Payment Model ---
\begin{frame}{The Four-Party Payment Model}
  \begin{columns}[T]
    \begin{column}{0.50\textwidth}
      \includegraphics[width=\textwidth]{03_four_party_payment_model/chart.pdf}
    \end{column}
    \begin{column}{0.47\textwidth}
      The card payment ecosystem involves four principals:

      \vspace{0.4em}
      \begin{itemize}
        \item \textbf{Cardholder} --- The consumer who initiates the
              transaction.
        \item \textbf{Issuer} --- The cardholder's bank. Issues the card,
              extends credit or debit access, bears fraud risk.
        \item \textbf{Acquirer} --- The merchant's bank. Processes the
              transaction, deposits funds, manages merchant risk.
        \item \textbf{Network} (Visa, Mastercard) --- Sets rules, routes
              messages, guarantees interoperability. Does not hold funds.
      \end{itemize}

      \vspace{0.3em}
      \begin{block}{Key Insight}
        The network is a \textcolor{mlpurple}{two-sided platform}: it must
        attract both cardholders (via issuers) and merchants (via acquirers)
        simultaneously.
      \end{block}
    \end{column}
  \end{columns}
  \bottomnote{Visa and Mastercard together process over 80\% of global card transactions. American Express and Discover operate three-party (closed-loop) models.}
\end{frame}

% --- Frame 4: Authorization, Clearing, and Settlement ---
\begin{frame}{Authorization, Clearing, and Settlement}
  \begin{columns}[T]
    \begin{column}{0.50\textwidth}
      \includegraphics[width=\textwidth]{04_payment_lifecycle_flow/chart.pdf}
    \end{column}
    \begin{column}{0.47\textwidth}
      Every card transaction passes through three stages:

      \vspace{0.4em}
      \begin{enumerate}
        \item \textcolor{mlpurple}{\textbf{Authorization}} (milliseconds) ---
              The issuer verifies the cardholder's identity, checks available
              funds or credit, and approves or declines. A hold is placed on
              the amount.
        \item \textcolor{mlpurple}{\textbf{Clearing}} (hours to one day) ---
              Transaction details are exchanged between acquirer and issuer
              via the network. Net positions are calculated.
        \item \textcolor{mlpurple}{\textbf{Settlement}} (one to three days) ---
              Actual funds transfer between issuer and acquirer banks. The
              merchant receives funds minus fees.
      \end{enumerate}

      \vspace{0.3em}
      \textbf{The gap matters:} Between authorization and settlement, the
      merchant has a \textit{promise}, not \textit{money}.
    \end{column}
  \end{columns}
  \bottomnote{Settlement timing creates working capital challenges for merchants: goods are delivered immediately, but payment arrives T+1 to T+3.}
\end{frame}

% --- Frame 5: The Global Payment Landscape ---
\begin{frame}{The Global Payment Landscape}
  \begin{columns}[T]
    \begin{column}{0.50\textwidth}
      \includegraphics[width=\textwidth]{02_global_payment_trends/chart.pdf}
    \end{column}
    \begin{column}{0.47\textwidth}
      Payment mix varies dramatically by region:

      \vspace{0.4em}
      \begin{itemize}
        \item \textbf{China:} Mobile payments (Alipay, WeChat Pay) dominate
              at over 85\% of consumer transactions. Cards were leapfrogged
              entirely.
        \item \textbf{Nordics:} Card and mobile payments exceed 95\% of
              retail volume. Cash infrastructure is actively being
              dismantled.
        \item \textbf{Germany \& Japan:} Cash remains king at 50--60\% of
              point-of-sale transactions despite high wealth and connectivity.
        \item \textbf{Sub-Saharan Africa:} Mobile money (M-Pesa model)
              serves populations with no card infrastructure.
        \item \textbf{India:} UPI processed over 12~billion transactions per
              month by late 2024, a government-driven revolution.
      \end{itemize}
    \end{column}
  \end{columns}
  \bottomnote{Global non-cash transaction volume exceeded 1.3~trillion in 2023 (Capgemini World Payments Report). Growth is concentrated in Asia-Pacific.}
\end{frame}

% --- Frame 6: The Rise of Real-Time Payments ---
\begin{frame}{The Rise of Real-Time Payments}
  \begin{columns}[T]
    \begin{column}{0.50\textwidth}
      \includegraphics[width=\textwidth]{07_realtime_payment_adoption/chart.pdf}
    \end{column}
    \begin{column}{0.47\textwidth}
      Real-time payment systems have emerged as national infrastructure:

      \vspace{0.4em}
      \begin{itemize}
        \item \textbf{UPI} (India, 2016) --- Unified Payments Interface.
              Account-to-account, interoperable, zero-fee for consumers.
              Over 400~million users.
        \item \textbf{PIX} (Brazil, 2020) --- Central bank mandated.
              Reached 150~million users in under two years. Free for
              individuals.
        \item \textbf{FedNow} (USA, 2023) --- The Federal Reserve's
              instant payment rail. Late entrant in a card-dominated market.
        \item \textbf{Faster Payments} (UK, 2008) --- Pioneer of 24/7
              settlement. Fifteen years of operational history.
        \item \textbf{SEPA Instant} (EU, 2017) --- Pan-European instant
              credit transfers. Adoption still uneven across member states.
      \end{itemize}
    \end{column}
  \end{columns}
  \bottomnote{Real-time payment volume grew 63\% year-over-year globally in 2023. India alone accounts for nearly half of all real-time transactions worldwide.}
\end{frame}

% --- Frame 7: The Merchant Cost Burden ---
\begin{frame}{The Merchant Cost Burden}
  \begin{columns}[T]
    \begin{column}{0.50\textwidth}
      \includegraphics[width=\textwidth]{05_merchant_cost_comparison/chart.pdf}
    \end{column}
    \begin{column}{0.47\textwidth}
      The cost of accepting payments varies dramatically by method:

      \vspace{0.4em}
      \begin{itemize}
        \item \textbf{Cash:} 0.5--1.5\% (handling, security, insurance,
              shrinkage). Often underestimated.
        \item \textbf{Debit card:} 0.5--1.0\% in regulated markets (EU);
              0.5--1.5\% in the US (post-Durbin).
        \item \textbf{Credit card:} 1.5--3.5\% (interchange + network fees +
              acquirer margin). Premium and rewards cards cost more.
        \item \textbf{Mobile wallet:} 0--1.5\%, depending on underlying
              funding (UPI is zero; Apple Pay passes through card fees).
        \item \textbf{BNPL:} 3--6\% merchant discount rate. The most
              expensive option, subsidized by the promise of higher
              conversion rates.
      \end{itemize}

      \vspace{0.3em}
      \textcolor{mlpurple}{Merchants do not choose payment methods ---
      consumers do. But merchants pay the cost.}
    \end{column}
  \end{columns}
  \bottomnote{In the EU, regulated interchange is capped at 0.2\% (debit) and 0.3\% (credit). In the US, credit interchange averages 2.2\% with no federal cap.}
\end{frame}

% --- Frame 8: Interchange Fees Explained ---
\begin{frame}{Interchange Fees Explained}
  \begin{columns}[T]
    \begin{column}{0.50\textwidth}
      \includegraphics[width=\textwidth]{06_interchange_fee_structure/chart.pdf}
    \end{column}
    \begin{column}{0.47\textwidth}
      The interchange fee is the largest component of the merchant discount
      rate. It flows from the \textbf{acquirer to the issuer} on every
      transaction:

      \vspace{0.4em}
      \begin{itemize}
        \item \textbf{Economic rationale:} Compensates the issuer for fraud
              risk, interest-free period (credit cards), and the cost of
              maintaining the cardholder relationship.
        \item \textbf{Set by networks:} Visa and Mastercard publish
              interchange schedules with hundreds of rate categories based
              on card type, merchant category, and transaction method.
        \item \textbf{Not negotiable:} Individual merchants cannot negotiate
              interchange rates. They can only negotiate the acquirer's
              markup above interchange.
      \end{itemize}

      \vspace{0.3em}
      \begin{block}{The Cross-Subsidy}
        Interchange funds card rewards programs: cardholders who earn
        cashback are subsidized by merchants ---
        \textcolor{mlpurple}{and ultimately by all consumers through higher
        retail prices}.
      \end{block}
    \end{column}
  \end{columns}
  \bottomnote{Baxter (1983) first formalized the economics of interchange as a balancing mechanism in two-sided markets. Rochet and Tirole (2003) extended the theory.}
\end{frame}

% --- Frame 9: Regulation Overview (Durbin and PSD2) ---
\begin{frame}{Payment Regulation: Durbin and PSD2}
  \begin{columns}[T]
    \begin{column}{0.50\textwidth}
      \textcolor{mlpurple}{\textbf{Durbin Amendment (USA, 2010):}}
      \begin{itemize}
        \item Capped debit interchange for large banks at $\approx$21c +
              0.05\% (down from 44c average --- a 45\% reduction).
        \item Routing mandate: merchants must have access to two
              unaffiliated debit networks per card.
        \item \textcolor{mlpurple}{Intended:} USD~6--8B in annual merchant
              savings.
        \item \textcolor{mlred}{Unintended:} Banks eliminated free checking
              and debit rewards; credit interchange rose as issuers shifted
              incentives --- the \textbf{waterbed effect}.
      \end{itemize}

      \vspace{0.3em}
      \begin{alertblock}{Global Comparison}
        EU caps: 0.20\% (debit), 0.30\% (credit). India (UPI): 0\%.
        USA: no credit cap. Approaches vary widely.
      \end{alertblock}
    \end{column}
    \begin{column}{0.47\textwidth}
      \textcolor{mlpurple}{\textbf{PSD2 (EU, effective 2018):}}
      \begin{itemize}
        \item \textbf{Open Banking:} Banks must provide API access to
              authorized third-party providers (TPPs) with customer consent.
        \item \textbf{PISPs} can initiate payments directly from customer
              accounts, bypassing card networks entirely --- near-zero cost
              for merchants.
        \item \textbf{Strong Customer Authentication (SCA):} Two-factor
              auth for payments above EUR~30. Conversion rates initially
              fell 20--30\% before exemption strategies matured.
      \end{itemize}

      \vspace{0.3em}
      \begin{block}{The Strategic Shift}
        PSD2 transforms banks from \textcolor{mlpurple}{gatekeepers} to
        \textcolor{mlpurple}{utilities}: they must share data and
        infrastructure with competitors.
      \end{block}
    \end{column}
  \end{columns}
  \bottomnote{PSD2 is being superseded by PSD3/PSR (proposed 2023). The Durbin Amendment is Section 1075 of the Dodd-Frank Act. Both illustrate unintended consequence risk in payment regulation.}
\end{frame}

% --- Frame 10: CBDC Design and Payment Evaluation Framework ---
\begin{frame}{CBDCs and a Payment Evaluation Framework}
  \begin{columns}[T]
    \begin{column}{0.50\textwidth}
      \includegraphics[width=\textwidth]{09_cbdc_design_comparison/chart.pdf}

      \vspace{0.4em}
      \textbf{CBDC design dimensions:}
      \begin{itemize}
        \item \textbf{Retail vs.\ Wholesale:} Public digital cash vs.\
              interbank settlement.
        \item \textbf{Account-based vs.\ token-based:} Identity-linked
              vs.\ bearer instrument.
        \item \textbf{Privacy spectrum:} Full anonymity to full
              traceability.
        \item \textbf{Interest-bearing:} Competes directly with bank
              deposits if yes.
      \end{itemize}
      \textcolor{mlpurple}{\small 130+ countries exploring CBDCs (98\% of global GDP).
      Only 3 have fully launched retail CBDCs.}
    \end{column}
    \begin{column}{0.47\textwidth}
      \textbf{Five questions to evaluate any payment system:}

      \vspace{0.3em}
      \begin{enumerate}
        \item \textbf{Who bears the cost?} Visible to payer, hidden in
              prices, or government-subsidized?
        \item \textbf{What is settlement finality?} Seconds, hours,
              or days?
        \item \textbf{How does it handle failure?} Who absorbs fraud
              losses and errors?
        \item \textbf{Who is excluded?} Bank account, smartphone,
              identity documents required?
        \item \textbf{What behavioral effects does it create?} Spending
              awareness up or down?
      \end{enumerate}

      \vspace{0.3em}
      \begin{block}{Key Takeaway}
        Payment system design is not merely an engineering problem ---
        it is a \textcolor{mlpurple}{policy choice} that determines who
        pays, who profits, and who is excluded.
      \end{block}
    \end{column}
  \end{columns}
  \bottomnote{Framework integrates L01 strategy, L02 behavioral, and L03 payment mechanics lenses. Apply it in Workshop~D to compare two payment systems.}
\end{frame}

\end{document}
