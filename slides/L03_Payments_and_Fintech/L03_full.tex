% L03_full.tex -- Lecture 3: Payments and Fintech (Full Variant)
% Frames: 31 | Charts: 12 \includegraphics | Arc: 10-role
% Generated for: Financial Technology (FinTech) -- MSc Course, Spring 2026
\documentclass[aspectratio=169, 11pt]{beamer}

% ============================================================
% THEME BASE
% ============================================================
\usetheme{Madrid}
\usecolortheme{whale}

% ============================================================
% PACKAGES
% ============================================================
\usepackage[T1]{fontenc}
\usepackage[utf8]{inputenc}
\usepackage{graphicx}
\usepackage{booktabs}
\usepackage{tikz}
\usepackage{pgfplots}
\usepackage{amsmath}
\usepackage{hyperref}
\usepackage{multicol}
\usepackage{xcolor}

% ============================================================
% TIKZ LIBRARIES
% ============================================================
\usetikzlibrary{arrows.meta, positioning, shapes.geometric, calc, decorations.pathmorphing}

% ============================================================
% PGFPLOTS COMPATIBILITY
% ============================================================
\pgfplotsset{compat=1.18}

% ============================================================
% COLOR DEFINITIONS (Fintech V4 Palette)
% ============================================================
\definecolor{MLPURPLE}{HTML}{9467BD}
\definecolor{MLBLUE}{HTML}{1F77B4}
\definecolor{MLRED}{HTML}{D62728}
\definecolor{MLORANGE}{HTML}{FF7F0E}
\definecolor{MLGREEN}{HTML}{2CA02C}
\definecolor{MLGRAY}{HTML}{7F7F7F}
\definecolor{MLTEAL}{HTML}{0D7377}
\definecolor{MLCYAN}{HTML}{14BDEB}

% Lowercase aliases for use in \textcolor{}
\colorlet{mlpurple}{MLPURPLE}
\colorlet{mlblue}{MLBLUE}
\colorlet{mlred}{MLRED}
\colorlet{mlorange}{MLORANGE}
\colorlet{mlgreen}{MLGREEN}
\colorlet{mlgray}{MLGRAY}
\colorlet{mlteal}{MLTEAL}
\colorlet{mlcyan}{MLCYAN}

% ============================================================
% BEAMER COLOR CUSTOMIZATION
% ============================================================
\setbeamercolor{structure}{fg=MLTEAL}
\setbeamercolor{palette primary}{bg=MLTEAL, fg=white}
\setbeamercolor{palette secondary}{bg=MLTEAL!80, fg=white}
\setbeamercolor{palette tertiary}{bg=MLTEAL!60, fg=white}
\setbeamercolor{palette quaternary}{bg=MLTEAL!40, fg=white}
\setbeamercolor{frametitle}{bg=MLTEAL!10, fg=MLTEAL}
\setbeamercolor{frametitle right}{bg=MLTEAL!5}
\setbeamercolor{block title}{bg=MLTEAL, fg=white}
\setbeamercolor{block body}{bg=MLTEAL!8, fg=black}
\setbeamercolor{block title alerted}{bg=MLRED, fg=white}
\setbeamercolor{block body alerted}{bg=MLRED!8, fg=black}
\setbeamercolor{block title example}{bg=MLGREEN, fg=white}
\setbeamercolor{block body example}{bg=MLGREEN!8, fg=black}
\setbeamercolor{title}{fg=white}
\setbeamercolor{subtitle}{fg=MLCYAN}
\setbeamercolor{author}{fg=white}
\setbeamercolor{institute}{fg=white}
\setbeamercolor{date}{fg=white}
\setbeamercolor{title page header}{bg=MLTEAL}

% ============================================================
% NAVIGATION AND FOOTLINE
% ============================================================
\setbeamertemplate{navigation symbols}{}

\setbeamertemplate{footline}{%
  \leavevmode%
  \hbox{%
    \begin{beamercolorbox}[wd=.333\paperwidth, ht=2.25ex, dp=1ex, center]{palette primary}%
      \usebeamerfont{author in head/foot}\insertshortauthor
    \end{beamercolorbox}%
    \begin{beamercolorbox}[wd=.334\paperwidth, ht=2.25ex, dp=1ex, center]{palette secondary}%
      \usebeamerfont{title in head/foot}\insertshorttitle
    \end{beamercolorbox}%
    \begin{beamercolorbox}[wd=.333\paperwidth, ht=2.25ex, dp=1ex, right]{palette tertiary}%
      \usebeamerfont{date in head/foot}%
      \insertframenumber{} / \inserttotalframenumber\hspace*{2ex}
    \end{beamercolorbox}%
  }%
  \vskip0pt%
}

% ============================================================
% BOTTOM NOTE COMMAND
% ============================================================
\newcommand{\bottomnote}[1]{%
  \vfill
  \begin{beamercolorbox}[wd=\textwidth, ht=2ex, dp=1ex]{palette primary}%
    \tiny\hspace{1em}#1
  \end{beamercolorbox}%
}

% ============================================================
% GRAPHICS PATH
% ============================================================
\graphicspath{{figures/}}

% ============================================================
% COURSE METADATA
% ============================================================
\title{Financial Technology (FinTech)}
\author{Joerg Osterrieder}
\institute{University of Zurich \\ Department of Finance}
\date{Spring 2026}


\subtitle{From Cash to Digital: The Transformation of Money Movement}

\begin{document}

% =============================================================================
% === WHY === (Frames 1--4: Title, Opening Cartoon, Learning Objectives, Bridge)
% =============================================================================

% --- Frame 1: Title Page ---
\begin{frame}{Title Page}
  \titlepage
\end{frame}

% --- Frame 2: Opening Cartoon ---
\begin{frame}{``Sorry, We Don't Accept That''}
  \begin{center}
    \includegraphics[width=0.85\textwidth]{11_opening_cartoon/cartoon.pdf}
  \end{center}
  \vspace{-0.5em}
  \begin{center}
    \textit{``Sorry, we don't accept cash anymore.'' --- ``But it's legal tender!''}
  \end{center}
  \bottomnote{Sweden's cash-in-circulation fell below 1\% of GDP by 2023 --- yet legal tender laws still mandate its acceptance in many jurisdictions.}
\end{frame}

% --- Frame 3: Learning Objectives ---
\begin{frame}{Learning Objectives}
  By the end of this lecture you will be able to:
  \begin{enumerate}
    \item \textbf{Describe} the evolution of payment systems from barter to
          real-time digital rails and explain the forces driving each
          transition. \hfill\textit{[Understand]}
    \item \textbf{Explain} the four-party payment model and the authorization,
          clearing, and settlement lifecycle for card-based transactions.
          \hfill\textit{[Understand]}
    \item \textbf{Apply} a cost-analysis framework to compare merchant fees
          across payment types and evaluate the impact of interchange
          regulation. \hfill\textit{[Apply]}
    \item \textbf{Analyze} how cross-border payment complexity and remittance
          costs affect financial inclusion in developing economies.
          \hfill\textit{[Analyze]}
    \item \textbf{Evaluate} the design trade-offs in central bank digital
          currencies and stablecoins as future payment infrastructure.
          \hfill\textit{[Evaluate]}
  \end{enumerate}
  \vspace{0.5em}
  \textcolor{mlpurple}{\textbf{Bloom's levels covered:}} Understand, Apply, Analyze, Evaluate
  \bottomnote{These objectives map directly to quiz and exercise assessments.}
\end{frame}

% --- Frame 4: Bridge from Lecture 2 ---
\begin{frame}{Bridge from Lecture~2}
  \begin{columns}[T]
    \begin{column}{0.55\textwidth}
      In Lecture~2 we explored the \textbf{behavioral} side of fintech:
      trust, nudging, choice architecture, and the inclusion-protection
      trade-off.

      \vspace{0.5em}
      Now we apply that lens to the largest fintech vertical:
      \textcolor{mlpurple}{\textbf{payments}}.
      \begin{itemize}
        \item \textbf{Choice architecture:} Every payment interface shapes
              spending behavior --- tap-to-pay removes the ``pain of paying.''
        \item \textbf{Trust:} Consumers entrust payment providers with every
              transaction. Trust failure here is existential.
        \item \textbf{Inclusion:} Real-time payment rails (UPI, PIX) are the
              most powerful inclusion infrastructure ever built.
      \end{itemize}

      \vspace{0.3em}
      L03 shifts from \textcolor{mlpurple}{behavioral theory} to
      \textcolor{mlpurple}{payment system design as applied choice architecture}.
    \end{column}
    \begin{column}{0.42\textwidth}
      \includegraphics[width=\textwidth]{01_payment_history_timeline/chart.pdf}
    \end{column}
  \end{columns}
  \bottomnote{L02 gave you the behavioral lens. L03 shows you the infrastructure through which billions of financial decisions flow every day.}
\end{frame}

% =============================================================================
% === FEEL === (Frame 5: Personal Connection)
% =============================================================================

% --- Frame 5: The Pain of Paying ---
\begin{frame}{The Pain of Paying}
  Think about the last time you paid for something expensive with
  \textbf{cash} --- peeling off banknotes, watching your wallet thin. Now
  compare that to tapping your phone. \textcolor{mlpurple}{The amount was
  the same. The pain was not.}

  \vspace{0.5em}
  Behavioral economists call this the \textbf{pain of paying} --- the
  negative affect experienced at the moment of parting with money. Research
  by Prelec and Loewenstein (1998) established that payment pain is a
  function of three factors:

  \vspace{0.3em}
  \begin{itemize}
    \item \textbf{Salience:} Cash is tangible; digital payments are abstract.
          Less salience means less pain.
    \item \textbf{Temporal coupling:} When payment and consumption are
          simultaneous, pain is highest. Credit cards decouple them.
    \item \textbf{Form of payment:} Physical currency activates loss aversion
          more strongly than electronic transfers.
  \end{itemize}

  \vspace{0.3em}
  \begin{exampleblock}{Quick Exercise}
    Check your last week of transactions. How many were cash? How many were
    contactless? Did you feel differently about spending in each mode?
  \end{exampleblock}
  \bottomnote{Prelec and Loewenstein (1998) and Soman (2003) show that credit card spending exceeds cash spending by 12--18\% for identical purchase decisions.}
\end{frame}

% =============================================================================
% === WHAT === (Frames 6--9: Foundational Concepts)
% =============================================================================

% --- Frame 6: A Brief History of Payments ---
\begin{frame}{A Brief History of Payments}
  \begin{itemize}
    \item \textcolor{mlpurple}{\textbf{Barter}} (pre-3000 BCE) --- Direct
          exchange of goods. Requires double coincidence of wants. Fundamentally
          unscalable.
    \item \textcolor{mlpurple}{\textbf{Commodity money and coinage}}
          (c.~600 BCE) --- Lydian electrum coins standardized value.
          Portability and divisibility enabled trade at distance.
    \item \textcolor{mlpurple}{\textbf{Paper money}} (c.~1000 CE, Song Dynasty)
          --- Promissory notes replaced heavy metal. Trust shifted from
          intrinsic value to issuer credibility.
    \item \textcolor{mlpurple}{\textbf{Checks and wire transfers}}
          (17th--19th century) --- The Bank of England (1694) and the telegraph
          (1872, Western Union) enabled non-physical value transfer.
    \item \textcolor{mlpurple}{\textbf{Payment cards}} (1950, Diners Club) ---
          Intermediated credit at point of sale. Created the multi-party model
          that still dominates.
    \item \textcolor{mlpurple}{\textbf{Digital and mobile payments}}
          (2007--present) --- M-Pesa, Apple Pay, UPI, PIX. From plastic to
          software. From batch to real-time.
  \end{itemize}

  \vspace{0.3em}
  \begin{block}{The Pattern}
    Each transition increased abstraction, reduced friction, and shifted trust
    from the medium to the \textcolor{mlpurple}{institution behind it}.
  \end{block}
  \bottomnote{The entire history of payments is a story of progressive dematerialization: from atoms to bits, from tangible to abstract.}
\end{frame}

% --- Frame 7: The Global Payment Landscape ---
\begin{frame}{The Global Payment Landscape}
  \begin{columns}[T]
    \begin{column}{0.50\textwidth}
      \includegraphics[width=\textwidth]{02_global_payment_trends/chart.pdf}
    \end{column}
    \begin{column}{0.47\textwidth}
      Payment mix varies dramatically by region:

      \vspace{0.4em}
      \begin{itemize}
        \item \textbf{China:} Mobile payments (Alipay, WeChat Pay) dominate
              at over 85\% of consumer transactions. Cards were leapfrogged
              entirely.
        \item \textbf{Nordics:} Card and mobile payments exceed 95\% of
              retail volume. Cash infrastructure is actively being
              dismantled.
        \item \textbf{Germany \& Japan:} Cash remains king at 50--60\% of
              point-of-sale transactions despite high wealth and connectivity.
        \item \textbf{Sub-Saharan Africa:} Mobile money (M-Pesa model)
              serves populations with no card infrastructure.
        \item \textbf{India:} UPI processed over 12~billion transactions per
              month by late 2024, a government-driven revolution.
      \end{itemize}
    \end{column}
  \end{columns}
  \bottomnote{Global non-cash transaction volume exceeded 1.3~trillion in 2023 (Capgemini World Payments Report). Growth is concentrated in Asia-Pacific.}
\end{frame}

% --- Frame 8: The Rise of Real-Time Payments ---
\begin{frame}{The Rise of Real-Time Payments}
  \begin{columns}[T]
    \begin{column}{0.50\textwidth}
      \includegraphics[width=\textwidth]{07_realtime_payment_adoption/chart.pdf}
    \end{column}
    \begin{column}{0.47\textwidth}
      Real-time payment systems have emerged as national infrastructure:

      \vspace{0.4em}
      \begin{itemize}
        \item \textbf{UPI} (India, 2016) --- Unified Payments Interface.
              Account-to-account, interoperable, zero-fee for consumers.
              Over 400~million users.
        \item \textbf{PIX} (Brazil, 2020) --- Central bank mandated.
              Reached 150~million users in under two years. Free for
              individuals.
        \item \textbf{FedNow} (USA, 2023) --- The Federal Reserve's
              instant payment rail. Late entrant in a card-dominated market.
        \item \textbf{Faster Payments} (UK, 2008) --- Pioneer of 24/7
              settlement. Fifteen years of operational history.
        \item \textbf{SEPA Instant} (EU, 2017) --- Pan-European instant
              credit transfers. Adoption still uneven across member states.
      \end{itemize}
    \end{column}
  \end{columns}
  \bottomnote{Real-time payment volume grew 63\% year-over-year globally in 2023. India alone accounts for nearly half of all real-time transactions worldwide.}
\end{frame}

% --- Frame 9: Why Cash Persists ---
\begin{frame}{Why Cash Persists}
  Despite the digital transition, cash remains the dominant payment method in
  most economies by transaction count. Four forces sustain it:

  \vspace{0.5em}
  \begin{itemize}
    \item \textcolor{mlpurple}{\textbf{Anonymity and privacy}} --- Cash
          leaves no digital trail. In an era of surveillance capitalism, this
          is a feature, not a bug. Approximately 25\% of Europeans cite
          privacy as a primary reason for using cash (ECB, 2022).
    \item \textcolor{mlpurple}{\textbf{Reliability}} --- Cash works without
          electricity, internet connectivity, or functioning servers. It is
          the payment method of last resort during outages and disasters.
    \item \textcolor{mlpurple}{\textbf{Zero marginal cost}} --- No
          interchange fees, no processing charges, no terminal costs. For
          small merchants, cash is the cheapest payment method.
    \item \textcolor{mlpurple}{\textbf{Cultural and behavioral inertia}} ---
          Cash provides tangible budgeting (the ``envelope method''), visible
          spending limits, and a sense of control that digital payments lack.
  \end{itemize}

  \vspace{0.3em}
  \begin{alertblock}{The Policy Tension}
    Eliminating cash without ensuring universal digital access creates a new
    form of \textcolor{mlred}{financial exclusion} --- one that
    disproportionately affects the elderly, rural populations, and the poor.
  \end{alertblock}
  \bottomnote{ECB study (2022): 59\% of eurozone point-of-sale transactions were still conducted in cash, though their share of total value was only 24\%.}
\end{frame}

% =============================================================================
% === CASE === (Frames 10--13: Payment System Mechanics)
% =============================================================================

% --- Frame 10: The Four-Party Payment Model ---
\begin{frame}{The Four-Party Payment Model}
  \begin{columns}[T]
    \begin{column}{0.50\textwidth}
      \includegraphics[width=\textwidth]{03_four_party_payment_model/chart.pdf}
    \end{column}
    \begin{column}{0.47\textwidth}
      The card payment ecosystem involves four principals:

      \vspace{0.4em}
      \begin{itemize}
        \item \textbf{Cardholder} --- The consumer who initiates the
              transaction.
        \item \textbf{Issuer} --- The cardholder's bank. Issues the card,
              extends credit or debit access, bears fraud risk.
        \item \textbf{Acquirer} --- The merchant's bank. Processes the
              transaction, deposits funds, manages merchant risk.
        \item \textbf{Network} (Visa, Mastercard) --- Sets rules, routes
              messages, guarantees interoperability. Does not hold funds.
      \end{itemize}

      \vspace{0.3em}
      \begin{block}{Key Insight}
        The network is a \textcolor{mlpurple}{two-sided platform}: it must
        attract both cardholders (via issuers) and merchants (via acquirers)
        simultaneously.
      \end{block}
    \end{column}
  \end{columns}
  \bottomnote{Visa and Mastercard together process over 80\% of global card transactions. American Express and Discover operate three-party (closed-loop) models.}
\end{frame}

% --- Frame 11: Authorization, Clearing, and Settlement ---
\begin{frame}{Authorization, Clearing, and Settlement}
  \begin{columns}[T]
    \begin{column}{0.50\textwidth}
      \includegraphics[width=\textwidth]{04_payment_lifecycle_flow/chart.pdf}
    \end{column}
    \begin{column}{0.47\textwidth}
      Every card transaction passes through three stages:

      \vspace{0.4em}
      \begin{enumerate}
        \item \textcolor{mlpurple}{\textbf{Authorization}} (milliseconds) ---
              The issuer verifies the cardholder's identity, checks available
              funds or credit, and approves or declines. A hold is placed on
              the amount.
        \item \textcolor{mlpurple}{\textbf{Clearing}} (hours to one day) ---
              Transaction details are exchanged between acquirer and issuer
              via the network. Net positions are calculated.
        \item \textcolor{mlpurple}{\textbf{Settlement}} (one to three days) ---
              Actual funds transfer between issuer and acquirer banks. The
              merchant receives funds minus fees.
      \end{enumerate}

      \vspace{0.3em}
      \textbf{The gap matters:} Between authorization and settlement, the
      merchant has a \textit{promise}, not \textit{money}.
    \end{column}
  \end{columns}
  \bottomnote{Settlement timing creates working capital challenges for merchants: goods are delivered immediately, but payment arrives T+1 to T+3.}
\end{frame}

% --- Frame 12: Batch vs. Real-Time Processing ---
\begin{frame}{Batch vs.\ Real-Time Processing}
  \begin{columns}[T]
    \begin{column}{0.50\textwidth}
      \textbf{Why does settlement still take days?}

      \vspace{0.5em}
      The card networks were designed in the 1960s--1970s for
      \textbf{batch processing}. Transactions are accumulated throughout the
      day and settled in bulk overnight. This architecture persists because:

      \vspace{0.4em}
      \begin{itemize}
        \item \textbf{Netting efficiency:} Batch settlement allows bilateral
              netting, reducing the total volume of interbank transfers by
              80--90\%.
        \item \textbf{Fraud windows:} The delay allows time for fraud
              detection, chargeback initiation, and dispute resolution.
        \item \textbf{Liquidity management:} Banks prefer predictable,
              scheduled settlement over continuous real-time obligations.
      \end{itemize}
    \end{column}
    \begin{column}{0.47\textwidth}
      \textbf{The real-time alternative:}

      \vspace{0.5em}
      Systems like UPI and PIX settle in seconds. But real-time settlement
      requires:
      \begin{itemize}
        \item Pre-funded accounts or central bank liquidity facilities
        \item Real-time fraud detection (no chargeback window)
        \item 24/7/365 operational infrastructure
        \item Irrevocability --- once settled, funds cannot be recalled
              through the system
      \end{itemize}

      \vspace{0.3em}
      \begin{alertblock}{Trade-off}
        Real-time settlement trades \textcolor{mlred}{fraud protection and
        netting efficiency} for \textcolor{mlpurple}{speed and finality}.
      \end{alertblock}
    \end{column}
  \end{columns}
  \bottomnote{The UK Faster Payments system processes over 4~billion transactions annually with real-time settlement and fraud rates comparable to batch systems.}
\end{frame}

% --- Frame 13: Cross-Border Payment Complexity ---
\begin{frame}{Cross-Border Payment Complexity}
  \begin{columns}[T]
    \begin{column}{0.50\textwidth}
      \includegraphics[width=\textwidth]{08_cross_border_payment_flows/chart.pdf}
    \end{column}
    \begin{column}{0.47\textwidth}
      Cross-border payments remain the most expensive, slowest, and least
      transparent segment of the payment system:

      \vspace{0.4em}
      \begin{itemize}
        \item \textbf{Correspondent banking:} Most cross-border payments
              traverse a chain of intermediary banks, each adding fees,
              delays, and opacity.
        \item \textbf{SWIFT:} A messaging network (not a settlement system).
              SWIFT carries payment instructions; actual settlement occurs
              through nostro/vostro account relationships.
        \item \textbf{FX conversion:} Each hop may involve a currency
              conversion with opaque markup, typically 1--4\% above
              mid-market rates.
        \item \textbf{Compliance layers:} AML/KYC checks at each
              intermediary add processing time (typically 2--5 business days
              end-to-end).
      \end{itemize}
    \end{column}
  \end{columns}
  \vspace{0.3em}
  \begin{block}{The G20 Target}
    The G20 set a 2027 target: cross-border payments should cost no more than
    3\%, arrive within one hour, and be available to all. Current averages:
    \textcolor{mlpurple}{6.2\%, 2--5 days, limited access}.
  \end{block}
  \bottomnote{The World Bank Remittance Prices Worldwide database tracks costs across 365 corridors. Sub-Saharan Africa remains the most expensive region.}
\end{frame}

% =============================================================================
% === HOW === (Frames 14--17: Payment Economics)
% =============================================================================

% --- Frame 14: The Merchant Cost Burden ---
\begin{frame}{The Merchant Cost Burden}
  \begin{columns}[T]
    \begin{column}{0.50\textwidth}
      \includegraphics[width=\textwidth]{05_merchant_cost_comparison/chart.pdf}
    \end{column}
    \begin{column}{0.47\textwidth}
      The cost of accepting payments varies dramatically by method:

      \vspace{0.4em}
      \begin{itemize}
        \item \textbf{Cash:} 0.5--1.5\% (handling, security, insurance,
              shrinkage). Often underestimated.
        \item \textbf{Debit card:} 0.5--1.0\% in regulated markets (EU);
              0.5--1.5\% in the US (post-Durbin).
        \item \textbf{Credit card:} 1.5--3.5\% (interchange + network fees +
              acquirer margin). Premium and rewards cards cost more.
        \item \textbf{Mobile wallet:} 0--1.5\%, depending on underlying
              funding (UPI is zero; Apple Pay passes through card fees).
        \item \textbf{BNPL:} 3--6\% merchant discount rate. The most
              expensive option, subsidized by the promise of higher
              conversion rates.
      \end{itemize}

      \vspace{0.3em}
      \textcolor{mlpurple}{Merchants do not choose payment methods ---
      consumers do. But merchants pay the cost.}
    \end{column}
  \end{columns}
  \bottomnote{In the EU, regulated interchange is capped at 0.2\% (debit) and 0.3\% (credit). In the US, credit interchange averages 2.2\% with no federal cap.}
\end{frame}

% --- Frame 15: Interchange Fees Explained ---
\begin{frame}{Interchange Fees Explained}
  \begin{columns}[T]
    \begin{column}{0.50\textwidth}
      \includegraphics[width=\textwidth]{06_interchange_fee_structure/chart.pdf}
    \end{column}
    \begin{column}{0.47\textwidth}
      The interchange fee is the largest component of the merchant discount
      rate. It flows from the \textbf{acquirer to the issuer} on every
      transaction:

      \vspace{0.4em}
      \begin{itemize}
        \item \textbf{Economic rationale:} Compensates the issuer for fraud
              risk, interest-free period (credit cards), and the cost of
              maintaining the cardholder relationship.
        \item \textbf{Set by networks:} Visa and Mastercard publish
              interchange schedules with hundreds of rate categories based
              on card type, merchant category, and transaction method.
        \item \textbf{Not negotiable:} Individual merchants cannot negotiate
              interchange rates. They can only negotiate the acquirer's
              markup above interchange.
      \end{itemize}

      \vspace{0.3em}
      \begin{block}{The Cross-Subsidy}
        Interchange funds card rewards programs: cardholders who earn
        cashback are subsidized by merchants ---
        \textcolor{mlpurple}{and ultimately by all consumers through higher
        retail prices}.
      \end{block}
    \end{column}
  \end{columns}
  \bottomnote{Baxter (1983) first formalized the economics of interchange as a balancing mechanism in two-sided markets. Rochet and Tirole (2003) extended the theory.}
\end{frame}

% --- Frame 16: Payment Facilitators and Aggregators ---
\begin{frame}{Payment Facilitators and Aggregators}
  \begin{columns}[T]
    \begin{column}{0.50\textwidth}
      The traditional acquiring model requires each merchant to establish a
      direct relationship with an acquirer: application, underwriting, risk
      assessment, and a dedicated merchant account.

      \vspace{0.5em}
      \textbf{Payment facilitators} (PayFacs) collapsed this process:

      \vspace{0.4em}
      \begin{itemize}
        \item \textbf{Stripe} (2010) --- Aggregates merchants under a single
              master merchant account. Onboarding reduced from weeks to
              minutes.
        \item \textbf{Square} (2009) --- Combined hardware (card reader) with
              software aggregation. Brought card acceptance to micro-merchants.
        \item \textbf{Adyen} (2006) --- Full-stack acquiring for enterprise.
              Single platform across geographies.
      \end{itemize}
    \end{column}
    \begin{column}{0.47\textwidth}
      \textbf{The PayFac economics:}

      \vspace{0.5em}
      \begin{itemize}
        \item PayFacs charge a flat rate (e.g., 2.9\% + \$0.30) regardless
              of card type. Simple, but often above cost.
        \item They absorb underwriting risk: if a sub-merchant commits
              fraud, the PayFac is liable.
        \item Revenue model: spread between blended rate charged and actual
              interchange plus network fees paid.
      \end{itemize}

      \vspace{0.3em}
      \begin{block}{The Trade-off}
        PayFacs offer \textcolor{mlpurple}{speed and simplicity} at the cost
        of \textcolor{mlpurple}{higher per-transaction fees}. As merchants
        grow, they graduate to direct acquiring relationships.
      \end{block}
    \end{column}
  \end{columns}
  \bottomnote{Stripe's 2023 valuation of approximately USD~50B reflects the market's belief that aggregation will capture an increasing share of global payment volume.}
\end{frame}

% --- Frame 17: Small vs. Large Merchant Economics ---
\begin{frame}{Small vs.\ Large Merchant Economics}
  Payment costs are \textbf{regressive}: they disproportionately burden small
  merchants.

  \vspace{0.5em}
  \begin{columns}[T]
    \begin{column}{0.50\textwidth}
      \textcolor{mlpurple}{\textbf{Small merchants}} (under USD~500K annual
      volume):
      \begin{itemize}
        \item Pay blended rates (2.5--3.5\%) with no interchange visibility
        \item Cannot negotiate acquirer markup
        \item Fixed per-transaction fees (\$0.20--0.30) erode margins on
              small tickets
        \item Terminal rental, PCI compliance, and chargeback fees add
              fixed costs
        \item A coffee shop selling a CHF~4.50 latte at 2.9\% + CHF~0.30
              pays an effective rate of \textbf{9.6\%}
      \end{itemize}
    \end{column}
    \begin{column}{0.47\textwidth}
      \textcolor{mlpurple}{\textbf{Large merchants}} (over USD~50M annual
      volume):
      \begin{itemize}
        \item Negotiate interchange-plus pricing with full transparency
        \item Acquirer margins as low as 0.05--0.10\%
        \item Can route transactions to lowest-cost networks
        \item Operate their own fraud detection, reducing chargebacks
        \item Effective all-in rate: 1.0--1.5\%
      \end{itemize}

      \vspace{0.3em}
      \begin{alertblock}{The Inequality}
        The smallest merchants pay the highest effective rates. Payment
        costs are a \textcolor{mlred}{regressive tax on small business}.
      \end{alertblock}
    \end{column}
  \end{columns}
  \bottomnote{The fixed per-transaction component creates a ``small ticket problem'': for transactions under USD~5, payment costs can exceed 10\% of transaction value.}
\end{frame}

% =============================================================================
% === RISK === (Frames 18--20: Regulatory Responses)
% =============================================================================

% --- Frame 18: The Durbin Amendment ---
\begin{frame}{The Durbin Amendment}
  \begin{columns}[T]
    \begin{column}{0.50\textwidth}
      The \textbf{Durbin Amendment} (2010, effective 2011) was the most
      significant US payment regulation in decades:

      \vspace{0.4em}
      \begin{itemize}
        \item \textbf{Scope:} Capped debit interchange fees for banks with
              over USD~10~billion in assets at approximately 21 cents + 0.05\%
              per transaction (plus a 1-cent fraud adjustment).
        \item \textbf{Pre-Durbin average:} 44 cents per transaction.
        \item \textbf{Post-Durbin average:} 24 cents per transaction ---
              a 45\% reduction.
        \item \textbf{Routing mandate:} Merchants must have access to at
              least two unaffiliated debit networks per card, enabling
              least-cost routing.
      \end{itemize}
    \end{column}
    \begin{column}{0.47\textwidth}
      \textbf{Intended and unintended consequences:}

      \vspace{0.5em}
      \begin{itemize}
        \item \textcolor{mlpurple}{Intended:} Lower costs for merchants,
              especially large retailers. Estimated annual merchant savings
              of USD~6--8~billion.
        \item \textcolor{mlred}{Unintended:} Banks eliminated free checking
              accounts and debit rewards programs to recover lost revenue.
              Small banks (exempt from the cap) saw limited benefit as
              merchants routed to the cheapest network regardless.
        \item \textcolor{mlred}{Unintended:} Credit card interchange ---
              not covered by Durbin --- actually \textit{increased} as
              issuers shifted incentives toward credit products.
      \end{itemize}
    \end{column}
  \end{columns}
  \bottomnote{The Durbin Amendment is Section 1075 of the Dodd-Frank Act. It illustrates the ``waterbed effect'': capping one fee causes others to rise.}
\end{frame}

% --- Frame 19: PSD2 and Open Banking ---
\begin{frame}{PSD2 and Open Banking}
  The EU's \textbf{Payment Services Directive 2} (PSD2, effective 2018)
  reshaped European payments through two mechanisms:

  \vspace{0.5em}
  \begin{columns}[T]
    \begin{column}{0.50\textwidth}
      \textcolor{mlpurple}{\textbf{Open Banking (Access to Accounts):}}
      \begin{itemize}
        \item Banks must provide API access to authorized third-party
              providers (TPPs) --- with customer consent.
        \item \textbf{AISPs} (Account Information Service Providers) can
              aggregate account data across institutions.
        \item \textbf{PISPs} (Payment Initiation Service Providers) can
              initiate payments directly from customer accounts, bypassing
              card networks entirely.
        \item Enables bank-to-bank payments at near-zero cost to merchants.
      \end{itemize}
    \end{column}
    \begin{column}{0.47\textwidth}
      \textcolor{mlpurple}{\textbf{Strong Customer Authentication (SCA):}}
      \begin{itemize}
        \item Two-factor authentication required for electronic payments
              above EUR~30 (with exemptions for trusted beneficiaries
              and low-risk transactions).
        \item Friction vs.\ security trade-off: conversion rates initially
              dropped 20--30\% for e-commerce merchants before exemption
              strategies matured.
      \end{itemize}

      \vspace{0.3em}
      \begin{block}{The Strategic Shift}
        PSD2 transforms banks from \textcolor{mlpurple}{gatekeepers} to
        \textcolor{mlpurple}{utilities}: they must share data and
        infrastructure with competitors.
      \end{block}
    \end{column}
  \end{columns}
  \bottomnote{PSD2 is being superseded by PSD3/PSR (proposed 2023), which extends open banking to instant payments and introduces liability clarity for fraud.}
\end{frame}

% --- Frame 20: Global Interchange Regulation ---
\begin{frame}{Global Interchange Regulation}
  Interchange regulation has spread globally, but with widely varying approaches:

  \vspace{0.5em}
  \begin{columns}[T]
    \begin{column}{0.50\textwidth}
      \vspace{-0.5em}
      \begin{tabular}{@{}l l l@{}}
        \toprule
        \textbf{Jurisdiction} & \textbf{Debit Cap} & \textbf{Credit Cap} \\
        \midrule
        EU / EEA         & 0.20\%       & 0.30\% \\
        Australia         & 0.08 AUD avg & 0.50\% avg \\
        USA (Durbin)      & 21c + 0.05\% & No cap \\
        India (UPI/RuPay) & 0\%          & 0\% \\
        China             & 0.35\% max   & 0.45\% max \\
        Canada            & Voluntary    & Voluntary \\
        UK (post-Brexit)  & 0.20\%       & 0.30\% \\
        \bottomrule
      \end{tabular}

      \vspace{0.5em}
      \textcolor{mlpurple}{India is the only major economy with zero
      interchange} --- subsidized by government as inclusion policy.
    \end{column}
    \begin{column}{0.47\textwidth}
      \textbf{The regulatory dilemma:}

      \vspace{0.5em}
      \begin{itemize}
        \item \textcolor{mlpurple}{Pro-regulation:} Interchange is a hidden
              tax on consumers. Caps lower prices and improve transparency.
        \item \textcolor{mlred}{Anti-regulation:} Caps reduce issuer revenue,
              leading to fewer card benefits, higher account fees, and
              reduced credit availability for marginal borrowers.
        \item \textbf{Evidence:} EU interchange regulation (IFR, 2015)
              reduced merchant costs by approximately EUR~5~billion annually,
              but consumer price pass-through has been incomplete and slow.
      \end{itemize}
    \end{column}
  \end{columns}
  \bottomnote{The ``merchant indifference test'' (tourist test) --- the theoretical basis for EU interchange caps --- asks: at what fee level is a merchant indifferent between cash and cards?}
\end{frame}

% =============================================================================
% === WHERE === (Frames 21--23: Evidence at Scale)
% =============================================================================

% --- Frame 21: Real-Time Payment Systems Compared ---
\begin{frame}{Real-Time Payment Systems Compared}
  \begin{columns}[T]
    \begin{column}{0.31\textwidth}
      \begin{block}{\centering UPI (India)}
        \textbf{Launch:} 2016 (NPCI)

        \vspace{0.3em}
        \textbf{Model:} Account-to-account via virtual payment address (VPA).
        Interoperable across all banks.

        \vspace{0.3em}
        \textbf{Cost:} Zero for consumers and merchants (government-subsidized).

        \vspace{0.3em}
        \textbf{Scale:} 12B+ transactions/month. 400M+ users.

        \vspace{0.3em}
        \textit{Lesson:} Government mandate + zero cost = explosive adoption.
      \end{block}
    \end{column}
    \begin{column}{0.31\textwidth}
      \begin{block}{\centering PIX (Brazil)}
        \textbf{Launch:} 2020 (Banco Central do Brasil)

        \vspace{0.3em}
        \textbf{Model:} Instant payment via CPF (tax ID), phone, email, or
        QR code. Central bank operated.

        \vspace{0.3em}
        \textbf{Cost:} Free for individuals. Small fee for businesses.

        \vspace{0.3em}
        \textbf{Scale:} 150M+ users in 2 years. 3.5B+ transactions/month.

        \vspace{0.3em}
        \textit{Lesson:} Central bank infrastructure can leapfrog card networks.
      \end{block}
    \end{column}
    \begin{column}{0.31\textwidth}
      \begin{block}{\centering FedNow (USA)}
        \textbf{Launch:} 2023 (Federal Reserve)

        \vspace{0.3em}
        \textbf{Model:} Bank-to-bank instant settlement. Voluntary bank
        participation.

        \vspace{0.3em}
        \textbf{Cost:} Banks set consumer pricing. No mandated zero fee.

        \vspace{0.3em}
        \textbf{Scale:} Fewer than 1,000 participating banks (of 10,000+)
        by end 2024.

        \vspace{0.3em}
        \textit{Lesson:} Voluntary adoption in a card-dominated market is slow.
      \end{block}
    \end{column}
  \end{columns}
  \bottomnote{The contrast between UPI/PIX (government-mandated, zero-cost) and FedNow (voluntary, market-priced) illustrates how policy design determines adoption speed.}
\end{frame}

% --- Frame 22: Cross-Border Payment Corridors ---
\begin{frame}{Cross-Border Payment Corridors}
  \begin{columns}[T]
    \begin{column}{0.50\textwidth}
      Cross-border payments serve two fundamentally different markets:

      \vspace{0.4em}
      \textbf{Wholesale (B2B and interbank):}
      \begin{itemize}
        \item SWIFT carries 45~million messages per day across 11,000+
              institutions in 200+ countries
        \item Average transaction: USD~500,000+
        \item Settlement via correspondent banking (nostro/vostro accounts)
        \item CLS (Continuous Linked Settlement) handles FX settlement for
              18 currencies, eliminating Herstatt risk
      \end{itemize}

      \vspace{0.3em}
      \textbf{Retail (remittances):}
      \begin{itemize}
        \item USD~656~billion in remittance flows to LMICs in 2022
              (World Bank)
        \item Average transaction: USD~200--500
        \item Average cost: 6.2\% of transaction value
      \end{itemize}
    \end{column}
    \begin{column}{0.47\textwidth}
      \textbf{Emerging alternatives to correspondent banking:}

      \vspace{0.5em}
      \begin{itemize}
        \item \textbf{Wise (TransferWise):} Peer-to-peer matching of
              opposite-direction flows. Avoids correspondent chain for
              major corridors.
        \item \textbf{Ripple / XRP Ledger:} Pre-funded liquidity pools for
              instant cross-border settlement. Adoption limited to
              non-major corridors.
        \item \textbf{Bilateral rail links:} UPI-PayNow (India-Singapore),
              UPI-PIX (India-Brazil). Direct account-to-account across
              borders.
        \item \textbf{Project mBridge:} Multi-CBDC bridge (BIS Innovation
              Hub) connecting central banks for wholesale settlement.
      \end{itemize}
    \end{column}
  \end{columns}
  \bottomnote{Remittances exceed foreign direct investment as a source of external finance for many developing economies. The Philippines, India, and Mexico are the largest recipients.}
\end{frame}

% --- Frame 23: The Cost of Sending Money Abroad ---
\begin{frame}{The Cost of Sending Money Abroad}
  The global average cost of sending USD~200 remains \textbf{6.2\%}
  (World Bank, Q3 2024) --- more than double the UN Sustainable Development
  Goal target of 3\%.

  \vspace{0.5em}
  \begin{columns}[T]
    \begin{column}{0.50\textwidth}
      \textbf{Where costs are highest:}
      \begin{itemize}
        \item Sub-Saharan Africa: 7.9\% average (South Africa to
              neighboring countries exceeds 15\%)
        \item Small Pacific Island nations: 8--10\% (low volumes,
              limited competition)
        \item South-South corridors: Often 10\%+ due to thin
              correspondent banking relationships
      \end{itemize}

      \vspace{0.3em}
      \textbf{Where costs are lowest:}
      \begin{itemize}
        \item UAE to India: 2.5\% (high volume, strong competition)
        \item GCC to South Asia: 3--4\% (mobile money operators compete)
        \item Intra-EU (SEPA): near zero for EUR transfers
      \end{itemize}
    \end{column}
    \begin{column}{0.47\textwidth}
      \textbf{What drives cost?}

      \vspace{0.5em}
      \begin{enumerate}
        \item \textbf{FX markup:} Typically 1--4\% above mid-market rate,
              often hidden from the sender.
        \item \textbf{Transfer fees:} Fixed and percentage components.
              Regressively burden small senders.
        \item \textbf{Correspondent charges:} Intermediary bank fees
              deducted from the remitted amount, often unpredictable.
        \item \textbf{Regulatory compliance:} AML/KYC costs passed to
              consumers, especially in high-risk corridors.
      \end{enumerate}

      \vspace{0.3em}
      \textcolor{mlpurple}{Every percentage point reduction in remittance
      costs releases USD~6.5~billion annually for recipient families.}
    \end{column}
  \end{columns}
  \bottomnote{SDG Target 10.c: reduce remittance costs to below 3\% by 2030. At current pace of decline (0.2 pp/year), the target will not be met until after 2040.}
\end{frame}

% =============================================================================
% === IMPACT === (Frames 24--25: CBDC and Stablecoins)
% =============================================================================

% --- Frame 24: Central Bank Digital Currencies ---
\begin{frame}{Central Bank Digital Currencies}
  \begin{columns}[T]
    \begin{column}{0.50\textwidth}
      \includegraphics[width=\textwidth]{09_cbdc_design_comparison/chart.pdf}
    \end{column}
    \begin{column}{0.47\textwidth}
      CBDCs are digital liabilities of a central bank, available to the
      public or to financial institutions:

      \vspace{0.4em}
      \begin{itemize}
        \item \textbf{Retail CBDC:} Digital cash for consumers. Direct claim
              on the central bank. Raises questions about bank
              disintermediation.
        \item \textbf{Wholesale CBDC:} Restricted to financial institutions
              for interbank settlement. Less disruptive, more immediately
              practical.
      \end{itemize}

      \vspace{0.3em}
      \textbf{Design dimensions:}
      \begin{itemize}
        \item \textbf{Account-based vs.\ token-based:} Identity-linked
              accounts (like bank deposits) vs.\ bearer instruments (like
              cash).
        \item \textbf{Interest-bearing vs.\ non-interest:} An
              interest-bearing CBDC competes directly with bank deposits.
        \item \textbf{Privacy spectrum:} Full anonymity (like cash) to full
              traceability (like bank transfers).
      \end{itemize}
    \end{column}
  \end{columns}
  \vspace{0.3em}
  \begin{block}{The Central Question}
    \textcolor{mlpurple}{A CBDC forces a society to decide what it values
    most: privacy, financial stability, monetary control, or inclusion.
    No design satisfies all four simultaneously.}
  \end{block}
  \bottomnote{As of 2024, 130+ countries (representing 98\% of global GDP) are exploring CBDCs. Only 3 have fully launched retail CBDCs: The Bahamas, Jamaica, and Nigeria.}
\end{frame}

% --- Frame 25: Stablecoins and Payment Rails ---
\begin{frame}{Stablecoins and Payment Rails}
  \begin{columns}[T]
    \begin{column}{0.50\textwidth}
      Stablecoins --- tokens pegged to fiat currencies --- have emerged as
      \textbf{de facto payment infrastructure}, particularly for cross-border
      and crypto-native transactions:

      \vspace{0.4em}
      \begin{itemize}
        \item \textbf{USDT (Tether):} Market cap exceeding USD~95~billion
              (2024). Dominant in crypto trading pairs. Reserve composition
              has been a persistent transparency concern.
        \item \textbf{USDC (Circle):} USD~30~billion. Positions itself as
              the ``regulated'' stablecoin with full attestation reports
              and US money transmitter licenses.
        \item \textbf{PYUSD (PayPal):} Launched 2023. First stablecoin from
              a major payment company. Signals mainstream institutional
              entry.
      \end{itemize}
    \end{column}
    \begin{column}{0.47\textwidth}
      \textbf{Stablecoins as payment rails:}

      \vspace{0.5em}
      \begin{itemize}
        \item Settlement in minutes, not days, across borders
        \item 24/7/365 availability (no banking hours)
        \item Programmable: escrow, conditional release, automated
              compliance
        \item Used for remittances in corridors with limited banking
              infrastructure (e.g., Turkey, Argentina, Nigeria)
      \end{itemize}

      \vspace{0.3em}
      \begin{alertblock}{The Risk}
        Stablecoins are only as stable as their reserves. The TerraUSD
        collapse (May 2022, USD~60~billion destroyed) demonstrated that
        \textcolor{mlred}{algorithmic pegs can fail catastrophically}.
      \end{alertblock}
    \end{column}
  \end{columns}
  \bottomnote{On-chain stablecoin settlement volume exceeded USD~7~trillion in 2023 (Chainalysis). Most volume is institutional, not retail.}
\end{frame}

% =============================================================================
% === SO WHAT === (Frames 26--27: Synthesis and Evaluation)
% =============================================================================

% --- Frame 26: Embedded and Invisible Payments ---
\begin{frame}{Embedded and Invisible Payments}
  \begin{columns}[T]
    \begin{column}{0.55\textwidth}
      The end state of payment innovation is the
      \textcolor{mlpurple}{\textbf{disappearance of the payment moment itself}}.

      \vspace{0.4em}
      \begin{itemize}
        \item \textbf{One-click buy buttons:} Amazon's patented 1-Click
              (1999) eliminated the checkout flow. Conversion increased by
              an estimated 5\% per friction step removed.
        \item \textbf{Ride-hailing:} Uber, Bolt, and Grab charge
              automatically at trip end. The passenger never ``pays'' ---
              the payment is embedded in the service.
        \item \textbf{Subscription models:} Netflix, Spotify, and SaaS
              platforms collect recurring payments invisibly. Churn
              reduction through payment invisibility.
        \item \textbf{IoT payments:} Connected cars paying tolls, smart
              refrigerators reordering groceries, industrial machines
              ordering their own replacement parts.
      \end{itemize}
    \end{column}
    \begin{column}{0.42\textwidth}
      \includegraphics[width=\textwidth]{10_payment_innovation_timeline/chart.pdf}

      \vspace{0.5em}
      \begin{alertblock}{The Behavioral Concern}
        Invisible payments maximize convenience but minimize the
        \textcolor{mlred}{pain of paying} --- the very friction that helps
        consumers control spending. L02's behavioral lens applies directly:
        \textit{less friction means less deliberation}.
      \end{alertblock}
    \end{column}
  \end{columns}
  \bottomnote{Embedded finance (payments integrated into non-financial platforms) is projected to reach USD~7~trillion in transaction value by 2026 (Bain \& Company).}
\end{frame}

% --- Frame 27: A Payment Evaluation Framework ---
\begin{frame}{A Payment Evaluation Framework}
  \begin{columns}[T]
    \begin{column}{0.55\textwidth}
      Five questions to evaluate any payment system, method, or innovation:

      \vspace{0.4em}
      \begin{enumerate}
        \item \textbf{Who bears the cost?}\\
              Is the cost visible to the payer, hidden in merchant prices,
              or subsidized by government?
        \item \textbf{What is the settlement finality?}\\
              When does the recipient have irrevocable access to funds?
              Minutes, hours, or days?
        \item \textbf{How does it handle failure?}\\
              What happens when the payment goes wrong? Who absorbs fraud
              losses, chargebacks, and errors?
        \item \textbf{Who is excluded?}\\
              What prerequisites does it require --- bank account,
              smartphone, identity documents, internet access?
        \item \textbf{What behavioral effects does it create?}\\
              Does it increase or decrease spending awareness? Does it
              nudge toward saving or consumption?
      \end{enumerate}
    \end{column}
    \begin{column}{0.42\textwidth}
      \textbf{Applying the framework:}

      \vspace{0.4em}
      \begin{block}{Cash}
        Cost: payer. Finality: instant. Failure: bearer risk. Exclusion:
        none. Behavior: high spending awareness.
      \end{block}

      \vspace{0.3em}
      \begin{block}{Credit Card}
        Cost: merchant (interchange). Finality: T+1 to T+3. Failure:
        chargeback (consumer protected). Exclusion: credit score required.
        Behavior: reduced spending pain.
      \end{block}

      \vspace{0.3em}
      \begin{block}{UPI / PIX}
        Cost: government-subsidized. Finality: seconds. Failure: limited
        recourse. Exclusion: smartphone + bank account. Behavior: variable.
      \end{block}
    \end{column}
  \end{columns}
  \bottomnote{This framework integrates L01's strategy lens, L02's behavioral lens, and L03's payment mechanics into a single evaluation tool.}
\end{frame}

% =============================================================================
% === ACT === (Frame 28: Forward Look)
% =============================================================================

% --- Frame 28: What Comes Next ---
\begin{frame}{What Comes Next}
  \begin{itemize}
    \item \textbf{Next:} L04 (Fintech Security and Regulation --- RegTech) ---
          How regulators are responding to the payment innovations discussed
          today. AML/KYC automation, regulatory sandboxes, and the rise of
          supervisory technology.
    \item \textbf{Before L04, reflect:} Trace your last online purchase from
          tap to settlement. How many intermediaries touched your money? What
          did each one charge? Could the transaction have been routed more
          cheaply?
    \item \textbf{Workshop preparation:} Review the payment evaluation
          framework (Frame~27). You will use it to compare two payment systems
          in Workshop~D.
  \end{itemize}

  \vspace{0.8em}
  \begin{block}{Why Regulation Matters}
    Every payment innovation discussed today --- real-time rails, stablecoins,
    CBDCs, embedded payments --- \textcolor{mlpurple}{exists within a regulatory
    framework that determines who can build it, who can use it, and who is
    protected when it fails}. L04 examines that framework.
  \end{block}

  \vspace{0.5em}
  \begin{exampleblock}{Course Progress}
    \small
    L01: Foundations~\checkmark \(\bullet\) L02: Ecosystem~\checkmark
    \(\bullet\) \textbf{L03: Payments}~\checkmark \(\bullet\)
    L04: Regulation \(\bullet\) L05: Wealth Mgmt \(\bullet\)
    L06: Insurance \(\bullet\) L07: Technology
  \end{exampleblock}
  \bottomnote{L04 builds directly on L03: payment regulation (PSD2, Durbin, interchange caps) is the bridge between payment infrastructure and regulatory architecture.}
\end{frame}

% =============================================================================
% === CLOSING === (Frames 29--31: Closing Cartoon, Key Takeaways, Summary)
% =============================================================================

% --- Frame 29: Closing Cartoon ---
\begin{frame}{``Remember When We Were the Future?''}
  \begin{center}
    \includegraphics[width=0.85\textwidth]{12_closing_cartoon/cartoon.pdf}
  \end{center}
  \vspace{-0.5em}
  \begin{center}
    \textit{``Remember when we were the future?'' --- Two ATMs watching a
    customer pay with their phone.}
  \end{center}
  \bottomnote{Global ATM numbers peaked in 2018 and have been declining since. The UK alone has lost over 15,000 ATMs since 2015.}
\end{frame}

% --- Frame 30: Key Takeaways ---
\begin{frame}{Key Takeaways}
  \begin{enumerate}
    \item \textbf{Payment history is dematerialization:} Every transition ---
          from barter to coinage to paper to digital --- increased
          abstraction and shifted trust from the medium to the institution.
    \item \textbf{The four-party model:} Card payments flow through issuer,
          acquirer, network, and merchant. Understanding this chain is
          essential to understanding payment costs.
    \item \textbf{Settlement is not instant:} Authorization takes milliseconds,
          but traditional settlement takes days. Real-time systems (UPI, PIX)
          are closing this gap, with trade-offs.
    \item \textbf{Payment costs are regressive:} Small merchants pay the
          highest effective rates. Interchange is a hidden cross-subsidy from
          merchants to cardholders.
    \item \textbf{Cross-border payments remain broken:} Average remittance
          costs of 6.2\% represent a multi-billion-dollar tax on the world's
          poorest populations.
    \item \textbf{CBDCs force design choices:} Privacy vs.\ traceability,
          interest vs.\ non-interest, account-based vs.\ token-based. No
          single design satisfies all objectives.
    \item \textbf{Invisible payments remove friction:} Embedded payments
          maximize convenience but eliminate the behavioral friction that
          supports deliberate spending.
  \end{enumerate}
  \bottomnote{Review question: Apply the five-question payment evaluation framework to a payment method you used today. What did you discover?}
\end{frame}

% --- Frame 31: Summary and Key Vocabulary ---
\begin{frame}{Summary and Key Vocabulary}
  \textbf{Summary:} Payment systems are the circulatory system of the
  financial economy --- and they are undergoing their most profound
  transformation since the invention of the credit card. Real-time domestic
  rails are replacing batch settlement, open banking is challenging card
  network dominance, and CBDCs and stablecoins are redefining what
  ``money'' means. Yet cross-border payments remain slow and expensive,
  interchange economics disproportionately burden small merchants, and the
  progressive \textcolor{mlpurple}{invisibility of payments} raises
  behavioral concerns about consumer spending control. The central lesson of
  L03 is that \textit{payment system design is not merely an engineering
  problem --- it is a policy choice that determines who pays, who profits,
  and who is excluded}.

  \vspace{0.5em}
  \textbf{Key Vocabulary:}
  \begin{multicols}{2}
    \begin{itemize}
      \item Four-Party Model
      \item Interchange Fee
      \item Authorization / Clearing / Settlement
      \item Real-Time Payments (UPI, PIX)
      \item Correspondent Banking
      \item Payment Facilitator (PayFac)
      \item Open Banking (PSD2)
      \item CBDC (Retail / Wholesale)
      \item Stablecoin
      \item Pain of Paying
    \end{itemize}
  \end{multicols}

  \vspace{0.3em}
  \textbf{Next lesson:} \textit{Lecture~4: Fintech Security and Regulation ---
  RegTech} --- AML/KYC automation, regulatory sandboxes, supervisory
  technology, and the global regulatory landscape for fintech.
  \bottomnote{L04 connects L03's payment infrastructure to the regulatory frameworks that govern it. The design principles you learned today apply directly.}
\end{frame}

\end{document}
