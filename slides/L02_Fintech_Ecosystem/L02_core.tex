% ============================================================
%  L02_core.tex  --  Fintech Ecosystem
%  Core Slides (10 frames)
%  Self-contained (no \input{} commands)
%  Compile: pdflatex L02_core.tex  (twice for overlays)
% ============================================================

\documentclass[aspectratio=169, 11pt]{beamer}
\usetheme{Madrid}
\usecolortheme{whale}
\usepackage{tikz,pgfplots,booktabs,multicol,amsmath,graphicx}
\usetikzlibrary{arrows.meta,positioning,shapes.geometric,calc,decorations.pathmorphing}
\pgfplotsset{compat=1.18}

% ---- Colour palette ----------------------------------------
\definecolor{mlpurple}{HTML}{9467BD}
\definecolor{mlblue}{HTML}{1F77B4}
\definecolor{mlred}{HTML}{D62728}
\definecolor{mlorange}{HTML}{FF7F0E}
\definecolor{mlgreen}{HTML}{2CA02C}
\definecolor{mlgray}{HTML}{7F7F7F}
\definecolor{mlteal}{HTML}{0D7377}
\definecolor{mlcyan}{HTML}{14BDEB}

% ---- Beamer colour settings --------------------------------
\setbeamercolor{structure}{fg=mlteal}
\setbeamercolor{palette primary}{bg=mlteal,fg=white}
\setbeamercolor{palette secondary}{bg=mlteal!80,fg=white}
\setbeamercolor{palette tertiary}{bg=mlteal!60,fg=white}
\setbeamercolor{palette quaternary}{bg=mlteal!40,fg=white}
\setbeamercolor{frametitle}{bg=mlteal!10,fg=mlteal}
\setbeamercolor{frametitle right}{bg=mlteal!5}
\setbeamercolor{block title}{bg=mlteal,fg=white}
\setbeamercolor{block body}{bg=mlteal!8,fg=black}
\setbeamercolor{block title alerted}{bg=mlred,fg=white}
\setbeamercolor{block body alerted}{bg=mlred!8,fg=black}
\setbeamercolor{block title example}{bg=mlgreen,fg=white}
\setbeamercolor{block body example}{bg=mlgreen!8,fg=black}
\setbeamercolor{title}{fg=white}
\setbeamercolor{subtitle}{fg=mlcyan}
\setbeamercolor{author}{fg=white}
\setbeamercolor{institute}{fg=white}
\setbeamercolor{date}{fg=white}

% ---- Bottom-note command -----------------------------------
\newcommand{\bottomnote}[1]{%
  \vfill
  \begin{beamercolorbox}[wd=\textwidth,ht=2ex,dp=1ex]{palette primary}%
    \tiny\hspace{1em}#1%
  \end{beamercolorbox}}

% ---- Graphics path -----------------------------------------
\graphicspath{{}}

% ---- Metadata ----------------------------------------------
\title{Fintech Ecosystem}
\subtitle{Core Slides}
\author{Joerg Osterrieder}
\institute{University of Zurich \\ Department of Finance}
\date{Spring 2026}
\setbeamertemplate{navigation symbols}{}

% ============================================================
\begin{document}
% ============================================================

% --- Frame 1: Title Page ---
\begin{frame}{Title Page}
  \titlepage
\end{frame}

% --- Frame 4: Bridge from Lecture 1 ---
\begin{frame}{Bridge from Lecture~1}
  \begin{columns}[T]
    \begin{column}{0.55\textwidth}
      In Lecture~1 we established \textbf{what} fintech is, \textbf{where} it
      came from, and \textbf{how} banks and fintechs collaborate.

      \vspace{0.5em}
      Now we ask the deeper questions:
      \begin{itemize}
        \item \textbf{Who} does fintech serve?
        \item \textbf{Why} do some people adopt it while others resist?
        \item \textbf{How} do product design choices shape financial decisions?
      \end{itemize}

      \vspace{0.5em}
      L02 shifts the lens from \textcolor{mlpurple}{supply-side strategy} to
      \textcolor{mlpurple}{demand-side behavior}.
    \end{column}
    \begin{column}{0.42\textwidth}
      \includegraphics[width=\textwidth]{01_fintech_ecosystem_map/chart.pdf}
    \end{column}
  \end{columns}
  \bottomnote{L01 gave you the supply-side view. L02 gives you the demand-side view.}
\end{frame}

% --- Frame 6: The Fintech Growth Engine ---
\begin{frame}{The Fintech Growth Engine --- Four Drivers}
  \begin{columns}[T]
    \begin{column}{0.50\textwidth}
      \includegraphics[width=\textwidth]{02_growth_drivers_dashboard/chart.pdf}
    \end{column}
    \begin{column}{0.47\textwidth}
      Four forces sustain fintech's growth trajectory:

      \vspace{0.4em}
      \begin{enumerate}
        \item \textcolor{mlpurple}{\textbf{Capital}} --- Venture funding, corporate
              venture arms, public market appetite
        \item \textcolor{mlpurple}{\textbf{Technology}} --- Cloud, APIs, AI/ML,
              biometric authentication
        \item \textcolor{mlpurple}{\textbf{Distribution}} --- Smartphones,
              app stores, social media virality
        \item \textcolor{mlpurple}{\textbf{Demand}} --- Trust erosion in incumbents,
              digital-native expectations, unbanked populations
      \end{enumerate}

      \vspace{0.3em}
      \begin{block}{The Real Question}
        The question is not ``Why is fintech growing?'' but ``Why did it take
        so long to start?''
      \end{block}
    \end{column}
  \end{columns}
  \bottomnote{Global fintech VC investment grew from approximately USD~4B in 2013 to over USD~50B by 2021 before correcting.}
\end{frame}

% --- Frame 10: Trust in Financial Services ---
\begin{frame}{Trust in Financial Services --- A Framework}
  \begin{columns}[T]
    \begin{column}{0.50\textwidth}
      \includegraphics[width=\textwidth]{05_trust_framework_comparison/chart.pdf}
    \end{column}
    \begin{column}{0.47\textwidth}
      \begin{itemize}
        \item \textbf{Trust is multidimensional:} Competence trust (``Can they do
              it?''), benevolence trust (``Do they care about me?''), and
              integrity trust (``Will they be fair?'') operate independently.
        \item \textbf{Provider differences:} Banks score high on competence but
              low on benevolence. Fintechs score high on convenience but low on
              integrity (because they are new and untested).
        \item \textbf{Building strategies:} Banks emphasize stability and
              insurance. Fintechs emphasize transparency, UX quality, and peer
              endorsement.
      \end{itemize}
    \end{column}
  \end{columns}
  \bottomnote{The distinction between calculative trust (rational cost-benefit) and relational trust (emotional bond) explains why switching is hard.}
\end{frame}

% --- Frame 14: Choice Architecture ---
\begin{frame}{Choice Architecture --- Designing Financial Decisions}
  \begin{columns}[T]
    \begin{column}{0.50\textwidth}
      \includegraphics[width=\textwidth]{08_nudging_architecture/chart.pdf}
    \end{column}
    \begin{column}{0.47\textwidth}
      \begin{itemize}
        \item \textbf{Every financial interface is a designed environment.}
              Screen layout, button placement, default selections, and
              information ordering all influence decisions.
        \item \textbf{There is no neutral design.} Presenting three investment
              options or thirty is a choice. Showing returns before fees or
              after fees is a choice. Every design decision is a nudge.
        \item \textbf{Fintech \textit{is} choice architecture.} Unlike a bank
              branch, where a human advisor mediates decisions, a fintech app
              \textit{is} the decision environment.
      \end{itemize}
    \end{column}
  \end{columns}
  \vspace{0.3em}
  \begin{block}{The Designer's Power}
    In fintech, the product designer has more influence over financial decisions
    than the financial advisor ever did.
  \end{block}
  \bottomnote{Thaler and Sunstein, \textit{Nudge} (2008): ``There is no such thing as a neutral design.'' This is the foundational text.}
\end{frame}

% --- Frame 18: The Financial Inclusion Paradox ---
\begin{frame}{The Financial Inclusion Paradox}
  Financial inclusion through fintech creates four categories of risk:

  \vspace{0.5em}
  \begin{itemize}
    \item \textcolor{mlred}{\textbf{Digital divide}} --- Inclusion assumes
          connectivity, smartphones, and digital literacy. Those without them
          are excluded \textit{more} as physical infrastructure closes.
    \item \textcolor{mlred}{\textbf{Predatory inclusion}} --- Giving people
          access to credit they cannot manage is not inclusion. Digital lending
          at 100\%+ APR to vulnerable populations is extraction.
    \item \textcolor{mlred}{\textbf{Over-indebtedness}} --- Frictionless
          borrowing removes the ``cooling off'' period that friction once
          provided. Instant access means instant debt.
    \item \textcolor{mlred}{\textbf{Data exploitation}} --- Alternative credit
          scoring uses personal data in ways consumers neither understand nor
          consent to meaningfully.
  \end{itemize}

  \vspace{0.3em}
  \begin{alertblock}{The Paradox}
    Financial inclusion without consumer protection is not inclusion --- it is
    \textbf{exploitation with better distribution}.
  \end{alertblock}
  \bottomnote{M-Shwari (Kenya) demonstrated both inclusion and risk: default rates exceeded 20\% within two years of launch.}
\end{frame}

% --- Frame 21: Fintech Ecosystem Stakeholder Map ---
\begin{frame}{Fintech Ecosystem Stakeholder Map}
  \begin{columns}[T]
    \begin{column}{0.50\textwidth}
      \includegraphics[width=\textwidth]{10_ecosystem_stakeholder_impact/chart.pdf}
    \end{column}
    \begin{column}{0.47\textwidth}
      The fintech ecosystem is not bilateral (bank vs.\ fintech). It is a
      \textbf{multi-stakeholder system}:

      \vspace{0.4em}
      \begin{itemize}
        \item \textbf{Asymmetric effects:} What benefits consumers (lower fees)
              hurts bank revenue. What helps regulators (transparency) raises
              compliance costs. No policy is universally positive.
        \item \textbf{Interconnected risks:} A fintech failure does not only
              affect its customers --- it cascades through partners, investors,
              and the regulatory ecosystem.
        \item \textbf{Design externalities:} A single app's choice architecture
              sets behavioral norms across the industry.
      \end{itemize}
    \end{column}
  \end{columns}
  \bottomnote{This stakeholder map extends L01's ecosystem overview by adding the behavioral and social dimensions.}
\end{frame}

% --- Frame 24: The Inclusion-Protection Trade-off ---
\begin{frame}{The Inclusion-Protection Trade-off}
  \begin{columns}[T]
    \begin{column}{0.50\textwidth}
      \includegraphics[width=\textwidth]{09_choice_architecture_examples/chart.pdf}
    \end{column}
    \begin{column}{0.47\textwidth}
      A quadrant framework for evaluating fintech outcomes:

      \vspace{0.4em}
      \begin{itemize}
        \item \textbf{Q1: High inclusion, high protection} --- M-Pesa with
              agent dispute resolution. The gold standard.
        \item \textbf{Q2: High inclusion, low protection} --- Predatory
              digital lending. Access without safety nets.
        \item \textbf{Q3: Low inclusion, high protection} --- Traditional
              banking. Safe but exclusionary.
        \item \textbf{Q4: Low inclusion, low protection} --- Unregulated
              crypto in vulnerable markets. The worst outcome.
      \end{itemize}

      \vspace{0.3em}
      \textcolor{mlpurple}{Every fintech product sits in one of these
      quadrants. The goal is Q1.}
    \end{column}
  \end{columns}
  \bottomnote{This quadrant framework is a tool for evaluating any fintech initiative. Use it in Workshop~C.}
\end{frame}

% --- Frame 26: An Ecosystem Evaluation Framework ---
\begin{frame}{An Ecosystem Evaluation Framework}
  \begin{columns}[T]
    \begin{column}{0.55\textwidth}
      Extending L01's five-question framework, ask five more:

      \vspace{0.4em}
      \begin{enumerate}
        \item \textbf{Who is excluded?}\\
              Which populations cannot access or use this product?
        \item \textbf{What behavioral assumptions does it make?}\\
              Does it assume rationality, digital literacy, or trust?
        \item \textbf{How does it nudge?}\\
              What defaults, frames, and social cues does it deploy?
        \item \textbf{What happens when it fails?}\\
              Is there a safety net, or does the user bear all risk?
        \item \textbf{Does it build or erode trust?}\\
              Will this product make users more or less willing to adopt
              the \textit{next} fintech product?
      \end{enumerate}
    \end{column}
    \begin{column}{0.42\textwidth}
      \textbf{Synthesis:}

      \vspace{0.4em}
      L01's framework evaluates \textit{strategy} --- whether a fintech can
      succeed as a business.

      \vspace{0.4em}
      L02's framework evaluates \textit{impact} --- whether a fintech
      \textit{should} succeed as a product.

      \vspace{0.5em}
      \begin{block}{The Combined Test}
        A fintech product that passes L01's strategy test but fails L02's
        ecosystem test may be \textcolor{mlpurple}{profitable but harmful}.
      \end{block}
    \end{column}
  \end{columns}
  \bottomnote{Apply both frameworks together in Workshop~C. A complete evaluation requires both the supply-side and demand-side lens.}
\end{frame}

% --- Frame 30: Key Takeaways ---
\begin{frame}{Key Takeaways}
  \begin{enumerate}
    \item \textbf{Growth engine:} Fintech growth is sustained by four
          interdependent drivers --- capital, technology, distribution, and
          demand. Remove any one and growth stalls.
    \item \textbf{Financial inclusion:} 1.7~billion adults remain unbanked.
          Mobile money (M-Pesa, PIX) proves inclusion is possible; predatory
          lending proves it is not automatic.
    \item \textbf{Trust:} Trust in financial services is multidimensional
          (competence, benevolence, integrity) and asymmetric (slow to build,
          fast to destroy).
    \item \textbf{Behavioral barriers:} Status quo bias, loss aversion, and
          complexity aversion explain most non-adoption --- not lack of features.
    \item \textbf{Choice architecture:} Every fintech product is a designed
          decision environment. Defaults, frames, and social cues shape
          financial behavior more than information does.
    \item \textbf{The ethical line:} The boundary between a helpful nudge and a
          dark pattern is alignment with the user's interest, not the company's
          revenue.
    \item \textbf{Inclusion-protection trade-off:} The goal is Q1 (high
          inclusion, high protection). Most fintech sits in Q2 or Q3. Q4 is
          failure.
  \end{enumerate}
  \bottomnote{Review question: Pick a fintech product you use. Which quadrant does it occupy in the inclusion-protection framework? Why?}
\end{frame}

\end{document}
