% L02_full.tex -- Lecture 2: Fintech Ecosystem (Full Variant)
% Frames: 31 | Charts: 12 \includegraphics | Arc: 10-role
% Generated for: Financial Technology (FinTech) -- MSc Course, Spring 2026
\documentclass[aspectratio=169, 11pt]{beamer}

% ============================================================
% THEME BASE
% ============================================================
\usetheme{Madrid}
\usecolortheme{whale}

% ============================================================
% PACKAGES
% ============================================================
\usepackage[T1]{fontenc}
\usepackage[utf8]{inputenc}
\usepackage{graphicx}
\usepackage{booktabs}
\usepackage{tikz}
\usepackage{pgfplots}
\usepackage{amsmath}
\usepackage{hyperref}
\usepackage{multicol}
\usepackage{xcolor}

% ============================================================
% TIKZ LIBRARIES
% ============================================================
\usetikzlibrary{arrows.meta, positioning, shapes.geometric, calc, decorations.pathmorphing}

% ============================================================
% PGFPLOTS COMPATIBILITY
% ============================================================
\pgfplotsset{compat=1.18}

% ============================================================
% COLOR DEFINITIONS (Fintech V4 Palette)
% ============================================================
\definecolor{MLPURPLE}{HTML}{9467BD}
\definecolor{MLBLUE}{HTML}{1F77B4}
\definecolor{MLRED}{HTML}{D62728}
\definecolor{MLORANGE}{HTML}{FF7F0E}
\definecolor{MLGREEN}{HTML}{2CA02C}
\definecolor{MLGRAY}{HTML}{7F7F7F}
\definecolor{MLTEAL}{HTML}{0D7377}
\definecolor{MLCYAN}{HTML}{14BDEB}

% Lowercase aliases for use in \textcolor{}
\colorlet{mlpurple}{MLPURPLE}
\colorlet{mlblue}{MLBLUE}
\colorlet{mlred}{MLRED}
\colorlet{mlorange}{MLORANGE}
\colorlet{mlgreen}{MLGREEN}
\colorlet{mlgray}{MLGRAY}
\colorlet{mlteal}{MLTEAL}
\colorlet{mlcyan}{MLCYAN}

% ============================================================
% BEAMER COLOR CUSTOMIZATION
% ============================================================
\setbeamercolor{structure}{fg=MLTEAL}
\setbeamercolor{palette primary}{bg=MLTEAL, fg=white}
\setbeamercolor{palette secondary}{bg=MLTEAL!80, fg=white}
\setbeamercolor{palette tertiary}{bg=MLTEAL!60, fg=white}
\setbeamercolor{palette quaternary}{bg=MLTEAL!40, fg=white}
\setbeamercolor{frametitle}{bg=MLTEAL!10, fg=MLTEAL}
\setbeamercolor{frametitle right}{bg=MLTEAL!5}
\setbeamercolor{block title}{bg=MLTEAL, fg=white}
\setbeamercolor{block body}{bg=MLTEAL!8, fg=black}
\setbeamercolor{block title alerted}{bg=MLRED, fg=white}
\setbeamercolor{block body alerted}{bg=MLRED!8, fg=black}
\setbeamercolor{block title example}{bg=MLGREEN, fg=white}
\setbeamercolor{block body example}{bg=MLGREEN!8, fg=black}
\setbeamercolor{title}{fg=white}
\setbeamercolor{subtitle}{fg=MLCYAN}
\setbeamercolor{author}{fg=white}
\setbeamercolor{institute}{fg=white}
\setbeamercolor{date}{fg=white}
\setbeamercolor{title page header}{bg=MLTEAL}

% ============================================================
% NAVIGATION AND FOOTLINE
% ============================================================
\setbeamertemplate{navigation symbols}{}

\setbeamertemplate{footline}{%
  \leavevmode%
  \hbox{%
    \begin{beamercolorbox}[wd=.333\paperwidth, ht=2.25ex, dp=1ex, center]{palette primary}%
      \usebeamerfont{author in head/foot}\insertshortauthor
    \end{beamercolorbox}%
    \begin{beamercolorbox}[wd=.334\paperwidth, ht=2.25ex, dp=1ex, center]{palette secondary}%
      \usebeamerfont{title in head/foot}\insertshorttitle
    \end{beamercolorbox}%
    \begin{beamercolorbox}[wd=.333\paperwidth, ht=2.25ex, dp=1ex, right]{palette tertiary}%
      \usebeamerfont{date in head/foot}%
      \insertframenumber{} / \inserttotalframenumber\hspace*{2ex}
    \end{beamercolorbox}%
  }%
  \vskip0pt%
}

% ============================================================
% BOTTOM NOTE COMMAND
% ============================================================
\newcommand{\bottomnote}[1]{%
  \vfill
  \begin{beamercolorbox}[wd=\textwidth, ht=2ex, dp=1ex]{palette primary}%
    \tiny\hspace{1em}#1
  \end{beamercolorbox}%
}

% ============================================================
% GRAPHICS PATH
% ============================================================
\graphicspath{{figures/}}

% ============================================================
% COURSE METADATA
% ============================================================
\title{Financial Technology (FinTech)}
\author{Joerg Osterrieder}
\institute{University of Zurich \\ Department of Finance}
\date{Spring 2026}


\subtitle{Growth, Social Impact, and Behavioral Dimensions}

\begin{document}

% =============================================================================
% === WHY === (Frames 1--4: Title, Opening Cartoon, Learning Objectives, Bridge)
% =============================================================================

% --- Frame 1: Title Page ---
\begin{frame}{Title Page}
  \titlepage
\end{frame}

% --- Frame 2: Opening Cartoon ---
\begin{frame}{The Most Important Branch}
  \begin{center}
    \includegraphics[width=0.85\textwidth]{11_opening_cartoon/cartoon.pdf}
  \end{center}
  \vspace{-0.5em}
  \begin{center}
    \textit{``The most important bank branch in history fits in your pocket.''}
  \end{center}
  \bottomnote{Approximately 1.7~billion adults worldwide remain unbanked --- yet most of them have access to a mobile phone.}
\end{frame}

% --- Frame 3: Learning Objectives ---
\begin{frame}{Learning Objectives}
  By the end of this lecture you will be able to:
  \begin{enumerate}
    \item \textbf{Identify} the four drivers of fintech growth and explain their
          interdependence. \hfill\textit{[Understand]}
    \item \textbf{Explain} why financial inclusion remains incomplete despite
          technological progress, distinguishing access barriers from behavioral
          barriers. \hfill\textit{[Understand]}
    \item \textbf{Apply} the technology adoption lifecycle to predict which
          fintech products will cross the chasm and which will stall.
          \hfill\textit{[Apply]}
    \item \textbf{Analyze} how choice architecture and nudging mechanisms shape
          financial decisions --- for good and for ill.
          \hfill\textit{[Analyze]}
    \item \textbf{Evaluate} the ethical boundary between helpful nudging and
          manipulative dark patterns in financial product design.
          \hfill\textit{[Evaluate]}
  \end{enumerate}
  \vspace{0.5em}
  \textcolor{mlpurple}{\textbf{Bloom's levels covered:}} Understand, Apply, Analyze, Evaluate
  \bottomnote{These objectives map directly to quiz and exercise assessments.}
\end{frame}

% --- Frame 4: Bridge from Lecture 1 ---
\begin{frame}{Bridge from Lecture~1}
  \begin{columns}[T]
    \begin{column}{0.55\textwidth}
      In Lecture~1 we established \textbf{what} fintech is, \textbf{where} it
      came from, and \textbf{how} banks and fintechs collaborate.

      \vspace{0.5em}
      Now we ask the deeper questions:
      \begin{itemize}
        \item \textbf{Who} does fintech serve?
        \item \textbf{Why} do some people adopt it while others resist?
        \item \textbf{How} do product design choices shape financial decisions?
      \end{itemize}

      \vspace{0.5em}
      L02 shifts the lens from \textcolor{mlpurple}{supply-side strategy} to
      \textcolor{mlpurple}{demand-side behavior}.
    \end{column}
    \begin{column}{0.42\textwidth}
      \includegraphics[width=\textwidth]{01_fintech_ecosystem_map/chart.pdf}
    \end{column}
  \end{columns}
  \bottomnote{L01 gave you the supply-side view. L02 gives you the demand-side view.}
\end{frame}

% =============================================================================
% === FEEL === (Frame 5: Personal Connection)
% =============================================================================

% --- Frame 5: The Nudge in Your Wallet ---
\begin{frame}{The Nudge in Your Wallet}
  Open your banking app right now.

  \vspace{0.5em}
  Look at the home screen. What is the \textbf{default} view --- spending,
  savings, or investments? Who decided that default? What happens if you try to
  \textbf{close your account} --- is it as easy as opening one was? Notice the
  rounding-up feature, the savings goals, the spending categories with traffic-light
  colors. \textcolor{mlpurple}{Every one of those choices is a nudge.}

  \vspace{0.8em}
  \begin{exampleblock}{Quick Exercise}
    Find one nudge in your financial apps --- a default, a prompt, a design choice
    that steers your behavior.
    \begin{itemize}
      \item Is it helping you or helping the company?
      \item Would you behave differently without it?
    \end{itemize}
    Bring your example to the discussion.
  \end{exampleblock}
  \bottomnote{The tension between helpful nudges and dark patterns is the ethical core of this lecture.}
\end{frame}

% =============================================================================
% === WHAT === (Frames 6--9: Foundational Concepts)
% =============================================================================

% --- Frame 6: The Fintech Growth Engine ---
\begin{frame}{The Fintech Growth Engine --- Four Drivers}
  \begin{columns}[T]
    \begin{column}{0.50\textwidth}
      \includegraphics[width=\textwidth]{02_growth_drivers_dashboard/chart.pdf}
    \end{column}
    \begin{column}{0.47\textwidth}
      Four forces sustain fintech's growth trajectory:

      \vspace{0.4em}
      \begin{enumerate}
        \item \textcolor{mlpurple}{\textbf{Capital}} --- Venture funding, corporate
              venture arms, public market appetite
        \item \textcolor{mlpurple}{\textbf{Technology}} --- Cloud, APIs, AI/ML,
              biometric authentication
        \item \textcolor{mlpurple}{\textbf{Distribution}} --- Smartphones,
              app stores, social media virality
        \item \textcolor{mlpurple}{\textbf{Demand}} --- Trust erosion in incumbents,
              digital-native expectations, unbanked populations
      \end{enumerate}

      \vspace{0.3em}
      \begin{block}{The Real Question}
        The question is not ``Why is fintech growing?'' but ``Why did it take
        so long to start?''
      \end{block}
    \end{column}
  \end{columns}
  \bottomnote{Global fintech VC investment grew from approximately USD~4B in 2013 to over USD~50B by 2021 before correcting.}
\end{frame}

% --- Frame 7: Economic Benefits of Fintech ---
\begin{frame}{Economic Benefits of Fintech}
  Fintech delivers measurable economic value across five dimensions:

  \vspace{0.5em}
  \begin{itemize}
    \item \textcolor{mlpurple}{\textbf{Cost reduction through automation}} ---
          Digital onboarding, automated underwriting, and algorithmic compliance
          reduce operating costs by orders of magnitude.
    \item \textcolor{mlpurple}{\textbf{Improved credit access}} --- Alternative
          data scoring (mobile usage, utility payments, social data) extends
          credit to populations invisible to traditional bureaus.
    \item \textcolor{mlpurple}{\textbf{Faster time-to-market}} --- API-first
          architectures let new products launch in weeks, not years.
    \item \textcolor{mlpurple}{\textbf{Market efficiency}} --- Real-time pricing,
          transparent fee structures, and reduced information asymmetry.
    \item \textcolor{mlpurple}{\textbf{New market creation}} --- Micro-insurance,
          micro-investing, and fractional ownership create markets that did not
          previously exist.
  \end{itemize}

  \vspace{0.3em}
  \begin{block}{Beyond Efficiency}
    Fintech's economic contribution is not just making existing services
    cheaper --- it is making previously impossible services possible.
  \end{block}
  \bottomnote{Neobank cost-to-income ratios can be 30--40\%, compared with 55--70\% at traditional banks.}
\end{frame}

% --- Frame 8: Financial Inclusion -- The Unbanked Challenge ---
\begin{frame}{Financial Inclusion --- The Unbanked Challenge}
  \begin{columns}[T]
    \begin{column}{0.50\textwidth}
      \includegraphics[width=\textwidth]{03_financial_inclusion_gap/chart.pdf}
    \end{column}
    \begin{column}{0.47\textwidth}
      \begin{itemize}
        \item \textbf{The gap:} 1.7~billion adults lack access to formal financial
              services. Two-thirds of them are women. Most live in Sub-Saharan
              Africa and South Asia.
        \item \textbf{The paradox:} Mobile phone penetration exceeds bank account
              penetration in nearly every developing economy --- connectivity
              exists, but financial access does not.
        \item \textbf{The behavioral layer:} Even where access exists, trust
              deficits, financial illiteracy, and cultural norms suppress
              adoption.
      \end{itemize}
    \end{column}
  \end{columns}
  \vspace{0.3em}
  \begin{block}{Access Is Necessary but Not Sufficient}
    Providing a product is not the same as achieving inclusion. People must also
    \textcolor{mlpurple}{trust} it, \textcolor{mlpurple}{understand} it, and
    \textcolor{mlpurple}{choose} to use it.
  \end{block}
  \bottomnote{World Bank Global Findex (2021): account ownership rose to 76\% of adults globally, up from 51\% in 2011.}
\end{frame}

% --- Frame 9: M-Pesa -- The Canonical Inclusion Story ---
\begin{frame}{M-Pesa --- The Canonical Inclusion Story}
  \begin{columns}[T]
    \begin{column}{0.50\textwidth}
      \includegraphics[width=\textwidth]{04_mpesa_adoption_flow/chart.pdf}
    \end{column}
    \begin{column}{0.47\textwidth}
      M-Pesa launched in Kenya in 2007 --- not as a bank, but as a
      \textbf{mobile money transfer service}.

      \vspace{0.4em}
      Key ingredients:
      \begin{itemize}
        \item Over 30~million active customers in Kenya alone
        \item 170,000+ agent locations (vs.\ fewer than 2,000 bank branches)
        \item No bank account required --- just a SIM card
        \item Built on \textbf{trust in the agent network}, not trust in banks
      \end{itemize}

      \vspace{0.3em}
      \begin{block}{A New Category}
        M-Pesa did not digitize banking. It invented a new category:
        \textcolor{mlpurple}{mobile money}.
      \end{block}
    \end{column}
  \end{columns}
  \bottomnote{M-Pesa processes more transactions domestically than all other payment systems combined in Kenya.}
\end{frame}

% =============================================================================
% === CASE === (Frames 10--13: Behavioral Economics of Fintech Adoption)
% =============================================================================

% --- Frame 10: Trust in Financial Services ---
\begin{frame}{Trust in Financial Services --- A Framework}
  \begin{columns}[T]
    \begin{column}{0.50\textwidth}
      \includegraphics[width=\textwidth]{05_trust_framework_comparison/chart.pdf}
    \end{column}
    \begin{column}{0.47\textwidth}
      \begin{itemize}
        \item \textbf{Trust is multidimensional:} Competence trust (``Can they do
              it?''), benevolence trust (``Do they care about me?''), and
              integrity trust (``Will they be fair?'') operate independently.
        \item \textbf{Provider differences:} Banks score high on competence but
              low on benevolence. Fintechs score high on convenience but low on
              integrity (because they are new and untested).
        \item \textbf{Building strategies:} Banks emphasize stability and
              insurance. Fintechs emphasize transparency, UX quality, and peer
              endorsement.
      \end{itemize}
    \end{column}
  \end{columns}
  \bottomnote{The distinction between calculative trust (rational cost-benefit) and relational trust (emotional bond) explains why switching is hard.}
\end{frame}

% --- Frame 11: Why People Resist New Financial Technology ---
\begin{frame}{Why People Resist New Financial Technology}
  \begin{columns}[T]
    \begin{column}{0.50\textwidth}
      The biggest competitor for any fintech product is not another fintech.
      It is \textcolor{mlpurple}{the user's current behavior}.

      \vspace{0.5em}
      Five behavioral barriers explain most non-adoption:
    \end{column}
    \begin{column}{0.47\textwidth}
      \begin{enumerate}
        \item \textbf{Status quo bias} --- ``My current bank is fine.'' The
              default always has an advantage.
        \item \textbf{Loss aversion} --- The pain of a potential loss (data
              breach, lost funds) outweighs the gain of better features.
        \item \textbf{Ambiguity aversion} --- Unknown risks feel worse than
              known risks. ``At least I know what my bank will do.''
        \item \textbf{Social proof dependency} --- ``Nobody I know uses it
              yet.'' Adoption requires visible peers.
        \item \textbf{Complexity aversion} --- If onboarding takes more than
              three minutes, most people quit.
      \end{enumerate}
    \end{column}
  \end{columns}
  \vspace{0.3em}
  \begin{block}{The Implication for Design}
    The biggest competitor for any fintech product is not another fintech. It is
    the user's current behavior.
  \end{block}
  \bottomnote{Kahneman and Tversky's prospect theory (1979) established that losses loom roughly twice as large as equivalent gains.}
\end{frame}

% --- Frame 12: The Technology Adoption Lifecycle ---
\begin{frame}{The Technology Adoption Lifecycle Applied to Fintech}
  \begin{columns}[T]
    \begin{column}{0.50\textwidth}
      \includegraphics[width=\textwidth]{06_technology_adoption_lifecycle/chart.pdf}
    \end{column}
    \begin{column}{0.47\textwidth}
      \begin{itemize}
        \item \textbf{Innovators} (2.5\%) --- Crypto early miners, DeFi
              experimenters. Motivated by novelty.
        \item \textbf{Early Adopters} (13.5\%) --- Neobank first users.
              Motivated by advantage over incumbents.
        \item \textbf{Early Majority} (34\%) --- Mainstream mobile banking
              users. Need social proof and low friction.
        \item \textbf{Late Majority} (34\%) --- Adopt only when the old
              option disappears. Need institutional endorsement.
        \item \textbf{Laggards} (16\%) --- Cash-only, branch-dependent. Adopt
              only under duress.
      \end{itemize}
    \end{column}
  \end{columns}
  \vspace{0.3em}
  \begin{block}{The Chasm}
    The gap between Early Adopters and Early Majority --- Geoffrey Moore's
    ``chasm'' --- is where most fintech products die.
    \textcolor{mlpurple}{Crossing it requires trust, not just features.}
  \end{block}
  \bottomnote{Geoffrey Moore, \textit{Crossing the Chasm} (1991). The model explains why many technically superior products fail commercially.}
\end{frame}

% --- Frame 13: Risk Aversion Across Demographics ---
\begin{frame}{Risk Aversion Across Demographics}
  \begin{columns}[T]
    \begin{column}{0.50\textwidth}
      \includegraphics[width=\textwidth]{07_adoption_barriers_matrix/chart.pdf}
    \end{column}
    \begin{column}{0.47\textwidth}
      Risk aversion toward financial technology is not uniform:

      \vspace{0.4em}
      \begin{itemize}
        \item \textbf{Age:} Older adults show higher aversion to digital-only
              providers. Trust in physical branches remains strong.
        \item \textbf{Income:} Low-income users face higher stakes per
              transaction. A single error matters more.
        \item \textbf{Geography:} Urban populations adopt faster due to network
              effects and peer visibility.
        \item \textbf{Digital literacy:} Smartphone ownership alone does not
              predict adoption. Comfort with digital interfaces does.
      \end{itemize}

      \vspace{0.3em}
      \textcolor{mlpurple}{One-size-fits-all fintech design systematically
      excludes the most vulnerable users.}
    \end{column}
  \end{columns}
  \bottomnote{Risk aversion correlates strongly with age and inversely with smartphone fluency, not merely smartphone ownership.}
\end{frame}

% =============================================================================
% === HOW === (Frames 14--17: Choice Architecture and Nudging)
% =============================================================================

% --- Frame 14: Choice Architecture ---
\begin{frame}{Choice Architecture --- Designing Financial Decisions}
  \begin{columns}[T]
    \begin{column}{0.50\textwidth}
      \includegraphics[width=\textwidth]{08_nudging_architecture/chart.pdf}
    \end{column}
    \begin{column}{0.47\textwidth}
      \begin{itemize}
        \item \textbf{Every financial interface is a designed environment.}
              Screen layout, button placement, default selections, and
              information ordering all influence decisions.
        \item \textbf{There is no neutral design.} Presenting three investment
              options or thirty is a choice. Showing returns before fees or
              after fees is a choice. Every design decision is a nudge.
        \item \textbf{Fintech \textit{is} choice architecture.} Unlike a bank
              branch, where a human advisor mediates decisions, a fintech app
              \textit{is} the decision environment.
      \end{itemize}
    \end{column}
  \end{columns}
  \vspace{0.3em}
  \begin{block}{The Designer's Power}
    In fintech, the product designer has more influence over financial decisions
    than the financial advisor ever did.
  \end{block}
  \bottomnote{Thaler and Sunstein, \textit{Nudge} (2008): ``There is no such thing as a neutral design.'' This is the foundational text.}
\end{frame}

% --- Frame 15: Five Nudges That Shape Financial Behavior ---
\begin{frame}{Five Nudges That Shape Financial Behavior}
  \begin{enumerate}
    \item \textcolor{mlpurple}{\textbf{Default settings}} --- Auto-enrollment in
          savings plans increases participation from approximately 40\% (opt-in)
          to over 90\% (opt-out). The default \textit{is} the decision for most
          people.
    \item \textcolor{mlpurple}{\textbf{Framing effects}} --- ``You pay
          CHF~47~per~month'' vs.\ ``This costs 3.2\% of your portfolio.'' Same
          fact, different decisions.
    \item \textcolor{mlpurple}{\textbf{Social proof}} --- ``87\% of users your
          age have started saving.'' Peer comparison is the most powerful
          motivator for financial behavior change.
    \item \textcolor{mlpurple}{\textbf{Commitment devices}} --- Savings lock-ups,
          goal-setting features, and voluntary restrictions exploit the gap
          between present and future selves.
    \item \textcolor{mlpurple}{\textbf{Simplification}} --- Reducing choices from
          40 options to 3 increases decision quality and completion rates.
  \end{enumerate}
  \vspace{0.3em}
  \begin{block}{Tools, Not Answers}
    Each nudge is a tool. Tools can build houses or break them.
  \end{block}
  \bottomnote{Madrian and Shea (2001): automatic enrollment in 401(k) plans raised participation from 49\% to 86\% --- the canonical nudge study.}
\end{frame}

% --- Frame 16: Dark Patterns ---
\begin{frame}{Dark Patterns --- When Nudging Goes Wrong}
  \begin{columns}[T]
    \begin{column}{0.50\textwidth}
      Five dark patterns common in financial apps:

      \vspace{0.4em}
      \begin{enumerate}
        \item \textcolor{mlred}{\textbf{Hidden fees}} --- Costs buried in
              scrollable terms, revealed only at checkout.
        \item \textcolor{mlred}{\textbf{Confirm-shaming}} --- ``No thanks, I
              don't want to save money.'' Guilt-driven opt-out language.
        \item \textcolor{mlred}{\textbf{Roach motel}} --- Easy to sign up,
              deliberately difficult to close an account or cancel.
        \item \textcolor{mlred}{\textbf{Urgency manipulation}} --- ``Only 2 hours
              left!'' applied to investment decisions.
        \item \textcolor{mlred}{\textbf{Default opt-in}} --- Pre-checked boxes
              for premium services, overdraft ``protection,'' data sharing.
      \end{enumerate}
    \end{column}
    \begin{column}{0.47\textwidth}
      \textbf{Where is the ethical line?}

      \vspace{0.4em}
      The difference between a nudge and a dark pattern is
      \textcolor{mlpurple}{alignment with the user's interest}.

      \vspace{0.4em}
      A nudge that helps users save more is ethical. A nudge that tricks users
      into spending more is not.

      \vspace{0.4em}
      But the line is blurry: is rounding up purchases to save the ``spare
      change'' a helpful nudge or a way to make users forget they are spending?

      \vspace{0.3em}
      \begin{alertblock}{Trust Erosion}
        Dark patterns erode the trust that fintech needs to cross the
        adoption chasm.
      \end{alertblock}
    \end{column}
  \end{columns}
  \bottomnote{The EU Digital Services Act (2022) and proposed AI Act explicitly target manipulative design patterns in digital services.}
\end{frame}

% --- Frame 17: Ethical Choice Architecture ---
\begin{frame}{Ethical Choice Architecture --- A Design Checklist}
  \begin{columns}[T]
    \begin{column}{0.55\textwidth}
      Five principles for ethical choice architecture:

      \vspace{0.4em}
      \begin{enumerate}
        \item \textcolor{mlpurple}{\textbf{Transparency}} --- Users can see
              \textit{that} they are being nudged and \textit{how}.
        \item \textcolor{mlpurple}{\textbf{Reversibility}} --- Every default can
              be changed. Every choice can be undone.
        \item \textcolor{mlpurple}{\textbf{Alignment}} --- The nudge serves the
              user's stated goals, not the company's revenue targets.
        \item \textcolor{mlpurple}{\textbf{Disclosure}} --- Conflicts of interest
              are visible. Referral fees, commissions, and incentives are shown.
        \item \textcolor{mlpurple}{\textbf{Optionality}} --- Users always have a
              clear path to ``none of the above.''
      \end{enumerate}
    \end{column}
    \begin{column}{0.42\textwidth}
      \textbf{Thaler's Public Defense Test:}

      \vspace{0.4em}
      ``Could you defend this design choice on the front page of a newspaper?''

      \vspace{0.4em}
      If the answer is no --- or if you hesitate --- the nudge has crossed from
      architecture into manipulation.

      \vspace{0.5em}
      \begin{block}{Design Responsibility}
        Fintech companies have a unique design responsibility: they are
        simultaneously the \textcolor{mlpurple}{advisor}, the
        \textcolor{mlpurple}{product}, and the
        \textcolor{mlpurple}{environment} in which financial decisions occur.
      \end{block}
    \end{column}
  \end{columns}
  \bottomnote{OECD (2023), \textit{Recommendation on Financial Consumer Protection}: member countries should ensure digital interfaces do not exploit behavioral biases.}
\end{frame}

% =============================================================================
% === RISK === (Frames 18--20: What Can Go Wrong)
% =============================================================================

% --- Frame 18: The Financial Inclusion Paradox ---
\begin{frame}{The Financial Inclusion Paradox}
  Financial inclusion through fintech creates four categories of risk:

  \vspace{0.5em}
  \begin{itemize}
    \item \textcolor{mlred}{\textbf{Digital divide}} --- Inclusion assumes
          connectivity, smartphones, and digital literacy. Those without them
          are excluded \textit{more} as physical infrastructure closes.
    \item \textcolor{mlred}{\textbf{Predatory inclusion}} --- Giving people
          access to credit they cannot manage is not inclusion. Digital lending
          at 100\%+ APR to vulnerable populations is extraction.
    \item \textcolor{mlred}{\textbf{Over-indebtedness}} --- Frictionless
          borrowing removes the ``cooling off'' period that friction once
          provided. Instant access means instant debt.
    \item \textcolor{mlred}{\textbf{Data exploitation}} --- Alternative credit
          scoring uses personal data in ways consumers neither understand nor
          consent to meaningfully.
  \end{itemize}

  \vspace{0.3em}
  \begin{alertblock}{The Paradox}
    Financial inclusion without consumer protection is not inclusion --- it is
    \textbf{exploitation with better distribution}.
  \end{alertblock}
  \bottomnote{M-Shwari (Kenya) demonstrated both inclusion and risk: default rates exceeded 20\% within two years of launch.}
\end{frame}

% --- Frame 19: Trust Fragility in Digital Finance ---
\begin{frame}{Trust Fragility in Digital Finance}
  \begin{columns}[T]
    \begin{column}{0.50\textwidth}
      Digital trust is \textcolor{mlpurple}{asymmetric}: it takes years to build
      and seconds to destroy.

      \vspace{0.5em}
      Unlike a branch bank, where trust is mediated by a human relationship, a
      fintech's trust rests entirely on:
      \begin{itemize}
        \item App reliability
        \item Transparent communication
        \item Brand reputation
        \item Regulatory endorsement
      \end{itemize}
    \end{column}
    \begin{column}{0.47\textwidth}
      Four factors amplify trust fragility in digital finance:

      \vspace{0.4em}
      \begin{enumerate}
        \item \textbf{No physical presence} --- No branch to visit when
              something goes wrong.
        \item \textbf{Deposit insurance gaps} --- Many fintechs hold funds
              outside traditional insurance schemes.
        \item \textbf{Viral reputation risk} --- A single outage or scandal
              spreads instantly on social media.
        \item \textbf{Regulatory uncertainty} --- Licensing changes can make a
              legal product illegal overnight.
      \end{enumerate}
    \end{column}
  \end{columns}
  \vspace{0.3em}
  \begin{block}{The Speed Asymmetry}
    The speed of digital trust destruction exceeds the speed of digital trust
    construction by an order of magnitude.
  \end{block}
  \bottomnote{The SVB collapse (2023) demonstrated how social-media-amplified bank runs can destroy institutional trust in hours, not weeks.}
\end{frame}

% --- Frame 20: Behavioral Manipulation at Scale ---
\begin{frame}{Behavioral Manipulation at Scale}
  \begin{columns}[T]
    \begin{column}{0.50\textwidth}
      \vspace{-0.5em}
      \begin{tabular}{@{}l p{2.8cm} p{2.8cm}@{}}
        \toprule
        \textbf{Mechanism} & \textbf{Beneficial Use} & \textbf{Harmful Use} \\
        \midrule
        Defaults    & Auto-save 10\%      & Auto-opt into overdraft \\
        Framing     & Show total cost      & Hide fees in fine print \\
        Social proof & ``Peers save more'' & ``Everyone is buying crypto'' \\
        Urgency     & Tax deadline reminder & ``Offer expires in 5\,min'' \\
        Simplification & 3 clear plans    & Hide the free option \\
        \bottomrule
      \end{tabular}
    \end{column}
    \begin{column}{0.47\textwidth}
      Every nudging mechanism is \textbf{dual-use}.

      \vspace{0.4em}
      The same technique that helps one user save more helps another user
      overspend. Scale amplifies both outcomes: a dark pattern in an app with
      50~million users causes 50~million instances of harm.

      \vspace{0.4em}
      \textcolor{mlpurple}{The ethical question is not whether to nudge --- it
      is whom the nudge serves.}
    \end{column}
  \end{columns}
  \bottomnote{UK FCA Consumer Duty (2023): firms must ``act to deliver good outcomes for retail customers'' --- explicitly targeting behaviorally exploitative designs.}
\end{frame}

% =============================================================================
% === WHERE === (Frames 21--23: Evidence at Scale)
% =============================================================================

% --- Frame 21: Fintech Ecosystem Stakeholder Map ---
\begin{frame}{Fintech Ecosystem Stakeholder Map}
  \begin{columns}[T]
    \begin{column}{0.50\textwidth}
      \includegraphics[width=\textwidth]{10_ecosystem_stakeholder_impact/chart.pdf}
    \end{column}
    \begin{column}{0.47\textwidth}
      The fintech ecosystem is not bilateral (bank vs.\ fintech). It is a
      \textbf{multi-stakeholder system}:

      \vspace{0.4em}
      \begin{itemize}
        \item \textbf{Asymmetric effects:} What benefits consumers (lower fees)
              hurts bank revenue. What helps regulators (transparency) raises
              compliance costs. No policy is universally positive.
        \item \textbf{Interconnected risks:} A fintech failure does not only
              affect its customers --- it cascades through partners, investors,
              and the regulatory ecosystem.
        \item \textbf{Design externalities:} A single app's choice architecture
              sets behavioral norms across the industry.
      \end{itemize}
    \end{column}
  \end{columns}
  \bottomnote{This stakeholder map extends L01's ecosystem overview by adding the behavioral and social dimensions.}
\end{frame}

% --- Frame 22: Financial Inclusion Success Stories and Failures ---
\begin{frame}{Financial Inclusion --- Success Stories and Cautionary Tales}
  \begin{columns}[T]
    \begin{column}{0.47\textwidth}
      \textcolor{mlpurple}{\textbf{Success Stories:}}
      \begin{itemize}
        \item \textbf{M-Pesa} (Kenya) --- Mobile money for 30M+ users
              without bank accounts
        \item \textbf{PIX} (Brazil) --- Instant payments reaching 140M+
              users in two years, government-driven
        \item \textbf{Jan Dhan Yojana} (India) --- 500M+ bank accounts
              opened via national campaign + Aadhaar ID
        \item \textbf{GCash} (Philippines) --- Mobile wallet reaching
              rural populations via agent network
      \end{itemize}
    \end{column}
    \begin{column}{0.50\textwidth}
      \textcolor{mlred}{\textbf{Cautionary Tales:}}
      \begin{itemize}
        \item \textbf{Micro-lending traps} --- Apps offering instant loans
              at predatory rates in East Africa and South Asia
        \item \textbf{Crypto inclusion narrative} --- ``Banking the
              unbanked'' claims masking speculative products
        \item \textbf{Aadhaar exclusion} --- Biometric failures denying
              benefits to the most vulnerable
        \item \textbf{Predatory BNPL} --- Buy-now-pay-later enabling
              debt spirals among young consumers
      \end{itemize}
    \end{column}
  \end{columns}
  \vspace{0.3em}
  \begin{block}{The Pattern}
    Every inclusion success shares three traits: local context awareness,
    trust infrastructure, and regulatory support. Every failure lacks at
    least one.
  \end{block}
  \bottomnote{Success and failure often coexist in the same market. Kenya has both M-Pesa (success) and predatory digital lending (failure).}
\end{frame}

% --- Frame 23: Behavioral Nudging at National Scale ---
\begin{frame}{Behavioral Nudging at National Scale}
  \begin{columns}[T]
    \begin{column}{0.31\textwidth}
      \begin{block}{\centering UK Nudge Unit}
        The Behavioural Insights Team (est.\ 2010) tested financial nudges at
        population scale.

        \vspace{0.3em}
        Tax payment reminders with social norms increased collection by
        15 percentage points.

        \vspace{0.3em}
        \textit{Lesson:} Government can nudge at scale.
      \end{block}
    \end{column}
    \begin{column}{0.31\textwidth}
      \begin{block}{\centering US 401(k) Defaults}
        The Pension Protection Act (2006) permitted auto-enrollment as default.

        \vspace{0.3em}
        Participation rates rose from approximately 50\% to 90\% with no
        change in plan design.

        \vspace{0.3em}
        \textit{Lesson:} Defaults are the most powerful nudge.
      \end{block}
    \end{column}
    \begin{column}{0.31\textwidth}
      \begin{block}{\centering India: Jan Dhan + UPI}
        Account creation (Jan Dhan) combined with instant payment rails (UPI)
        created inclusion infrastructure.

        \vspace{0.3em}
        UPI processes over 10~billion transactions per month.

        \vspace{0.3em}
        \textit{Lesson:} Infrastructure is the ultimate nudge.
      \end{block}
    \end{column}
  \end{columns}
  \vspace{0.3em}
  \begin{center}
    \textcolor{mlpurple}{\textbf{When nudges are embedded in national infrastructure,
    they become invisible --- and irresistible.}}
  \end{center}
  \bottomnote{Infrastructure-level nudges (auto-enrollment, default payment rails) are orders of magnitude more powerful than app-level nudges.}
\end{frame}

% =============================================================================
% === IMPACT === (Frames 24--25: Who Wins, Who Loses)
% =============================================================================

% --- Frame 24: The Inclusion-Protection Trade-off ---
\begin{frame}{The Inclusion-Protection Trade-off}
  \begin{columns}[T]
    \begin{column}{0.50\textwidth}
      \includegraphics[width=\textwidth]{09_choice_architecture_examples/chart.pdf}
    \end{column}
    \begin{column}{0.47\textwidth}
      A quadrant framework for evaluating fintech outcomes:

      \vspace{0.4em}
      \begin{itemize}
        \item \textbf{Q1: High inclusion, high protection} --- M-Pesa with
              agent dispute resolution. The gold standard.
        \item \textbf{Q2: High inclusion, low protection} --- Predatory
              digital lending. Access without safety nets.
        \item \textbf{Q3: Low inclusion, high protection} --- Traditional
              banking. Safe but exclusionary.
        \item \textbf{Q4: Low inclusion, low protection} --- Unregulated
              crypto in vulnerable markets. The worst outcome.
      \end{itemize}

      \vspace{0.3em}
      \textcolor{mlpurple}{Every fintech product sits in one of these
      quadrants. The goal is Q1.}
    \end{column}
  \end{columns}
  \bottomnote{This quadrant framework is a tool for evaluating any fintech initiative. Use it in Workshop~C.}
\end{frame}

% --- Frame 25: Who Benefits Most from Behavioral Fintech? ---
\begin{frame}{Who Benefits Most from Behavioral Fintech?}
  \begin{columns}[T]
    \begin{column}{0.50\textwidth}
      Behavioral fintech is not equally valuable to all users. Its benefits
      concentrate among populations with the most to gain from better
      decision environments:

      \vspace{0.5em}
      \begin{itemize}
        \item \textbf{Low-income users:} Auto-savings, spending alerts, and
              budgeting tools have disproportionate impact when margins are thin.
        \item \textbf{Young adults:} First-time financial decision-makers
              benefit most from guided defaults and simplification.
      \end{itemize}
    \end{column}
    \begin{column}{0.47\textwidth}
      \begin{itemize}
        \item \textbf{Elderly users:} Fraud detection, simplified interfaces,
              and proactive alerts protect against exploitation.
        \item \textbf{Small businesses:} Automated invoicing, cash flow
              forecasting, and simplified tax tools reduce administrative
              burden.
      \end{itemize}

      \vspace{0.5em}
      \begin{block}{Force Multiplier}
        Behavioral fintech is a force multiplier: it amplifies good decisions
        for those who need the most help --- \textcolor{mlpurple}{but only if
        designed with their constraints in mind}.
      \end{block}
    \end{column}
  \end{columns}
  \bottomnote{The Robinhood/GameStop episode (2021) demonstrated that behavioral fintech can also amplify harmful decisions when gamification meets speculation.}
\end{frame}

% =============================================================================
% === SO WHAT === (Frames 26--27: Synthesis and Evaluation)
% =============================================================================

% --- Frame 26: An Ecosystem Evaluation Framework ---
\begin{frame}{An Ecosystem Evaluation Framework}
  \begin{columns}[T]
    \begin{column}{0.55\textwidth}
      Extending L01's five-question framework, ask five more:

      \vspace{0.4em}
      \begin{enumerate}
        \item \textbf{Who is excluded?}\\
              Which populations cannot access or use this product?
        \item \textbf{What behavioral assumptions does it make?}\\
              Does it assume rationality, digital literacy, or trust?
        \item \textbf{How does it nudge?}\\
              What defaults, frames, and social cues does it deploy?
        \item \textbf{What happens when it fails?}\\
              Is there a safety net, or does the user bear all risk?
        \item \textbf{Does it build or erode trust?}\\
              Will this product make users more or less willing to adopt
              the \textit{next} fintech product?
      \end{enumerate}
    \end{column}
    \begin{column}{0.42\textwidth}
      \textbf{Synthesis:}

      \vspace{0.4em}
      L01's framework evaluates \textit{strategy} --- whether a fintech can
      succeed as a business.

      \vspace{0.4em}
      L02's framework evaluates \textit{impact} --- whether a fintech
      \textit{should} succeed as a product.

      \vspace{0.5em}
      \begin{block}{The Combined Test}
        A fintech product that passes L01's strategy test but fails L02's
        ecosystem test may be \textcolor{mlpurple}{profitable but harmful}.
      \end{block}
    \end{column}
  \end{columns}
  \bottomnote{Apply both frameworks together in Workshop~C. A complete evaluation requires both the supply-side and demand-side lens.}
\end{frame}

% --- Frame 27: The Central Tension Revisited ---
\begin{frame}{The Central Tension Revisited}
  This lecture has circled a single tension:

  \vspace{0.5em}
  Fintech has the \textbf{tools} to include the excluded, empower the
  underserved, and improve financial decisions at scale. But the same tools
  can exclude, exploit, and manipulate.

  \vspace{0.5em}
  The difference is not the technology. The difference is the
  \textcolor{mlpurple}{\textbf{design choices}} --- the defaults, the
  frames, the incentives, and the governance structures that shape how
  technology meets behavior.

  \vspace{0.5em}
  Every fintech product embeds a theory of its user. The question is
  whether that theory respects the user's autonomy or exploits the user's
  biases.

  \vspace{0.5em}
  \begin{block}{The Thesis of Lecture~2}
    Fintech is not a technology problem with a technology solution. It is a
    \textcolor{mlpurple}{design problem with a behavioral solution}.
  \end{block}
  \bottomnote{L03 (Payments) will show these principles in action: how payment system design shapes spending behavior, merchant economics, and national policy.}
\end{frame}

% =============================================================================
% === ACT === (Frame 28: Forward Look)
% =============================================================================

% --- Frame 28: What Comes Next ---
\begin{frame}{What Comes Next}
  \begin{itemize}
    \item \textbf{Next:} L03 (Payments and Digital Money) --- real-time payments,
          CBDC design, cross-border flows, and the behavioral economics of
          spending
    \item \textbf{Before L03, reflect:} Think about a financial decision you made
          recently. Was it shaped by a default, a frame, or a nudge? Would you
          have decided differently in a different interface?
    \item \textbf{Workshop preparation:} Review the inclusion-protection quadrant
          (Frame~24). You will use it to evaluate a case study in Workshop~C.
  \end{itemize}

  \vspace{0.8em}
  \begin{block}{Why Payments Matter}
    Payments are where behavioral fintech meets everyday life.
    \textcolor{mlpurple}{Every payment interface is a choice architecture.}
    L03 shows you how.
  \end{block}

  \vspace{0.5em}
  \begin{exampleblock}{Course Progress}
    \small
    L01: Foundations~\checkmark \(\bullet\) \textbf{L02: Ecosystem}~\checkmark
    \(\bullet\) L03: Payments \(\bullet\) L04: Regulation \(\bullet\)
    L05: Wealth Mgmt \(\bullet\) L06: Insurance \(\bullet\) L07: Technology
  \end{exampleblock}
  \bottomnote{L03 begins with the question: ``Why does paying with a card feel different from paying with cash?'' --- a behavioral question with trillion-dollar consequences.}
\end{frame}

% =============================================================================
% === CLOSING === (Frames 29--31: Closing Cartoon, Key Takeaways, Summary)
% =============================================================================

% --- Frame 29: Closing Cartoon ---
\begin{frame}{The Choice Architect}
  \begin{center}
    \includegraphics[width=0.85\textwidth]{12_closing_cartoon/cartoon.pdf}
  \end{center}
  \vspace{-0.5em}
  \begin{center}
    \textit{``We didn't change the options. We just changed which one was pre-selected.''}
  \end{center}
  \bottomnote{Choice architecture is invisible power. The most influential financial decisions are the ones users never realize they are making.}
\end{frame}

% --- Frame 30: Key Takeaways ---
\begin{frame}{Key Takeaways}
  \begin{enumerate}
    \item \textbf{Growth engine:} Fintech growth is sustained by four
          interdependent drivers --- capital, technology, distribution, and
          demand. Remove any one and growth stalls.
    \item \textbf{Financial inclusion:} 1.7~billion adults remain unbanked.
          Mobile money (M-Pesa, PIX) proves inclusion is possible; predatory
          lending proves it is not automatic.
    \item \textbf{Trust:} Trust in financial services is multidimensional
          (competence, benevolence, integrity) and asymmetric (slow to build,
          fast to destroy).
    \item \textbf{Behavioral barriers:} Status quo bias, loss aversion, and
          complexity aversion explain most non-adoption --- not lack of features.
    \item \textbf{Choice architecture:} Every fintech product is a designed
          decision environment. Defaults, frames, and social cues shape
          financial behavior more than information does.
    \item \textbf{The ethical line:} The boundary between a helpful nudge and a
          dark pattern is alignment with the user's interest, not the company's
          revenue.
    \item \textbf{Inclusion-protection trade-off:} The goal is Q1 (high
          inclusion, high protection). Most fintech sits in Q2 or Q3. Q4 is
          failure.
  \end{enumerate}
  \bottomnote{Review question: Pick a fintech product you use. Which quadrant does it occupy in the inclusion-protection framework? Why?}
\end{frame}

% --- Frame 31: Summary / Next Lesson ---
\begin{frame}{Summary and Key Vocabulary}
  \textbf{Summary:} The fintech ecosystem is shaped not only by technology and
  capital but by \textcolor{mlpurple}{human behavior} --- trust, risk aversion,
  cognitive biases, and the design of decision environments. Financial inclusion
  requires more than access: it requires products designed with behavioral
  realism, ethical nudging, and consumer protection. The central lesson of L02
  is that fintech's impact depends less on what technology can do and more on
  \textit{how product designers choose to deploy it}.

  \vspace{0.5em}
  \textbf{Key Vocabulary:}
  \begin{multicols}{2}
    \begin{itemize}
      \item Financial Inclusion
      \item Choice Architecture
      \item Nudge / Dark Pattern
      \item Status Quo Bias
      \item Loss Aversion
      \item Technology Adoption Lifecycle
      \item Mobile Money
      \item Social Proof
      \item Commitment Device
      \item Behavioral Fintech
    \end{itemize}
  \end{multicols}

  \vspace{0.3em}
  \textbf{Next lesson:} \textit{Lecture~3: Payments and Digital Money} ---
  Real-time payments, cross-border flows, CBDC design, and the behavioral
  economics of spending vs.\ saving.
  \bottomnote{L03 connects L02's behavioral framework to the largest fintech vertical: payments. The design principles you learned today apply directly.}
\end{frame}

\end{document}
